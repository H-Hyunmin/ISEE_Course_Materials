%!TeX program = xelatex
\documentclass[12pt,hyperref,a4paper,UTF8]{ctexart}
\usepackage{zjureport}
\usepackage{listings}
\usepackage{enumitem}
\usepackage{float}
\usepackage{xcolor}
\lstset{
    %backgroundcolor=\color{red!50!green!50!blue!50},%代码块背景色为浅灰色
    rulesepcolor= \color{gray}, %代码块边框颜色
    breaklines=true,  %代码过长则换行
    numbers=left, %行号在左侧显示
    numberstyle= \small,%行号字体
    keywordstyle= \color{blue},%关键字颜色
    commentstyle=\color{gray}, %注释颜色
    frame=shadowbox%用方框框住代码块
    basicstyle=\small\ttfamily % 设置代码块的基本字体样式
    }


%%-------------------------------正文开始---------------------------%%
\begin{document}

%%-----------------------封面--------------------%%
\cover

%%------------------摘要-------------%%
%\begin{abstract}
%
%在此填写摘要内容
%
%\end{abstract}

\thispagestyle{empty} % 首页不显示页码

%%--------------------------目录页------------------------%%
\newpage
\tableofcontents

%%------------------------正文页从这里开始-------------------%
\newpage

%%可选择这里也放一个标题
%\begin{center}
%    \title{ \Huge \textbf{{标题}}}
%\end{center}

\section{项目介绍}
在本次编程训练中,将从软硬件协同设计的思想出发,利用拓展指令实现特
定程序的优化。你的任务是根据拓展指令集,完成仿真器编码、译码与执行过程
中的拓展内容,从而得到一个功能完整的指令集仿真器。然后,在此基础上,修
改汇编代码添加拓展指令,利用仿真器进行仿真,比较拓展指令减少的指令条目,
优化性能。

\section{项目任务与要求}

\subsection*{任务1:仿真器拓展}
完成以下 R-Type 指令的拓展:

  \begin{figure}[H]
      \centering
      \includegraphics[width =1.0\textwidth]{figures/fig/image1.png}
      \caption{R-Type 拓展指令}
  \end{figure}

\subsection*{任务2:优化 SM3 加密算法}
SM3 算法是散列算法的一种,是我国国家密码管理局编制的商用算法。 为
满足电子认证服务系统等应用的需求,国家密码管理局在 2010 年 12 月发布了
SM3 算法,用于密码应用中的数字签名和验证、消息认证码的生成与验证以及随
机数的生成,可满足多种密码应用的安全需求。

在 /work/task2 文件夹中提供了实现 SM3 算法的 c 文件 sm3.c,汇
编文件 sm3.s。

项目具体任务如下:


\begin{enumerate}
    \item 利用拓展指令 rotleft、 rotright、 reverse、 notand,对照 sm3.c 
修改 sm3.s 汇编文件,将原来汇编文件中的部分指令等价转换为拓展指令,对
SM3 算法实现指令级的优化, 并将修改后的结果保存到 sm3opt.s 中。实验重
点关注 sm3\_str\_summ()函数。
    \item 在对比过程中, 你需要分别输入”Zhejiang University”、 ”COD”、 你的学
号,计算其哈希值,在验证你汇编的正确性同时比较优化效果。
    \item 最后,需要按照此前的设计思路, 在已提供的五条拓展指令以外, 至少
自主设计并完成一条拓展指令, 同时对仿真器和哈希加密的汇编文件进行修改,
从而实现程序的进一步优化。
\end{enumerate}




\section{项目实现过程}
\subsection{任务1:仿真器拓展}
根据文档说明,直接修改task1文件夹下的代码即可完成任务1。具体流程如下:
\begin{itemize}
    \item instforms指令编码部分:仿真器在 instforms.hpp 文件中针对不同指令类型构建了用于编码和译码
的结构体,需要在 instforms.cpp 中实现对应的编码函数
    \item 如下图git diff所示,参考cube指令的实现,在 instforms.cpp 中实现对应的编码函数,只需要修改func3字段即可
  \begin{figure}[H]
      \centering
      \includegraphics[width =1.0\textwidth]{figures/fig/image2.png}
      \caption{增加指令编码部分}
  \end{figure}

    \item decode指令译码部分:本部分实现了仿真器的译码功能.
    \item 参照 cube 及其他指令格式,在 InstEntry.cpp 和 InstId.hpp 中添加其他四条拓展指令内容,
并及时修改 InstId.hpp 中 maxId 的值。
    \item 如下图git diff所示,InstId.hpp的枚举中添加了拓展指令
  \begin{figure}[H]
      \centering
      \includegraphics[width =1.0\textwidth]{figures/fig/image3.png}
      \caption{添加拓展指令声明}
  \end{figure}
      \item 如下图git diff所示,在InstEntry.cpp中添加了拓展指令的条目。只需要修改InstId枚举字段和指令编码部分即可。
  \begin{figure}[H]
      \centering
      \includegraphics[width =1.0\textwidth]{figures/fig/image4.png}
      \caption{添加拓展指令条目}
  \end{figure}


    \item Hart指令执行:Hart 是 Hardware Thread 的缩写,本pro主要关注从 Hart.cpp 文件 979 
行开始的内容,即指令的执行过程
    \item 参照 cube指令的实现完成其他四条拓展指令的执行过程。
    \item 如下图git diff所示,指令的实现直接参照图1中的指令格式用C++实现即可。
    \item 图中notand指令的实现有误,在git后一个commit中已经修正。
  \begin{figure}[H]
      \centering
      \includegraphics[width =1.0\textwidth]{figures/fig/image5.png}
      \caption{指令执行部分的具体实现}
  \end{figure}

  \begin{figure}[H]
      \centering
      \includegraphics[width =1.0\textwidth]{figures/fig/image6.png}
      \caption{notand指令修正}
  \end{figure}


    \item 最后,需要参照 cube 指令在 Hart.cpp 文件 5442 行开始
的 execute函数中添加拓展指令的 label 以及相应的跳转执行
  \begin{figure}[H]
      \centering
      \includegraphics[width =1.0\textwidth]{figures/fig/image7.png}
      \caption{execute函数中添加拓展指令的label和跳转}
  \end{figure}

\end{itemize}

\textbf{测试结果:}

make\_build 编译模拟器,之后测试新旧指令集。测试结果如下:

可以看到,新旧指令集测试均通过,仿真器拓展基本没有问题。

  \begin{figure}[H]
      \centering
      \includegraphics[width =1.0\textwidth]{figures/fig/image8.png}
      \caption{测试结果}
  \end{figure}

\subsection{任务2:优化 SM3 加密算法}
\textcolor{red}{\textbf{该部分我进行了多个版本的实现,
v1:仅使用拓展指令优化数据操作部分的汇编。
v2:在v1基础上重写了几个调用最多的函数,在数据存储层面上优化了汇编。
v3:使用自定义的大小端转换拓展指令,进一步优化。
v4:在c源码嵌入汇编,函数内联,使用编译器等进行极致的优化。}}

\textbf{准备工作:}

首先,根据sm3.s的内容,不难推断sm3.s是sm3.c的汇编版本。
手动用gcc编译sm3.c,得到sm3-O0.s,然后对比sm3.s,发现是一样的。如下图:

  \begin{figure}[H]
      \centering
      \includegraphics[width =1.0\textwidth]{figures/fig/image9.png}
      \caption{sm3-O0.s与sm3.s对比}
  \end{figure}

  打开gcc的汇编注释选项-fverbose-asm,得到带注释的汇编文件sm3opt.s方便阅读。
  在makefile中写好几个测试用例,方便测试。
  接着初始化git仓库,开始任务2的实现。

  \begin{figure}[H]
      \centering
      \includegraphics[width =1.0\textwidth]{figures/fig/image10.png}
      \caption{准备工作}
  \end{figure}

\newpage

\subsubsection{v1版本}
\textbf{实现过程:}

首先修改long\_to\_str函数。本函数本质是实现大小端转换,将原来的lbu指令单独取byte再存储的复杂方式,改为用reverse依次取出4个byte,用临时寄存器进行移位和与操作,最后再存储到内存中。如下图:

  \begin{figure}[H]
      \centering
      \includegraphics[width =1.0\textwidth]{figures/fig/image11.png}
      \caption{long\_to\_str函数优化}
  \end{figure}

  然后修改str\_to\_long函数。该函数也是实现大小端转换,同理进行修改即可,如下图:
  \begin{figure}[H]
      \centering
      \includegraphics[width =1.0\textwidth]{figures/fig/image12.png}
      \caption{str\_to\_long函数优化}
  \end{figure}

    然后修改group\_a函数中,$x[5] = P1(x[0],x[1],x[2],x[3],x[4])$这一步,即下图703line开始的部分。

    P1宏展开如下图所示,将原来循环移位的方式,改为用rotright和rotleft指令进行移位即可
    \begin{figure}[H]
      \centering
      \includegraphics[width =1.0\textwidth]{figures/fig/image14.png}
      \caption{P1宏展开}
  \end{figure}

  \begin{figure}[H]
      \centering
      \includegraphics[width =1.0\textwidth]{figures/fig/image13.png}
      \caption{P1宏优化}
  \end{figure}



    修改$C[2] = TT1(FF1(A[0],A[1],A[2]),A[3],C[0],A[0],A[4],T1,j);$
对应的汇编代码,即下图1400line开始的部分。同理,将原来的循环移位的方式,改为用rotright和rotleft指令进行移位即可


    \begin{figure}[H]
      \centering
      \includegraphics[width =1.0\textwidth]{figures/fig/image16.png}
      \caption{C[2] = TT1(FF1(A[0],A[1],A[2]),A[3],C[0],A[0],A[4],T1,j) 宏展开}
  \end{figure}

  \begin{figure}[H]
      \centering
      \includegraphics[width =1.0\textwidth]{figures/fig/image15.png}
      \caption{C[2] = TT1(FF1(A[0],A[1],A[2]),A[3],C[0],A[0],A[4],T1,j) 宏优化}
  \end{figure}

    下一条$C[3] = TT2(GG1(A[4],A[5],A[6]),A[7],C[1],A[0],A[4],T1,j)$同理,不多赘述。

  \begin{figure}[H]
      \centering
      \includegraphics[width =1.0\textwidth]{figures/fig/image17.png}
      \caption{C[3] = TT2(GG1(A[4],A[5],A[6]),A[7],C[1],A[0],A[4],T1,j) 宏优化}
  \end{figure}

    之后有两个ROTL宏,同样用rotright和rotleft指令进行优化即可。
  \begin{figure}[H]
      \centering
      \includegraphics[width =1.0\textwidth]{figures/fig/image18.png}
      \caption{ROTL宏优化}
  \end{figure}

  目前已经优化了下图红框内的部分,接下来优化下图绿框内的部分。
  完全和上述步骤一样,用rotright和rotleft指令进行优化即可,
  其中有一个notand指令,用notand指令进行优化即可。
  具体内容见sm3opt\_v1.s中源码,不多赘述
  \begin{figure}[H]
      \centering
      \includegraphics[width =1.0\textwidth]{figures/fig/image19.png}
      \caption{优化部分对应的源c代码}
  \end{figure}
  \begin{figure}[H]
      \centering
      \includegraphics[width =1.0\textwidth]{figures/fig/image20.png}
      \caption{notand优化处}
  \end{figure}
\newpage
\textbf{v1测试结果:}
  如图,使用makefile中的测试用例测试三个字符串的哈希值进行比较,结果均正确。
相比原始版本,优化后的版本减少了3k条左右的指令,性能有所提升。

  \begin{figure}[H]
      \centering
      \includegraphics[width =1.0\textwidth]{figures/fig/image21.png}
      \caption{notand优化处}
  \end{figure}

  \newpage
\subsubsection{v2版本:重写部分函数}
\textbf{实现过程:}
经过测试发现,v1版本的优化效果并不是很明显,因此我决定重写一些调用最多的函数,优化数据存储层面的汇编代码。

经常测试会发现long\_to\_str和str\_to\_long函数调用次数最多,v2版本是对这两个函数进行重写。

如下图,直接把所有栈操作去掉,直接用参数寄存器和临时寄存器进行操作,减少了大量的指令条目。
  \begin{figure}[H]
      \centering
      \includegraphics[width =1.0\textwidth]{figures/fig/image22.png}
      \caption{long\_to\_str函数重写}
  \end{figure}

另一个函数str\_to\_long同理,如下图:

  \begin{figure}[H]
      \centering
      \includegraphics[width =1.0\textwidth]{figures/fig/image23.png}
      \caption{str\_to\_long函数重写}
  \end{figure}

\textbf{v2测试结果:}
  如图,现在变成6w条左右的指令,性能有所提升。
  \begin{figure}[H]
      \centering
      \includegraphics[width =1.0\textwidth]{figures/fig/image24.png}
      \caption{v2测试结果}
  \end{figure}

\newpage

\subsubsection{v3版本:增加自定义大小端转换指令}

\textbf{实现过程:}
因为这两个函数调用次数最多,且两个函数都是实现大小端转换,因此自定义一个大小端转换的指令,用于优化这两个函数。

具体操作在任务一中已经详细说明,这里不再赘述,仅介绍hart.cpp中的实现。

  \begin{figure}[H]
      \centering
      \includegraphics[width =1.0\textwidth]{figures/fig/image25.png}
      \caption{增加自定义大小端转换指令}
  \end{figure}

然后把汇编中的有关的函数调用全部改为一条指令,如下图:
具体修改查看源码中的git commit记录。

  \begin{figure}[H]
      \centering
      \includegraphics[width =1.0\textwidth]{figures/fig/image26.png}
      \caption{函数调用改为一条指令}
  \end{figure}

\textbf{v3测试结果:}
  如图,一下子减少到了5w条左右的指令。说明这两个函数是主要的瓶颈。
  \begin{figure}[H]
      \centering
      \includegraphics[width =1.0\textwidth]{figures/fig/image27.png}
      \caption{v3测试结果}
  \end{figure}

\subsubsection{v4版本:爆改源码,极致优化}
  源汇编代码没有任何编译器优化,导致指令条目过多,其它部分的性能也有待提升。
于是直接从源码上手,借助编译器进行优化。

操作一:将宏改成函数调用,函数内联,函数直接用拓展指令汇编实现,如下图:

  \begin{figure}[H]
      \centering
      \includegraphics[width =1.0\textwidth]{figures/fig/image28.png}
      \caption{宏改成函数调用}
  \end{figure}

  操作二:将两个大小端转换函数改成内联函数,并用自定义的大小端转换指令进行优化,函数也直接用拓展指令汇编实现,如下图:

    \begin{figure}[H]
        \centering
        \includegraphics[width =1.0\textwidth]{figures/fig/image29.png}
        \caption{大小端转换函数改成内联函数}
    \end{figure}

  操作三:将其余部分调用到long\_to\_str和str\_to\_long函数的函数直接改成汇编实现,如下图:

    \begin{figure}[H]
        \centering
        \includegraphics[width =1.0\textwidth]{figures/fig/image30.png}
        \caption{函数直接改成汇编实现}
    \end{figure}

    详细修改见源码中的git commit记录。
    
    最后使用v4/makefile中的脚本进行编译,得到sm3opt\_v4.s,这里直接开启了-O3优化,得到了最后的优化版本。
    \begin{figure}[H]
        \centering
        \includegraphics[width =1.0\textwidth]{figures/fig/image31.png}
        \caption{makefile脚本}
    \end{figure}

\textbf{v4测试结果:}
    如图,对比原始版本,哈希值计算结果均正确,
    指令条目爆减到2.9w条左右。
    \begin{figure}[H]
        \centering
        \includegraphics[width =1.0\textwidth]{figures/fig/image32.png}
        \caption{v4测试结果}
    \end{figure}

\section{项目总结}

\textcolor{red}{最终v4版本的优化效果最好,指令条目从原始版本的7.3w左右减少到了2.9w条左右,性能有了极大的提升。}


\textcolor{red}{
可以再进一步优化,比如设计使用更多的拓展指令来替换c源码中的一些操作,
或者直接用汇编实现源码中的一些函数和过程,进一步提升性能。预计还能减少几千条指令。
不过懒得再写,就这样了。
}


%%----------- 参考文献 -------------------%%
%在reference.bib文件中填写参考文献,此处自动生成




\end{document}