% !TEX program = xelatex
% Homework template

\documentclass[cn,12pt]{homework}
% en is for English language
% cn is for Chinese language

%----- text fonts -----
% \usepackage{newtxtext}
% \setmainfont{Times New Roman}

%----- math font -----
\usepackage{newtxmath}
\usepackage{mathptmx}
\usepackage{mathpazo}
\usepackage{listings}
\usepackage{enumitem}
\usepackage{float}
\usepackage{xcolor}
\usepackage{fancyhdr}
\pagestyle{fancy}
\lstset{
    %backgroundcolor=\color{red!50!green!50!blue!50},%代码块背景色为浅灰色
    rulesepcolor= \color{gray}, %代码块边框颜色
    breaklines=true,  %代码过长则换行
    numbers=left, %行号在左侧显示
    numberstyle= \small,%行号字体
    keywordstyle= \color{blue},%关键字颜色
    commentstyle=\color{gray}, %注释颜色
    frame=shadowbox%用方框框住代码块
    }

%----- custom theorem -----
\newtheorem{innercustomgeneric}{\customgenericname}
\providecommand{\customgenericname}{}
\newcommand{\newcustomtheorem}[2]{%
  \newenvironment{#1}[1]
  {%
   \renewcommand\customgenericname{#2}%
   \renewcommand\theinnercustomgeneric{##1}%
   \innercustomgeneric
  }
  {\endinnercustomgeneric}
}

\newcustomtheorem{ntheorem}{定理}
\newcustomtheorem{nlemma}{引理}

%----- list style -----
\setlist{nolistsep}

% differential operator
\newcommand{\dif}{\mathop{}\!\mathrm{d}}

% new command
\newcommand{\CC}{\ensuremath{\mathbb{C}}}
\newcommand{\RR}{\ensuremath{\mathbb{R}}}
\newcommand{\A}{\mathcal{A}}
\newcommand{\bA}{\boldsymbol{A}}
\newcommand{\ii}{\mathrm{i}\,}
\newcommand{\dx}[1][x]{\mathop{}\!\mathrm{d}#1}
\newcommand{\abs}[1]{\lvert#1\rvert}
\newcommand{\norm}[1]{\left\lVert#1\right\rVert}
\newcommand{\red}[1]{\textcolor{red}{#1}}

%----------------------------------------------------
%	HOMEWORK INFORMATION
%----------------------------------------------------

\lhead{\itshape Computer Organization and Design   } % 页眉
\rhead{3220103462}
\title{HOMEWORK \#3} % 作业名字

\date{Date: \today} % 日期

\institute{ZHEJIANG UNIVERSITY\quad COLLEGE OF INFORMATION SCIENCE AND ELECTRONICS ENGINEERING} % 学院或学校

\courseinfo{Computer Organization and Design } % 课程信息

\studentinfo{Name: \textit{黄贤敏}  \quad  \quad Student ID: \textit{3220103462}  \quad \\ Major: \textit{Electronic Science and Technology}} % 学生信息


\begin{document}

\maketitle

%----------------------------------------------------
%	作业内容
%----------------------------------------------------

%\section*{作业题目}

%------------------------------%
%------------------------------%
\begin{problem}
\begin{figure}[h!]
  \centering
  \includegraphics[width=1\textwidth]{./figures/image1.png}
  \label{fig:pro1}
\end{figure}

\end{problem}




\begin{solution}
  \quad
  1.According to the MIPS equation, the MIPS for each processor  are as follows:

  \begin{figure}[h!]
    \centering
    \includegraphics[width=0.8\textwidth]{./figures/mips.png}
    \label{fig:pro1}
  \end{figure}

    \begin{equation}
      \text{MIPS}_{P1} = \frac{2*10^9}{1.5*10^6} = \frac{4}{3} \times 10^3
  \end{equation}
    \begin{equation}
      \text{MIPS}_{P2} = \frac{1.5*10^9}{1.2*10^6} = 1.25 \times 10^3
  \end{equation}
    \begin{equation}
      \text{MIPS}_{P3} = \frac{3*10^9}{2.0*10^6} = 1.5 \times 10^3
  \end{equation}

  Therefore, the P3 processor has the best performance.

  2.\begin{equation}
      Clock rate = MIPS \times CPI \times 10^6
  \end{equation}

  \begin{equation}
    \text{Clock rate}_{P1} = 130\% \times \frac{4}{3} \times 10^3 \times 115\% \times 1.5 \times 10^6 = 2.99 GHz
  \end{equation}
  \begin{equation}
    \text{Clock rate}_{P2} = 130\% \times 1.25 \times 10^3 \times 115\% \times 1.2 \times 10^6 = 2.2425 GHz
  \end{equation}
  \begin{equation}
    \text{Clock rate}_{P3} = 130\% \times 1.5 \times 10^3 \times 115\% \times 2.0 \times 10^6 = 4.485 GHz
  \end{equation}
\end{solution}
%------------------------------%
\begin{problem}
  \quad
  \begin{figure}[H]
    \centering
    \includegraphics[width=1\textwidth]{./figures/image2.png}
    \label{fig:pro1}
  \end{figure}


  \begin{figure}[H]
    \centering
    \includegraphics[width=1\textwidth]{./figures/image3.png}
    \label{fig:pro1}
  \end{figure}
\end{problem}



\begin{solution}
  \quad
\subsection*{1.1}
\begin{itemize}
    \item For \textbf{1 Processor}:
    \[
    \text{Total Instructions per Processor} = 2560 + 1280 + 256 = 4096
    \]
    \[
    \text{Aggregate Instructions} = 4096 \times 1 = 4096
    \]

    \item For \textbf{2 Processors}:
    \[
    \text{Total Instructions per Processor} = 1280 + 640 + 128 = 2048
    \]
    \[
    \text{Aggregate Instructions} = 2048 \times 2 = 4096
    \]

    \item For \textbf{4 Processors}:
    \[
    \text{Total Instructions per Processor} = 640 + 320 + 64 = 1024
    \]
    \[
    \text{Aggregate Instructions} = 1024 \times 4 = 4096
    \]

    \item For \textbf{8 Processors}:
    \[
    \text{Total Instructions per Processor} = 320 + 160 + 32 = 512
    \]
    \[
    \text{Aggregate Instructions} = 512 \times 8 = 4096
    \]
\end{itemize}

\subsection*{1.2}
\[
\text{Execution Time (per processor)} = \sum \left( \frac{\text{Instructions of Type}}{\text{Clock Frequency}} \times \text{CPI of Type} \right)
\]
Assume clock frequency \(f = 2 \, \text{GHz}\):
\[
\text{Execution Time} = \frac{\text{Instructions (Arithmetic)} \times \text{CPI (Arithmetic)}}{f} + \frac{\text{Instructions (Load/Store)} \times \text{CPI (Load/Store)}}{f} + \frac{\text{Instructions (Branch)} \times \text{CPI (Branch)}}{f}
\]

\begin{itemize}
    \item For \textbf{1 Processor}:
    \[
    \text{Execution Time} = \frac{2560 \times 1}{2} + \frac{1280 \times 4}{2} + \frac{256 \times 2}{2} = 1280 + 2560 + 256 = 4096 \, \text{ns}
    \]

    \item For \textbf{2 Processors}:
    \[
    \text{Execution Time} = \frac{1280 \times 1}{2} + \frac{640 \times 5}{2} + \frac{128 \times 2}{2} = 640 + 1600 + 128 = 2368 \, \text{ns}
    \]

    \item For \textbf{4 Processors}:
    \[
    \text{Execution Time} = \frac{640 \times 1}{2} + \frac{320 \times 7}{2} + \frac{64 \times 2}{2} = 320 + 1120 + 64 = 1504 \, \text{ns}
    \]

    \item For \textbf{8 Processors}:
    \[
    \text{Execution Time} = \frac{320 \times 1}{2} + \frac{160 \times 12}{2} + \frac{32 \times 2}{2} = 160 + 960 + 32 = 1152 \, \text{ns}
    \]
\end{itemize}

\subsection*{1.3}
\begin{itemize}
    \item For \textbf{1 Processor}:
    \[
    \text{Execution Time} = \frac{2560 \times 2}{2} + \frac{1280 \times 4}{2} + \frac{256 \times 2}{2} = 2560 + 2560 + 256 = 5376 \, \text{ns}
    \]

    \item For \textbf{2 Processors}:
    \[
    \text{Execution Time} = \frac{1280 \times 2}{2} + \frac{640 \times 5}{2} + \frac{128 \times 2}{2} = 1280 + 1600 + 128 = 3008 \, \text{ns}
    \]

    \item For \textbf{4 Processors}:
    \[
    \text{Execution Time} = \frac{640 \times 2}{2} + \frac{320 \times 7}{2} + \frac{64 \times 2}{2} = 640 + 1120 + 64 = 1824 \, \text{ns}
    \]

    \item For \textbf{8 Processors}:
    \[
    \text{Execution Time} = \frac{320 \times 2}{2} + \frac{160 \times 12}{2} + \frac{32 \times 2}{2} = 320 + 960 + 32 = 1312 \, \text{ns}
    \]
\end{itemize}

\subsection*{1.4}

\begin{itemize}
    \item For \textbf{1 Core}:
    \[
    \text{Execution Time} = \frac{1.00 \times 10^{10} \times 1.2}{3 \times 10^9} = \frac{1.2 \times 10^{10}}{3 \times 10^9} = 4 \, \text{seconds}
    \]

    \item For \textbf{2 Cores}:
    \[
    \text{Execution Time} = \frac{5.00 \times 10^9 \times 1.4}{3 \times 10^9} = \frac{7.00 \times 10^9}{3 \times 10^9} = \frac{7}{3} \, \text{seconds}
    \]

    \item For \textbf{4 Cores}:
    \[
    \text{Execution Time} = \frac{2.50 \times 10^9 \times 1.8}{3 \times 10^9} = \frac{4.50 \times 10^9}{3 \times 10^9} = 1.5 \, \text{seconds}
    \]

    \item For \textbf{8 Cores}:
    \[
    \text{Execution Time} = \frac{1.25 \times 10^9 \times 2.6}{3 \times 10^9} = \frac{3.25 \times 10^9}{3 \times 10^9} = \frac{13}{12} \, \text{seconds}
    \]
\end{itemize}

\subsection*{1.5}

\begin{itemize}
    \item For \textbf{2 Cores}:
    \[
    \text{Aggregate Instructions} = 5.00 \times 10^9 \times 2 = 1.00 \times 10^{10}
    \]
    \[
    \text{CPI}_{\text{required}} = \frac{\frac{7}{3} \, \text{seconds} \times 3 \times 10^9}{1.00 \times 10^{10}} = \frac{7 \times 10^9}{1.00 \times 10^{10}} = 0.7
    \]

    \item For \textbf{4 Cores}:
    \[
    \text{Aggregate Instructions} = 2.50 \times 10^9 \times 4 = 1.00 \times 10^{10}
    \]
    \[
    \text{CPI}_{\text{required}} = \frac{1.5 \, \text{seconds} \times 3 \times 10^9}{1.00 \times 10^{10}} = \frac{4.5 \times 10^9}{1.00 \times 10^{10}} = 0.45
    \]

    \item For \textbf{8 Cores}:
    \[
    \text{Aggregate Instructions} = 1.25 \times 10^9 \times 8 = 1.00 \times 10^{10}
    \]
    \[
    \text{CPI}_{\text{required}} = \frac{\frac{13}{12} \, \text{seconds} \times 3 \times 10^9}{1.00 \times 10^{10}} = \frac{3.25 \times 10^9}{1.00 \times 10^{10}} = 0.325
    \]
\end{itemize}

\end{solution}
\newpage




%------------------------------%
\end{document}


