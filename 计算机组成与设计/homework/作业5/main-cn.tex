% !TEX program = xelatex
% Homework template

\documentclass[cn,12pt]{homework}
% en is for English language
% cn is for Chinese language

%----- text fonts -----
% \usepackage{newtxtext}
% \setmainfont{Times New Roman}

%----- math font -----
\usepackage{newtxmath}
\usepackage{mathptmx}
\usepackage{mathpazo}
\usepackage{listings}
\usepackage{enumitem}
\usepackage{float}
\usepackage{xcolor}
\usepackage{fancyhdr}
\pagestyle{fancy}
\lstset{
    %backgroundcolor=\color{red!50!green!50!blue!50},%代码块背景色为浅灰色
    rulesepcolor= \color{gray}, %代码块边框颜色
    breaklines=true,  %代码过长则换行
    numbers=left, %行号在左侧显示
    numberstyle= \small,%行号字体
    keywordstyle= \color{blue},%关键字颜色
    commentstyle=\color{gray}, %注释颜色
    frame=shadowbox%用方框框住代码块
    }

%----- custom theorem -----
\newtheorem{innercustomgeneric}{\customgenericname}
\providecommand{\customgenericname}{}
\newcommand{\newcustomtheorem}[2]{%
  \newenvironment{#1}[1]
  {%
   \renewcommand\customgenericname{#2}%
   \renewcommand\theinnercustomgeneric{##1}%
   \innercustomgeneric
  }
  {\endinnercustomgeneric}
}

\newcustomtheorem{ntheorem}{定理}
\newcustomtheorem{nlemma}{引理}

%----- list style -----
\setlist{nolistsep}

% differential operator
\newcommand{\dif}{\mathop{}\!\mathrm{d}}

% new command
\newcommand{\CC}{\ensuremath{\mathbb{C}}}
\newcommand{\RR}{\ensuremath{\mathbb{R}}}
\newcommand{\A}{\mathcal{A}}
\newcommand{\bA}{\boldsymbol{A}}
\newcommand{\ii}{\mathrm{i}\,}
\newcommand{\dx}[1][x]{\mathop{}\!\mathrm{d}#1}
\newcommand{\abs}[1]{\lvert#1\rvert}
\newcommand{\norm}[1]{\left\lVert#1\right\rVert}
\newcommand{\red}[1]{\textcolor{red}{#1}}

%----------------------------------------------------
%	HOMEWORK INFORMATION
%----------------------------------------------------

\lhead{\itshape Computer Organization and Design   } % 页眉
\rhead{3220103462}
\title{HOMEWORK \#5} % 作业名字

\date{Date: \today} % 日期

\institute{ZHEJIANG UNIVERSITY\quad COLLEGE OF INFORMATION SCIENCE AND ELECTRONICS ENGINEERING} % 学院或学校

\courseinfo{Computer Organization and Design } % 课程信息

\studentinfo{Name: \textit{黄贤敏}  \quad  \quad Student ID: \textit{3220103462}  \quad \\ Major: \textit{Electronic Science and Technology}} % 学生信息


\begin{document}

\maketitle

%----------------------------------------------------
%	作业内容
%----------------------------------------------------

%\section*{作业题目}

%------------------------------%
%------------------------------%
\begin{problem}
\begin{figure}[h!]
  \centering
  \includegraphics[width=1\textwidth]{./figures/image1.png}
  \label{fig:pro1}
\end{figure}

\end{problem}




\begin{solution}



\subsection*{a) Clock cycle time in a pipelined and non-pipelined processor}

\textbf{Non-pipelined processor:}  
The clock cycle time of a non-pipelined processor is the sum of the latencies of all stages:  
\[
T_{\text{non-pipelined}} = \text{IF} + \text{ID} + \text{EX} + \text{MEM} + \text{WB}
\]  
Substitute the values:  
\[
T_{\text{non-pipelined}} = 250 \, \text{ps} + 400 \, \text{ps} + 150 \, \text{ps} + 350 \, \text{ps} + 200 \, \text{ps} = 1350 \, \text{ps}
\]

\textbf{Pipelined processor:}  
The clock cycle time of a pipelined processor is determined by the stage with the maximum latency:  
\[
T_{\text{pipelined}} = \max(\text{IF}, \text{ID}, \text{EX}, \text{MEM}, \text{WB})
\]  
Substitute the values:  
\[
T_{\text{pipelined}} = \max(250 \, \text{ps}, 400 \, \text{ps}, 150 \, \text{ps}, 350 \, \text{ps}, 200 \, \text{ps}) = 400 \, \text{ps}
\]

\subsection*{b) Total latency of an LW instruction}

\textbf{Non-pipelined processor:}  
no matter pipelined or non-pipelined, the total latency of an LW 
instruction is the sum of the latencies of all the stages that the instruction passes through:

\textbf{because the LW instruction passes through all stages, the total latency is the same as the clock cycle time:}\[
\text{Latency}_{\text{non-pipelined, LW}} = T_{\text{non-pipelined}} = 1350 \, \text{ps}
\]

\textbf{Pipelined processor:}  
\[
\text{Latency}_{\text{pipelined, LW}} = T_{\text{non-pipelined}} = 1350 \, \text{ps}
\]  

\subsection*{c) Splitting one stage to reduce clock cycle time}

\textbf{Selecting the stage to split:}  
we choose the stage with the maximum latency, which is the \(\text{ID}\) stage (400 ps).

If we split the \(\text{ID}\) stage (400 ps) into two stages with half the latency each (200 ps), the new maximum stage latency becomes:  

\[
T_{\text{new-pipelined}} = \max(\text{IF}, \text{ID/2}, \text{EX}, \text{MEM}, \text{WB})=400 \, \text{ps}
\]  

\textbf{New clock cycle time:}  
\[
T_{\text{new-pipelined}} = 350 \, \text{ps}
\]







\end{solution}
\newpage



%------------------------------%
\begin{problem}
  \quad
  \begin{figure}[H]
    \centering
    \includegraphics[width=1\textwidth]{./figures/image2.png}
    \label{fig:pro1}
  \end{figure}



\end{problem}



\begin{solution}
\subsection*{Identification of Hazards}
1. \textbf{Between Instructions 1 and 2:}  
   - Instruction 2 requires the value of \texttt{t0}, which is produced in the \textbf{EX} stage of Instruction 1 .
   But  in the \textbf{ID} stage of Instruction 2, the value of \texttt{t0} is not yet write back to the register file.


2. \textbf{Between Instructions 1 and 3:}  
   - Instruction 3 also requires the value of \texttt{t0}, which is produced in the \textbf{EX} stage of Instruction 1 .
   but  in the \textbf{ID} stage of Instruction 3, the value of \texttt{t0} is not yet write back to the register file.

\subsection*{Resolving Hazards Using Forwarding}

Forwarding allows the processor to bypass the pipeline register and directly use the result from an earlier stage. Here's how forwarding resolves these hazards:

1. \textbf{Instruction 1 → Instruction 2:}  
   - Forward the result of \texttt{t0} from the end of \textbf{EX} stage of Instruction 1 to the \textbf{EX} stage of Instruction 2.  

  \begin{figure}[H]
    \centering
    \includegraphics[width=0.71\textwidth]{./figures/forward1.png}
    \label{fig:pro1}
  \end{figure}


2. \textbf{Instruction 1 → Instruction 3:}  
   - Forward the result of \texttt{t0} from the end of \textbf{MEM} stage of Instruction 1 to the \textbf{ID} stage of Instruction 3.  


  \begin{figure}[H]
    \centering
    \includegraphics[width=0.71\textwidth]{./figures/forward2.png}
    \label{fig:pro1}
  \end{figure}

\end{solution}
\newpage


%------------------------------%

\begin{problem}
  \quad
  \begin{figure}[H]
    \centering
    \includegraphics[width=1\textwidth]{./figures/image3.png}
    \label{fig:pro1}
  \end{figure}

\end{problem}



\begin{solution}
\textcolor{red}{Assume the register files are read after write in the same clock cycle, so we don't need to consider the 3rd data hazard between
 Instruction 1 and Instruction 4.
 Oherwise, we need to insert one more NOP to prevent data hazards.}


  \begin{figure}[H]
    \centering
    \includegraphics[width=0.71\textwidth]{./figures/forward3.png}
    \label{fig:pro1}
    \caption{the 3rd data hazard between Instruction 1 and Instruction 4}
  \end{figure}

\subsection*{Code Sequence}
\begin{verbatim}
1. add x5, x2, x1
2. lw x3, 4(x5)
3. or x3, x5, x3
4. lw x2, 0(x2)
5. sw x3, 0(x5)
\end{verbatim}

\subsection*{(1) No Forwarding or Hazard Detection}

\textbf{Hazard Analysis:}
\begin{itemize}
    \item \textbf{Instruction 1 → Instruction 2}: data hazard on \texttt{x5}.
    \item \textbf{Instruction 2 → Instruction 3}: data hazard on \texttt{x3}.
    \item \textbf{Instruction 4 → Instruction 5}: data hazard on \texttt{x3}.
\end{itemize}

\textbf{Resulting Code with NOPs:}
\begin{verbatim}
1. add x5, x2, x1
2. NOP
3. NOP
4. lw x3, 4(x5)
5. NOP
6. NOP
7. or x3, x5, x3
8. lw x2, 0(x2)
9. NOP
10. sw x3, 0(x5)
\end{verbatim}

\subsection*{(2) Reordering Instructions with Temporary Register}


the instruction 1,2,3,5 can't be reordered because of the data dependencies.
we can only reorder the instruction 4.

\textcolor{red}{But no matter how we reorder the instruction 4, we can't eliminate the NOPs.
so we can't eliminate the NOPs by reordering instructions.
}

but we can avoid the data hazards between Instruction 2 and Instruction 3
by reordering the instruction 3 and 4.
\begin{verbatim}
1. add x5, x2, x1
2. lw x3, 4(x5)
3. lw x2, 0(x2)
4. or x3, x5, x3
5. sw x3, 0(x5)
\end{verbatim}

\subsection*{(3) Forwarding Without Hazard Detection}

When the instruction following a load uses the result of the load,
the hazard is happening.

the original code has a hazard between Instruction lw x3, 4(x5) and or x3, x5, x3.

so the code execution will be wrong.

\end{solution}
\begin{problem}
  \quad
  \begin{figure}[H]
    \centering
    \includegraphics[width=1\textwidth]{./figures/image4.png}
    \label{fig:pro1}
  \end{figure}

\end{problem}

\begin{solution}
1. \textbf{Pipeline Flush Cost:}  
   - A misprediction causes the pipeline to be flushed, with the dependent instruction replayed.  
   - The flush penalty is \textbf{3 cycles}.

2. \textbf{Predictor Accuracy Threshold:}  
   - Let \( p \) be the probability of a correct prediction (predictor accuracy).  
   - On a correct prediction, a 1-cycle stall is avoided.  
   - On a misprediction, a 3-cycle flush penalty occurs.

\subsection*{Performance Analysis}
To calculate the required accuracy \( p \), we compare the average cycle costs with and without speculation.

\subsubsection*{Cost Without Speculation}
\[
\text{Cost}_{\text{no speculation}} = 1 \quad \text{(1-cycle stall)}
\]

\subsubsection*{Cost With Speculation}
\[
\text{Cost}_{\text{speculation}} = p \cdot 0 + (1 - p) \cdot 3 = 3(1 - p)
\]

The predictor improves performance if:
\[
\text{Cost}_{\text{speculation}} < \text{Cost}_{\text{no speculation}}
\]

\subsubsection*{Solving the Inequality}
\[
3(1 - p) < 1
\]
\[
3 - 3p < 1
\]
\[
3p > 2
\]
\[
p > \frac{2}{3}
\]

\subsection*{Conclusion}
The load-zero predictor must have an accuracy of at least:
\[
\boxed{\frac{2}{3} \, (66.67\%)}
\]
to improve pipeline performance.
\end{solution}

\end{document}


