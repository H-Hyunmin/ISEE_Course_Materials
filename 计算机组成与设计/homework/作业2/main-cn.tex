% !TEX program = xelatex
% Homework template

\documentclass[cn,12pt]{homework}
% en is for English language
% cn is for Chinese language

%----- text fonts -----
% \usepackage{newtxtext}
% \setmainfont{Times New Roman}

%----- math font -----
\usepackage{newtxmath}
\usepackage{mathptmx}
\usepackage{mathpazo}
\usepackage{listings}
\usepackage{enumitem}
\usepackage{float}
\usepackage{xcolor}
\usepackage{fancyhdr}
\pagestyle{fancy}
\lstset{
    %backgroundcolor=\color{red!50!green!50!blue!50},%代码块背景色为浅灰色
    rulesepcolor= \color{gray}, %代码块边框颜色
    breaklines=true,  %代码过长则换行
    numbers=left, %行号在左侧显示
    numberstyle= \small,%行号字体
    keywordstyle= \color{blue},%关键字颜色
    commentstyle=\color{gray}, %注释颜色
    frame=shadowbox%用方框框住代码块
    }

%----- custom theorem -----
\newtheorem{innercustomgeneric}{\customgenericname}
\providecommand{\customgenericname}{}
\newcommand{\newcustomtheorem}[2]{%
  \newenvironment{#1}[1]
  {%
   \renewcommand\customgenericname{#2}%
   \renewcommand\theinnercustomgeneric{##1}%
   \innercustomgeneric
  }
  {\endinnercustomgeneric}
}

\newcustomtheorem{ntheorem}{定理}
\newcustomtheorem{nlemma}{引理}

%----- list style -----
\setlist{nolistsep}

% differential operator
\newcommand{\dif}{\mathop{}\!\mathrm{d}}

% new command
\newcommand{\CC}{\ensuremath{\mathbb{C}}}
\newcommand{\RR}{\ensuremath{\mathbb{R}}}
\newcommand{\A}{\mathcal{A}}
\newcommand{\bA}{\boldsymbol{A}}
\newcommand{\ii}{\mathrm{i}\,}
\newcommand{\dx}[1][x]{\mathop{}\!\mathrm{d}#1}
\newcommand{\abs}[1]{\lvert#1\rvert}
\newcommand{\norm}[1]{\left\lVert#1\right\rVert}
\newcommand{\red}[1]{\textcolor{red}{#1}}

%----------------------------------------------------
%	HOMEWORK INFORMATION
%----------------------------------------------------

\lhead{\itshape Computer Organization and Design   } % 页眉
\rhead{3220103462}
\title{HOMEWORK \#2} % 作业名字

\date{Date: \today} % 日期

\institute{ZHEJIANG UNIVERSITY\quad COLLEGE OF INFORMATION SCIENCE AND ELECTRONICS ENGINEERING} % 学院或学校

\courseinfo{Computer Organization and Design } % 课程信息

\studentinfo{Name: \textit{黄贤敏}  \quad  \quad Student ID: \textit{3220103462}  \quad \\ Major: \textit{Electronic Science and Technology}} % 学生信息


\begin{document}

\maketitle

%----------------------------------------------------
%	作业内容
%----------------------------------------------------

%\section*{作业题目}

%------------------------------%
%------------------------------%
\begin{problem}
\begin{figure}[h!]
  \centering
  \includegraphics[width=1\textwidth]{./figures/image1.png}
  \caption{pro1}
  \label{fig:pro1}
\end{figure}

\end{problem}




\begin{solution}
  \quad
  The code calculates the sum of all odd numbers from 1 to n.
  \begin{lstlisting}[language=C, caption=Code for Problem 1, label=code:example, frame=single, numbers=left, numberstyle=\tiny, breaklines=true, backgroundcolor=\color{lightgray}]
  begin: addi t0, x0, 0   // t0 = sum = 0
  addi t1, x0, 1          // t1 = i = 1
  loop: slt t2, a0, t1    // if i < n
  bne t2, x0, finish      // goto finish
  add t0, t0, t1          // sum += i
  addi t1, t1, 2          // i += 2
  j loop                  // goto loop for next iteration
  finish: add a0, t0, x0  // a0 = sum
  \end{lstlisting}

\end{solution}
%------------------------------%
\begin{problem}
  \quad
Write a loop that reverses order of the bits of an 8-bit number in s0 and stores the result in s1.
RISC-V registers have 4 bytes, but in this problem, the higher 24 bits of s0 are always 0. For
example, if the lower 8 bits of s0 are 00001101, the lower 8 bites s1 should be 10110000.
\end{problem}

\begin{solution}
  \quad
  \begin{lstlisting}[language=C, caption=Code for Problem 2, label=code:example, frame=single, numbers=left, numberstyle=\tiny, breaklines=true, backgroundcolor=\color{lightgray}]
  _start:
      li t0, 8          # 设置循环计数器 t0 为 8
      li s1, 0          # 将 s1 初始化为 0
      li t1, 1          # t1 用于提取 s0 中的每一位
      li t2, 128        # t2 用于设置 s1 中的每一位

  reverse_loop:
      beqz t0, end_loop # 如果 t0 为 0,跳转到 end_loop
      and t3, s0, t1    # 提取 s0 中的当前位
      beqz t3, skip_set # 如果当前位为 0,跳过设置 
      or s1, s1, t2     # 设置 s1 中的当前位

  skip_set:
      srli s0, s0, 1    # 右移 s0
      srli t2, t2, 1    # 右移 t2
      addi t0, t0, -1   # 递减 t0
      j reverse_loop    # 跳转到 reverse_loop

  end_loop:
      # 结束程序
      li a7, 10         # ecall 10 系统调用结束程序
      ecall
  \end{lstlisting}

\end{solution}
\newpage

\begin{problem}
  \quad
Assume we have an array in memory that contains int* arr = {1,2,3,4,5,6,0}. Let the values of arr 
be a multiple of 4 and stored in register s0. What do the snippets of RISC-V code do? Assume that 
all the instructions are run one after the other in the same context.
  \begin{lstlisting}[language=C, caption=Code for Problem 2, label=code:example, frame=single, numbers=left, numberstyle=\tiny, breaklines=true, backgroundcolor=\color{lightgray}]
a)  lw t0, 8(s0)
b)  slli t1, t0, 2
    add t2, s0, t1
    lw t3, 0(t2)
    addi t3, t3, 1
    sw t3, 0(t2)
c)  lw t0, 16(s0)
    xori t0, t0, 0xFFF 
    addi t0, t0, 1
  \end{lstlisting}
  
\end{problem}

\begin{solution}
  \quad

\begin{itemize}
  \item \textbf{Snippet a:} \textbf{t0=arr[2]=3;}
  \begin{itemize}
    \item \texttt{lw t0, 8(s0)} \\
    t0=arr[2]=3
  \end{itemize}

  \item \textbf{Snippet b:} \textbf{arr[3]++;}
  \begin{itemize}
    \item \texttt{slli t1, t0, 2} \\
    t1=t0<<2, so \texttt{t1} will be set to \texttt{3 << 2 = 12}.
    \item \texttt{add t2, s0, t1} \\
    t2=s0+t1=s0+12
    \item \texttt{lw t3, 0(t2)} \\
    equivalent to \texttt{lw t3, 12(s0)}, t3=arr[3]=4
    \item \texttt{addi t3, t3, 1} \\
    t3=t3+1=5
    \item \texttt{sw t3, 0(t2)} \\
    equivalent to \texttt{sw t3, 12(s0)}, arr[3]=5
  \end{itemize}

  \item \textbf{Snippet c:}\textbf{t0=-arr[4];} 
  
  \begin{itemize}
    \item \texttt{lw t0, 16(s0)} \\
    t0=arr[4]=5
    \item \texttt{xori t0, t0, 0xFFF} \\
    t0=t0$\oplus$0xFFF=5$\oplus$0xFFF=0xFFA
    \item \texttt{addi t0, t0, 1} \\
    t0=t0+1=0xFFB
  \end{itemize}
\end{itemize}

\end{solution}


\begin{problem}
  \quad
Write a function sumSquare in RISC-V that, when given an integer n, returns the summation 
below. If n is not positive, then the function returns 0.
\begin{equation}
  \Sigma_{i=1}^{n} i^2
\end{equation}
For this problem, you are given a RISC-V function called square that takes in an integer and 
returns its square. Implement sumSquare using square as a subroutine
\end{problem}

\begin{solution}
  \quad

  \begin{lstlisting}[language=C, caption=Code for Problem 2, label=code:example, frame=single, numbers=left, numberstyle=\tiny, breaklines=true, backgroundcolor=\color{lightgray}]
.globl sumSquare
.text
sumSquare:
    // YOUR CODE BEGIN
    // 保存上下文
    addi sp, sp, -16
    sw ra, 12(sp)
    sw s0, 8(sp)
    sw s1, 4(sp)
    sw s2, 0(sp)
    // 检查 n 是否为正数
    blez a0, return_zero
    // 初始化累加器 sum 为 0
    li s0, 0
    // 初始化循环变量
    mv s1, a0
loop:
    // 将当前 n 的值传递给 square 函数
    mv a0, s1
    jal ra, square
    // 将平方值加到累加器 sum 中
    add s0, s0, a0
    // 减少 n 的值
    addi s1, s1, -1
    // 如果 n > 0,继续循环
    bgtz s1, loop
    // 将累加器 sum 的值放入 a0 寄存器中作为返回值
    mv a0, s0
    j end
return_zero:
    // 返回 0
    li a0, 0
end:
    // 恢复上下文
    lw ra, 12(sp)
    lw s0, 8(sp)
    lw s1, 4(sp)
    lw s2, 0(sp)
    addi sp, sp, 16
    // YOUR CODE END
    ret
.end


  \end{lstlisting}

\begin{figure}[h!]
  \centering
  \includegraphics[width=1\textwidth]{./figures/result.png}
  \caption{result}
  \label{fig:pro1}
\end{figure}


\end{solution}



%------------------------------%
\end{document}


