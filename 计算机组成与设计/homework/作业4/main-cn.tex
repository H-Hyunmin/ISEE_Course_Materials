% !TEX program = xelatex
% Homework template

\documentclass[cn,12pt]{homework}
% en is for English language
% cn is for Chinese language

%----- text fonts -----
% \usepackage{newtxtext}
% \setmainfont{Times New Roman}

%----- math font -----
\usepackage{newtxmath}
\usepackage{mathptmx}
\usepackage{mathpazo}
\usepackage{listings}
\usepackage{enumitem}
\usepackage{float}
\usepackage{xcolor}
\usepackage{fancyhdr}
\pagestyle{fancy}
\lstset{
    %backgroundcolor=\color{red!50!green!50!blue!50},%代码块背景色为浅灰色
    rulesepcolor= \color{gray}, %代码块边框颜色
    breaklines=true,  %代码过长则换行
    numbers=left, %行号在左侧显示
    numberstyle= \small,%行号字体
    keywordstyle= \color{blue},%关键字颜色
    commentstyle=\color{gray}, %注释颜色
    frame=shadowbox%用方框框住代码块
    }

%----- custom theorem -----
\newtheorem{innercustomgeneric}{\customgenericname}
\providecommand{\customgenericname}{}
\newcommand{\newcustomtheorem}[2]{%
  \newenvironment{#1}[1]
  {%
   \renewcommand\customgenericname{#2}%
   \renewcommand\theinnercustomgeneric{##1}%
   \innercustomgeneric
  }
  {\endinnercustomgeneric}
}

\newcustomtheorem{ntheorem}{定理}
\newcustomtheorem{nlemma}{引理}

%----- list style -----
\setlist{nolistsep}

% differential operator
\newcommand{\dif}{\mathop{}\!\mathrm{d}}

% new command
\newcommand{\CC}{\ensuremath{\mathbb{C}}}
\newcommand{\RR}{\ensuremath{\mathbb{R}}}
\newcommand{\A}{\mathcal{A}}
\newcommand{\bA}{\boldsymbol{A}}
\newcommand{\ii}{\mathrm{i}\,}
\newcommand{\dx}[1][x]{\mathop{}\!\mathrm{d}#1}
\newcommand{\abs}[1]{\lvert#1\rvert}
\newcommand{\norm}[1]{\left\lVert#1\right\rVert}
\newcommand{\red}[1]{\textcolor{red}{#1}}

%----------------------------------------------------
%	HOMEWORK INFORMATION
%----------------------------------------------------

\lhead{\itshape Computer Organization and Design   } % 页眉
\rhead{3220103462}
\title{HOMEWORK \#4} % 作业名字

\date{Date: \today} % 日期

\institute{ZHEJIANG UNIVERSITY\quad COLLEGE OF INFORMATION SCIENCE AND ELECTRONICS ENGINEERING} % 学院或学校

\courseinfo{Computer Organization and Design } % 课程信息

\studentinfo{Name: \textit{黄贤敏}  \quad  \quad Student ID: \textit{3220103462}  \quad \\ Major: \textit{Electronic Science and Technology}} % 学生信息


\begin{document}

\maketitle

%----------------------------------------------------
%	作业内容
%----------------------------------------------------

%\section*{作业题目}

%------------------------------%
%------------------------------%
\begin{problem}
\begin{figure}[h!]
  \centering
  \includegraphics[width=1\textwidth]{./figures/image1.png}
  \label{fig:pro1}
\end{figure}

\end{problem}




\begin{solution}
  \quad
 \[
151_{10} = 10010111_2
\]
\[
214_{10} = 11010110_2
\]

 \[
  10010111_2= -105_{10}
\]

\[
  11010110_2= -42_{10}
\]

\[
  10010111_2 + 11010110_2 = 1\ 01101101_2= -147_{10}
\]

the result overflowed.Accoring to the saturing arithmetic, the result should be \textbf{\large-128.}

\end{solution}
\newpage
%------------------------------%
\begin{problem}
  \quad
  \begin{figure}[H]
    \centering
    \includegraphics[width=1\textwidth]{./figures/image2.png}
    \label{fig:pro1}
  \end{figure}



\end{problem}



\begin{solution}
  \quad

\[
-1.5625 \times 10^{-1} = -0.15625_{10} = -0.00101_2=-1.01_2 \times 2^{-3}
\]

\[
  s=1\quad Exponent=-3+15=12=01100_2 \quad Mantissa=0100000000_2
\]
Thus,the 16-bit floating-point representation of -0.15625 is \textbf{\large 1 01100 0100000000.}

The 16-bit half precision floating point format has a smaller range and lower accuracy compared to the single precision IEEE 754 standard.

\begin{itemize}
  \item \textbf{Range}: The single precision format uses 8 bits for the exponent, allowing for a larger range of representable values compared to the 5-bit exponent in the half precision format.
  \item \textbf{Accuracy}: The single precision format has a 23-bit mantissa, providing higher precision than the 10-bit mantissa in the half precision format.
\end{itemize}
\end{solution}
\newpage


%------------------------------%

\begin{problem}
  \quad
  \begin{figure}[H]
    \centering
    \includegraphics[width=1\textwidth]{./figures/image3.png}
    \label{fig:pro1}
  \end{figure}

\end{problem}



\begin{solution}
  \quad
\[
2.6125 \times 10^1 = 26.125_{10} = 11010.001_2=1.1010001_2 \times 2^4
\]

\[
4.150390625 \times 10^{-1} = 0.4150390625_{10} = 0.011010100100_2=1.1010100100_2 \times 2^{-2}
\]

\[
1.1010100100_2 \times 2^{-2} \times 2^{-2} = 0.0000011010100100_2 \times 2^4
\]
\[
    1.1010001000_2 \times 2^4
  + 0.0000011010 \quad 100100_2 \times 2^4
  = 1.1010100010 \quad 100100_2 \times 2^4
\]
the result with guard = 1, round = 0, and sticky is set .
Thus the result round up  and the result is:

\[
1.1010100011 \times 2^4  = 26.546875_{10} =2.6546875 \times 10^1
\]
\end{solution}
\newpage


\end{document}


