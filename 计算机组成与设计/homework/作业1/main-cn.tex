% !TEX program = xelatex
% Homework template

\documentclass[cn,12pt]{homework}
% en is for English language
% cn is for Chinese language

%----- text fonts -----
% \usepackage{newtxtext}
% \setmainfont{Times New Roman}

%----- math font -----
\usepackage{newtxmath}
\usepackage{mathptmx}
\usepackage{mathpazo}
\usepackage{listings}
\usepackage{enumitem}
\usepackage{float}
\usepackage{xcolor}
\usepackage{fancyhdr}
\pagestyle{fancy}
\lstset{
    %backgroundcolor=\color{red!50!green!50!blue!50},%代码块背景色为浅灰色
    rulesepcolor= \color{gray}, %代码块边框颜色
    breaklines=true,  %代码过长则换行
    numbers=left, %行号在左侧显示
    numberstyle= \small,%行号字体
    keywordstyle= \color{blue},%关键字颜色
    commentstyle=\color{gray}, %注释颜色
    frame=shadowbox%用方框框住代码块
    }

%----- custom theorem -----
\newtheorem{innercustomgeneric}{\customgenericname}
\providecommand{\customgenericname}{}
\newcommand{\newcustomtheorem}[2]{%
  \newenvironment{#1}[1]
  {%
   \renewcommand\customgenericname{#2}%
   \renewcommand\theinnercustomgeneric{##1}%
   \innercustomgeneric
  }
  {\endinnercustomgeneric}
}

\newcustomtheorem{ntheorem}{定理}
\newcustomtheorem{nlemma}{引理}

%----- list style -----
\setlist{nolistsep}

% differential operator
\newcommand{\dif}{\mathop{}\!\mathrm{d}}

% new command
\newcommand{\CC}{\ensuremath{\mathbb{C}}}
\newcommand{\RR}{\ensuremath{\mathbb{R}}}
\newcommand{\A}{\mathcal{A}}
\newcommand{\bA}{\boldsymbol{A}}
\newcommand{\ii}{\mathrm{i}\,}
\newcommand{\dx}[1][x]{\mathop{}\!\mathrm{d}#1}
\newcommand{\abs}[1]{\lvert#1\rvert}
\newcommand{\norm}[1]{\left\lVert#1\right\rVert}
\newcommand{\red}[1]{\textcolor{red}{#1}}

%----------------------------------------------------
%	HOMEWORK INFORMATION
%----------------------------------------------------

\lhead{\itshape Computer Organization and Design   } % 页眉
\rhead{3220103462}
\title{HOMEWORK \#1} % 作业名字

\date{Date: \today} % 日期

\institute{ZHEJIANG UNIVERSITY\quad COLLEGE OF INFORMATION SCIENCE AND ELECTRONICS ENGINEERING} % 学院或学校

\courseinfo{Computer Organization and Design } % 课程信息

\studentinfo{Name: \textit{黄贤敏}  \quad  \quad Student ID: \textit{3220103462}  \quad \\ Major: \textit{Electronic Science and Technology}} % 学生信息


\begin{document}

\maketitle

%----------------------------------------------------
%	作业内容
%----------------------------------------------------

%\section*{作业题目}

%------------------------------%
%------------------------------%
\begin{problem}


1.14 Another pitfall cited in Section 1.11 is expecting to improve the overall 
performance of a computer by improving only one aspect of the computer. Consider 
a computer running a program that requires 250s, with 70s spent executing FP 
instructions, 85s executing L/S instructions, and 40s spent executing branch instructions.

1.14.1 [5] <§1.11> By how much is the total time reduced if the time for FP 
operations is reduced by 20\%?

1.14.2 [5] <§1.11> By how much is the time for INT operations reduced if the 
total time is reduced by 20\%?

1.14.3 [5] <§1.11> Can the total time be reduced by 20\% by reducing only 
the time for branch instructions?
\end{problem}


\begin{solution}
  \quad

\textbf{1.14.1} The FP operations time is reduced by $70s \times 20\% = 14s$. Therefore, the total time is reduced by 14s, resulting in a new total time of $250s - 14s = 236s$.
the reduction in total time is (250 - 236)/250 = 5.6\%.
 
\textbf{1.14.2} The time for INT is 250-70-85-40=55s. The total time is reduced by 20\%, 
so the new total time is $250s \times 0.8 = 200s$. The time for INT is reduced by $250s - 200s = 50s$. 
The time for INT is reduced by 50s, resulting in a new time for INT of $55s - 50s = 5s$.
The reduction in time for INT is (55 - 5)/55 = 90.9\%.

\textbf{1.14.3} No. The time for branch instructions is only 40s. Even if the time for branch instructions is completely eliminated, the total time can only be reduced by 40s, which is not enough to achieve a 20\% reduction (50s).

\end{solution}
%------------------------------%


%------------------------------%
\end{document}

