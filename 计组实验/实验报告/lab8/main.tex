%!TeX program = xelatex
\documentclass[12pt,hyperref,a4paper,UTF8]{ctexart}
\usepackage{zjureport}
\usepackage{listings}
\usepackage{enumitem}
\usepackage{float}
\usepackage{xcolor}
\lstset{
    %backgroundcolor=\color{red!50!green!50!blue!50},%代码块背景色为浅灰色
    rulesepcolor= \color{gray}, %代码块边框颜色
    breaklines=true,  %代码过长则换行
    numbers=left, %行号在左侧显示
    numberstyle= \small,%行号字体
    keywordstyle= \color{blue},%关键字颜色
    commentstyle=\color{gray}, %注释颜色
    frame=shadowbox%用方框框住代码块
    basicstyle=\small\ttfamily % 设置代码块的基本字体样式
    }


%%-------------------------------正文开始---------------------------%%
\begin{document}

%%-----------------------封面--------------------%%
\cover

%%------------------摘要-------------%%
%\begin{abstract}
%
%在此填写摘要内容
%
%\end{abstract}

\thispagestyle{empty} % 首页不显示页码

%%--------------------------目录页------------------------%%
\newpage
\tableofcontents

%%------------------------正文页从这里开始-------------------%
\newpage

%%可选择这里也放一个标题
%\begin{center}
%    \title{ \Huge \textbf{{标题}}}
%\end{center}

\section{实验目的}
\begin{enumerate}
  \item 掌握快速加法器的设计方法。
  \item 熟悉流水线技术。
  \item 掌握时序仿真的工作流程
\end{enumerate}




\section{实验任务与要求}

\begin{enumerate}
  \item 采用“进位选择加法”技术设计32位加法器,并对设计进行功能仿真和时序仿真。
\item 采用四级流水线技术设计32位加法器,并对设计进行功能仿真和时序仿真。
\end{enumerate}

\section{实验设备}
装有Vivado、ModelSimSE软件的计算机

\section{实验内容}

1.编写4位先行进位加法器的VerilogHDL代码,并用ModelSim软件进行功能仿真。

2.32位进位选择加法器的设计

(1)编写用32位进位选择加法器的VerilogHDL代码,并用ModelSim软件进行功能

(2)对32位加法器进行时序仿真

3.编写采用4级流水线技术的32位加法器的VerilogHDL代码,并对设计进行时序仿真。

\newpage

\section{实验过程与结果}

\subsection{4位先行进位加法器}


\subsubsection{设计原理}
  设计原理如下所示

    \begin{figure}[H]
        \centering
        \includegraphics[width =1.1\textwidth]{figures/fig/image.png}
        \caption{4位先行进位加法器设计原理}
        \label{fig:enter-label}
    \end{figure}

\subsubsection{Verilog HDL代码与设计分析}

\begin{itemize}
    \item \textbf{模块接口}:
    \begin{itemize}
        \item \texttt{a} 和 \texttt{b} 是 4 位的输入信号,表示两个需要相加的 4 位二进制数。
        \item \texttt{ci} 是进位输入,用于处理低位加法器的进位。
        \item \texttt{s} 是 4 位的和输出,表示加法结果。
        \item \texttt{co} 是进位输出,表示最高位的进位。
    \end{itemize}
    \item \textbf{内部信号}:
    \begin{itemize}
        \item \texttt{g} 是生成进位信号,用于表示在某些位上是否会产生进位。
        \item \texttt{p} 是传播进位信号,用于表示在某些位上是否会传播进位。
        \item \texttt{c} 是内部进位信号,用于存储每个位的进位值。
    \end{itemize}
    \item \textbf{生成和传播进位}:
    \begin{itemize}
        \item \texttt{g} 是通过 \texttt{a} 和 \texttt{b} 的按位与操作生成的。如果 \texttt{a} 和 \texttt{b} 的某一位都是 1,则该位会产生进位。
        \item \texttt{p} 是通过 \texttt{a} 和 \texttt{b} 的按位异或操作生成的。如果 \texttt{a} 和 \texttt{b} 的某一位不同,则该位会传播进位。
    \end{itemize}
    \item \textbf{计算进位}:
    \begin{itemize}
        \item \texttt{c[0]} 直接等于输入的进位 \texttt{ci}。
        \item \texttt{c[1]} 到 \texttt{c[3]} 是通过生成进位和传播进位计算的。每一位的进位由前一位的进位和当前位的生成进位或传播进位决定。
        \item \texttt{co} 是最高位的进位输出,由最高位的生成进位和传播进位决定。
    \end{itemize}
    \item \textbf{计算和}:
    \begin{itemize}
        \item \texttt{s} 是通过 \texttt{p} 和 \texttt{c} 的按位异或操作计算的。每一位的和由该位的传播进位和进位值决定。
    \end{itemize}
\end{itemize}


\begin{lstlisting}[language=Verilog]

module adder_4bits (
    input [3:0] a,    // 4位输入a
    input [3:0] b,    // 4位输入b
    input ci,         // 进位输入
    output [3:0] s,   // 4位和输出
    output co         // 进位输出
);

    wire [3:0] g;     // 生成进位
    wire [3:0] p;     // 传播进位
    wire [3:0] c;     // 内部进位

    // 生成和传播进位
    assign g = a & b;
    assign p = a ^ b;

    // 计算进位
    assign c[0] = ci;
    assign c[1] = g[0] | (p[0] & c[0]);
    assign c[2] = g[1] | (p[1] & c[1]);
    assign c[3] = g[2] | (p[2] & c[2]);
    assign co = g[3] | (p[3] & c[3]);

    // 计算和
    assign s = p ^ c;

endmodule


\end{lstlisting}

\subsubsection{ModelSim 仿真}

对4bits先行进位加法器进行功能仿真,仿真结果如下所示。仿真结果表示,设计的4位先行进位加法器功能正常

  \begin{figure}[H]
      \centering
      \includegraphics[width =1.0\textwidth]{figures/fig/Snipaste_2024-09-23_08-50-18.png}
      \caption{4位先行进位加法器功能仿真}
      \label{fig:enter-label}
  \end{figure}

  \newpage

\subsection{32位进位选择加法器}
\subsubsection{设计原理}
  设计原理如下所示

    \begin{figure}[H]
        \centering
        \includegraphics[width =1.0\textwidth]{figures/fig/Snipaste_2024-09-23_08-54-27.png}
        \caption{32位进位选择加法器设计原理}
        \label{fig:enter-label}
    \end{figure}

\subsubsection{Verilog HDL代码与设计分析}

\paragraph{设计分析}
这段 Verilog 代码实现了一个 32 位超前进位加法器。该加法器通过调用多个 4 位超前进位加法器模块和多路复用器来实现。下面是对代码的详细分析:

\begin{itemize}
    \item \textbf{模块接口}:
    \begin{itemize}
        \item \texttt{a} 和 \texttt{b} 是 32 位的输入信号,表示两个需要相加的 32 位二进制数。
        \item \texttt{ci} 是进位输入,用于处理低位加法器的进位。
        \item \texttt{s} 是 32 位的和输出,表示加法结果。
        \item \texttt{co} 是进位输出,表示最高位的进位。
    \end{itemize}
    \item \textbf{内部信号}:
    \begin{itemize}
        \item \texttt{C3, C7, C11, C15, C19, C23, C27, C31} 是级间进位信号,用于存储每个 4 位加法器的进位值。
        \item \texttt{C7\_0, C7\_1, C11\_0, C11\_1, C15\_0, C15\_1, C19\_0, C19\_1, C23\_0, C23\_1, C27\_0, C27\_1, C31\_0, C31\_1} 是内部进位信号,用于存储每个 4 位加法器的进位值。
        \item \texttt{S3\_0, S3\_1, S7\_0, S7\_1, S11\_0, S11\_1, S15\_0, S15\_1, S19\_0, S19\_1, S23\_0, S23\_1, S27\_0, S27\_1, S31\_0, S31\_1} 是内部和信号,用于存储每个 4 位加法器的和值。
    \end{itemize}
    \item \textbf{生成 4 位加法器实例}:
    \begin{itemize}
        \item 使用多个 4 位加法器实例来实现 32 位加法器。
        \item 对于第一个 4 位加法器实例,进位输入 \texttt{ci} 直接连接到模块的 \texttt{ci} 输入。
        \item 对于后续的 4 位加法器实例,进位输入连接到前一个 4 位加法器的进位输出。
    \end{itemize}
    \item \textbf{多路复用器}:
    \begin{itemize}
        \item 每个 4 位加法器实例都有两个版本,一个版本的进位输入为 0,另一个版本的进位输入为 1。
        \item 使用多路复用器选择正确的和输出和进位输出。
    \end{itemize}
    \item \textbf{计算进位输出}:
    \begin{itemize}
        \item 最终的进位输出 \texttt{co} 由最后一个 4 位加法器的进位输出决定。
    \end{itemize}
\end{itemize}

\paragraph{代码实现}
下面是代码的实现:



\begin{lstlisting}[language=Verilog]
//32bit超前进位加法器
module adder_32bits (
    input [31:0] a,   // 32位输入a
    input [31:0] b,   // 32位输入b
    input ci,         // 进位输入
    output [31:0] s,  // 32位和输出
    output co         // 进位输出
);

    wire C3,C7,C11,C15,C19,C23,C27,C31;//级间进位
    wire C7_0,C7_1,C11_0,C11_1,C15_0,C15_1,C19_0,C19_1,C23_0,C23_1,C27_0,C27_1,C31_0,C31_1;//内部进位
    wire [3:0]S3_0,S3_1,S7_0,S7_1,S11_0,S11_1,S15_0,S15_1,S19_0,S19_1,S23_0,S23_1,S27_0,S27_1,S31_0,S31_1;//内部和

    //adder_4bits_X_X表示第x级,ci为X的加法器模块
    //mux_2to1_X表示第x级的mux模块
    //第0级
    adder_4bits adder_4bits_0(.a(a[3:0]),.b(b[3:0]),.ci(ci),.s(s[3:0]),.co(C3));
    //第1级
    adder_4bits adder_4bits_1_1(.a(a[7:4]),.b(b[7:4]),.ci(1'b1),.s(S3_1),.co(C7_1));
    adder_4bits adder_4bits_1_0(.a(a[7:4]),.b(b[7:4]),.ci(1'b0),.s(S3_0),.co(C7_0));
    mux_2to1 mux_2to1_1(.sel(C3),.d0(S3_0),.d1(S3_1),.y(s[7:4]));
    assign C7 = C7_0 | (C7_1 & C3);
    //第2级
    adder_4bits adder_4bits_2_1(.a(a[11:8]),.b(b[11:8]),.ci(1'b1),.s(S7_1),.co(C11_1));
    adder_4bits adder_4bits_2_0(.a(a[11:8]),.b(b[11:8]),.ci(1'b0),.s(S7_0),.co(C11_0));
    mux_2to1 mux_2to1_2(.sel(C7),.d0(S7_0),.d1(S7_1),.y(s[11:8]));
    assign C11 = C11_0 | (C11_1 & C7);
    //第3级
    adder_4bits adder_4bits_3_1(.a(a[15:12]),.b(b[15:12]),.ci(1'b1),.s(S11_1),.co(C15_1));
    adder_4bits adder_4bits_3_0(.a(a[15:12]),.b(b[15:12]),.ci(1'b0),.s(S11_0),.co(C15_0));
    mux_2to1 mux_2to1_3(.sel(C11),.d0(S11_0),.d1(S11_1),.y(s[15:12]));
    assign C15 = C15_0 | (C15_1 & C11);
    //第4级
    adder_4bits adder_4bits_4_1(.a(a[19:16]),.b(b[19:16]),.ci(1'b1),.s(S15_1),.co(C19_1));
    adder_4bits adder_4bits_4_0(.a(a[19:16]),.b(b[19:16]),.ci(1'b0),.s(S15_0),.co(C19_0));
    mux_2to1 mux_2to1_4(.sel(C15),.d0(S15_0),.d1(S15_1),.y(s[19:16]));
    assign C19 = C19_0 | (C19_1 & C15);
    //第5级
    adder_4bits adder_4bits_5_1(.a(a[23:20]),.b(b[23:20]),.ci(1'b1),.s(S19_1),.co(C23_1));
    adder_4bits adder_4bits_5_0(.a(a[23:20]),.b(b[23:20]),.ci(1'b0),.s(S19_0),.co(C23_0));
    mux_2to1 mux_2to1_5(.sel(C19),.d0(S19_0),.d1(S19_1),.y(s[23:20]));
    assign C23 = C23_0 | (C23_1 & C19);
    //第6级
    adder_4bits adder_4bits_6_1(.a(a[27:24]),.b(b[27:24]),.ci(1'b1),.s(S23_1),.co(C27_1));
    adder_4bits adder_4bits_6_0(.a(a[27:24]),.b(b[27:24]),.ci(1'b0),.s(S23_0),.co(C27_0));
    mux_2to1 mux_2to1_6(.sel(C23),.d0(S23_0),.d1(S23_1),.y(s[27:24]));
    assign C27 = C27_0 | (C27_1 & C23);
    //第7级
    adder_4bits adder_4bits_7_1(.a(a[31:28]),.b(b[31:28]),.ci(1'b1),.s(S27_1),.co(C31_1));
    adder_4bits adder_4bits_7_0(.a(a[31:28]),.b(b[31:28]),.ci(1'b0),.s(S27_0),.co(C31_0));
    mux_2to1 mux_2to1_7(.sel(C27),.d0(S27_0),.d1(S27_1),.y(s[31:28]));
    assign C31 = C31_0 | (C31_1 & C27);
    assign co = C31;



endmodule


\end{lstlisting}

\subsubsection{ModelSim 仿真}

对32位进位选择加法器进行功能仿真,仿真结果如下所示。功能仿真结果表示,设计的32位进位选择加法器功能正常

  \begin{figure}[H]
      \centering
      \includegraphics[width =1.0\textwidth]{figures/fig/Snipaste_2024-09-23_09-00-59.png}
      \caption{32位进位选择加法器功能仿真}
      \label{fig:enter-label}
  \end{figure}

  在vivado中进行综合和实现后,对32位进位选择加法器进行时序仿真,仿真结果如下所示。

  如图可以看到时序仿真中信号存在建立时间和毛刺等现象,时序仿真结果表示,设计的32位进位选择加法器在考虑时序和硬件时延的情况下功能正常。


  \begin{figure}[H]
      \centering
      \includegraphics[width =1.0\textwidth]{figures/fig/Snipaste_2024-09-23_09-06-08.png}
      \caption{32位进位选择加法器时序仿真}
      \label{fig:enter-label}
  \end{figure}

  \newpage

\subsection{32位流水线加法器}
\subsubsection{设计原理}
  设计原理如下所示

    \begin{figure}[H]
        \centering
        \includegraphics[width =1.0\textwidth]{figures/fig/Snipaste_2024-09-23_08-54-27.png}
        \caption{32位进位选择加法器设计原理}
        \label{fig:enter-label}
    \end{figure}



\subsubsection{Verilog HDL代码与设计分析}
\paragraph{设计分析}
这段 Verilog 代码实现了一个 32 位四级流水线加法器。该加法器通过将 32 位加法运算分解为多个 8 位加法运算,并在每一级流水线中处理一部分加法运算,从而实现高效的 32 位加法运算。下面是对代码的详细分析:

\begin{itemize}
    \item \textbf{模块接口}:
    \begin{itemize}
        \item \texttt{clk} 是时钟信号,用于同步流水线寄存器。
        \item \texttt{a} 和 \texttt{b} 是 32 位的输入信号,表示两个需要相加的 32 位二进制数。
        \item \texttt{ci} 是进位输入,用于处理低位加法器的进位。
        \item \texttt{s} 是 32 位的和输出,表示加法结果。
        \item \texttt{co} 是进位输出,表示最高位的进位。
    \end{itemize}
    \item \textbf{流水线寄存器}:
    \begin{itemize}
        \item 定义了 5 级流水线寄存器,用于存储中间结果和进位信号。
        \item 每一级流水线寄存器在时钟上升沿时更新数据。
    \end{itemize}
    \item \textbf{8 位加法器实例}:
    \begin{itemize}
        \item 实例化了 4 个 8 位加法器,每个加法器处理 8 位的加法运算。
        \item 每个 8 位加法器的进位输入连接到前一级流水线寄存器的进位输出。
    \end{itemize}
    \item \textbf{计算进位输出}:
    \begin{itemize}
        \item 最终的进位输出 \texttt{co} 由最后一级流水线寄存器的进位输出决定。
    \end{itemize}
    \item \textbf{计算和输出}:
    \begin{itemize}
        \item 每一级流水线寄存器的和输出由当前级的 8 位加法器输出和前一级的部分和组成。
        \item 最终的和输出 \texttt{s} 由最后一级流水线寄存器的和输出决定。
    \end{itemize}
\end{itemize}


\begin{lstlisting}[language=Verilog]
module adder_8bits (
    input [7:0] a,   // 8位输入a
    input [7:0] b,   // 8位输入b
    input ci,         // 进位输入
    output [7:0] s,  // 8位和输出
    output co         // 进位输出
);

    wire [1:0] carry; // 内部进位

    // 生成2个4位加法器实例
    genvar i;
    generate
        for (i = 0; i < 2; i = i + 1) begin : adder_gen
            if (i == 0) begin
                adder_4bits adder_4bits_inst (
                    .a(a[3:0]),
                    .b(b[3:0]),
                    .ci(ci),
                    .s(s[3:0]),
                    .co(carry[0])
                );
            end else begin
                adder_4bits adder_4bits_inst (
                    .a(a[i*4+3:i*4]),
                    .b(b[i*4+3:i*4]),
                    .ci(carry[i-1]),
                    .s(s[i*4+3:i*4]),
                    .co(carry[i])
                );
            end
        end
    endgenerate

    assign co = carry[1];

endmodule

// 32位四级流水线加法器
module pipeline_adder (
    input clk,         // 时钟信号
    input [31:0] a,    // 32位输入a
    input [31:0] b,   // 32位输入b
    input ci,          // 进位输入
    output [31:0] s,  // 32位和输出
    output  co      // 进位输出
);
    // 定义5级流水线寄存器
    reg [31:0] a_reg1, a_reg2, a_reg3, a_reg4, a_reg5;
    reg [31:0] b_reg1, b_reg2, b_reg3, b_reg4, b_reg5;
    reg ci_reg1=0, ci_reg2=0, ci_reg3=0, ci_reg4=0, ci_reg5=0;
    reg [31:0] s_reg1 = 32'h0, s_reg2 = 32'h0, s_reg3 = 32'h0, s_reg4 = 32'h0;
    // 定义4个8位加法器的输入输出
    wire [7:0] s_wire1, s_wire2, s_wire3, s_wire4;
    wire co_wire1, co_wire2, co_wire3, co_wire4;

    // 第一级流水线寄存器
    always @(posedge clk) begin
        a_reg1 <= a;
        b_reg1 <= b;
        ci_reg1 <= ci;
    end

    // 第二级流水线寄存器
    always @(posedge clk) begin
        a_reg2 <= a_reg1;
        b_reg2 <= b_reg1;
        ci_reg2 <= co_wire1;
        s_reg1 <= {24'b0, s_wire1};

    end

    // 第三级流水线寄存器
    always @(posedge clk) begin
        a_reg3 <= a_reg2;
        b_reg3 <= b_reg2;
        ci_reg3 <= co_wire2;
        s_reg2 <= {16'b0, s_wire2, s_reg1[7:0]};

    end

    // 第四级流水线寄存器
    always @(posedge clk) begin
        a_reg4 <= a_reg3;
        b_reg4 <= b_reg3;
        ci_reg4 <= co_wire3;
        s_reg3 <= {8'b0, s_wire3, s_reg2[15:0]};

    end

    // 第五级流水线寄存器
    always @(posedge clk) begin
        a_reg5 <= a_reg4;
        b_reg5 <= b_reg4;
        ci_reg5 <= co_wire4;
        s_reg4 <= {s_wire4, s_reg3[23:0]};

    end

    // 实例化4个8位加法器
    adder_8bits adder_8bits_inst1 (
        .a(a_reg1[7:0]),
        .b(b_reg1[7:0]),
        .ci(ci_reg1),
        .s(s_wire1),
        .co(co_wire1)
    );

    adder_8bits adder_8bits_inst2 (
        .a(a_reg2[15:8]),
        .b(b_reg2[15:8]),
        .ci(ci_reg2),
        .s(s_wire2),
        .co(co_wire2)
    );

    adder_8bits adder_8bits_inst3 (
        .a(a_reg3[23:16]),
        .b(b_reg3[23:16]),
        .ci(ci_reg3),
        .s(s_wire3),
        .co(co_wire3)
    );

    adder_8bits adder_8bits_inst4 (
        .a(a_reg4[31:24]),
        .b(b_reg4[31:24]),
        .ci(ci_reg4),
        .s(s_wire4),
        .co(co_wire4)
    );

    // 输出结果
    assign s = s_reg4;
    assign co = ci_reg5;

endmodule



\end{lstlisting}

\subsubsection{ModelSim 仿真}

对32位流水线加法器进行功能仿真,仿真结果如下所示。功能仿真结果表示,设计的32位流水线加法器功能正常,在第一个数据到达后,经过四个时钟周期后,输出了正确的结果。并实现了流水线的效果。

  \begin{figure}[H]
      \centering
      \includegraphics[width =1.0\textwidth]{figures/fig/Snipaste_2024-09-23_09-32-47.png}
      \caption{32位流水线加法器功能仿真}
      \label{fig:enter-label}
  \end{figure}

  \newpage
  在vivado中进行综合和实现后,对32位流水线加法器进行时序仿真,仿真结果如下所示。

  如图可以看到时序仿真中信号存在建立时间和时钟上升沿后的时延等现象,时序仿真结果表示,设计的32位流水线加法器在考虑时序和硬件时延的情况下功能正常。


  \begin{figure}[H]
      \centering
      \includegraphics[width =1.0\textwidth]{figures/fig/Snipaste_2024-09-23_09-43-14.png}
      \caption{32位进位选择加法器时序仿真}
      \label{fig:enter-label}
  \end{figure}

\section{遇到的问题及解决}
在做流水线的时序仿真时,发现每次第一个信号经过四个时钟周期后才能输出结果为0,但是后续的信号经过四个时钟周期后能够输出正确的结果。

经过研究发现,这跟GSR有关系,在综合后和实现后的时序仿真中,会自动触发全局置位/复位脉冲(GSR),这会让所有的寄存器在仿真的前100ns内锁定其值
在timing-based simulation中,仿真器会等待这个100ns的GSR释放后才会使激励生效。而功能仿真由于跟timing和器件特性无关,因此不会有这个问题。

一开始给的tb文件在45ns就给了输入,而在45ns时,GSR还没有释放,所以仿真结果会丢失第一个信号。解决方法是修改tb文件,仿真开始的时候等待100ns以上,等GSR释放后再给输入信号。
\newpage
\section{思考题}

\subsection{为什么要进行时序仿真}
\begin{itemize}
    \item \textbf{验证时序特性}:时序仿真可以验证设计是否满足时序要求,包括建立时间、保持时间和时钟周期等。通过时序仿真,可以确保电路在实际工作中不会出现时序违例。
    \item \textbf{检测时序问题}:时序仿真可以帮助发现设计中的时序问题,如竞争条件、毛刺和时钟偏斜等。这些问题在功能仿真中可能无法检测到,但在实际电路中会导致不稳定或错误的行为。
    \item \textbf{优化设计性能}:通过时序仿真,可以分析电路的时序路径,找出关键路径和瓶颈,从而优化设计,提高电路的工作频率和性能。
    \item \textbf{确保设计可靠性}:时序仿真可以验证设计在不同工艺角、温度和电压条件下的性能,确保电路在各种工作环境下都能正常运行,提高设计的可靠性。
\end{itemize}
\subsection{采用流水线技术有什么优缺点}
\begin{itemize}
    \item \textbf{优点}:
    \begin{itemize}
        \item \textbf{提高吞吐量}:流水线技术可以在每个时钟周期内同时处理多条指令,从而显著提高处理器的吞吐量和整体性能。
        \item \textbf{提高资源利用率}:通过将指令执行过程分解为多个阶段,流水线技术可以更有效地利用处理器资源,如算术逻辑单元(ALU)、寄存器和存储器等。
        \item \textbf{缩短指令执行时间}:流水线技术可以将指令执行时间分摊到多个时钟周期内,从而缩短每条指令的平均执行时间,提高处理器的响应速度。
    \end{itemize}
    \item \textbf{缺点}:
    \begin{itemize}
        \item \textbf{增加设计复杂度}:流水线技术需要对指令执行过程进行精细的分解和调度,增加了处理器设计的复杂度和实现难度。
        \item \textbf{引入流水线冒险}:流水线技术会引入数据冒险、控制冒险和结构冒险等问题,需要额外的硬件和控制逻辑来解决这些问题,从而增加了设计的复杂性和功耗。
        \item \textbf{增加分支延迟}:在流水线中处理分支指令时,需要等待分支条件的计算结果,从而引入分支延迟,影响处理器的性能。
    \end{itemize}
\end{itemize}




%%----------- 参考文献 -------------------%%
%在reference.bib文件中填写参考文献,此处自动生成




\end{document}