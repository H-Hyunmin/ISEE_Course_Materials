%!TeX program = xelatex
\documentclass[12pt,hyperref,a4paper,UTF8]{ctexart}
\usepackage{zjureport}
\usepackage{listings}
\usepackage{enumitem}
\usepackage{float}
\usepackage{xcolor}
\usepackage{amsmath}
\lstset{
    %backgroundcolor=\color{red!50!green!50!blue!50},%代码块背景色为浅灰色
    rulesepcolor= \color{gray}, %代码块边框颜色
    breaklines=true,  %代码过长则换行
    numbers=left, %行号在左侧显示
    numberstyle= \small,%行号字体
    keywordstyle= \color{blue},%关键字颜色
    commentstyle=\color{gray}, %注释颜色
    frame=shadowbox%用方框框住代码块
    }


%%-------------------------------正文开始---------------------------%%
\begin{document}

%%-----------------------封面--------------------%%
\cover

%%------------------摘要-------------%%
%\begin{abstract}
%
%在此填写摘要内容
%
%\end{abstract}

\thispagestyle{empty} % 首页不显示页码

%%--------------------------目录页------------------------%%
\newpage
\tableofcontents

%%------------------------正文页从这里开始-------------------%
\newpage

%%可选择这里也放一个标题
%\begin{center}
%    \title{ \Huge \textbf{{标题}}}
%\end{center}
\section{实验介绍}
\subsection*{实验简介}
本课程为《大数据》实验课程,模拟高校学生系统,在华为MRS上进行HBase操作,通过创建表、对表进行CURD操作、创建预分region表及Filter过滤器的使用,让学员达到在MRS上熟练掌握KV查询.
\subsection*{实验流程}
\begin{figure}[H]
    \centering
    \includegraphics[width=0.55\textwidth]{figures/fig/图片1.png}
    \caption{实验流程图}
    \label{fig:1}
\end{figure}

\subsection*{实验环境}
为了满足基于HBase-MRS实验需要,提前注册华为云账号并充值,确保通过PUTTY访问外网畅通。购买MapReduce 1个
购买弹性IP 1个


\section{实验准备:申请弹性公网IP和创建虚拟云}
\subsection*{申请弹性公网IP}
\begin{enumerate}
    \item 选择弹性公网IP
    \item 点击右上角的“购买弹性公网IP”
    \item 相关配置如图
    \begin{figure}[H]
        \centering
        \includegraphics[width=0.8\textwidth]{figures/fig/Snipaste_2024-06-06_21-08-56.png}
        \caption{购买弹性公网IP}
        \label{fig:4}
    \end{figure}
    \item 查看购买的弹性公网IP
\end{enumerate}

\subsection*{创建虚拟云}

\begin{enumerate}
    \item 进入虚拟私有云控制台后,点击右上角的“创建虚拟私有云”。
    \item 配置基本信息
    \begin{figure}[H]
        \centering
        \includegraphics[width=0.78\textwidth]{figures/fig/Snipaste_2024-06-06_19-20-25.png}
        \caption{配置虚拟云}
        \label{fig:6}
    \end{figure}
    \item 查看创建的VPC
    \begin{figure}[H]
        \centering
        \includegraphics[width=0.8\textwidth]{figures/fig/Snipaste_2024-06-06_19-20-43.png}
        \caption{查看创建的VPC}
        \label{fig:7}
    \end{figure}
\end{enumerate}



\section{实验一:购买MapReduce服务}


\begin{enumerate}
    \item 打开MRS服务申请集群
    \item 集群配置:配置如下
    \begin{figure}[H]
        \centering
        \includegraphics[width=0.78\textwidth]{figures/fig/Snipaste_2024-06-06_20-35-03.png}
        \caption{集群配置}
        \label{fig:8}
    \end{figure}
    \item 点击集群名称,“前往Manager”。输入用户名admin及密码,点击“登录”,进入MRS Manager
    \begin{figure}[H]
        \centering
        \includegraphics[width=0.78\textwidth]{figures/fig/Snipaste_2024-06-06_21-42-18.png}
        \caption{进入MRS Manager}
        \label{fig:8}
    \end{figure}    
\end{enumerate}


\section{实验二:远程登录云服务器操作HBase}

\begin{enumerate}
    \item 配置安全组,选择“节点管理”,点击含有“master1”的节点,在弹出页面中选择“安全组”,点击“更改安全组规则”,选择“入方向规则”,点击“一键放通”,确认即可。
    \item 远程登录master节点,在安全组配置项,点击右上方“远程登录”,选择cloudshell登录。输入密码,点击连接即可。
    \begin{figure}[H]
        \centering
        \includegraphics[width=0.85\textwidth]{figures/fig/Snipaste_2024-06-06_21-45-14.png}
        \caption{远程连接登入}
        \label{fig:8}
    \end{figure}
    \item 设置环境变量
    \begin{verbatim}
# source env_file 
# hbase shell
    \end{verbatim}
    \item 创建基表
    \begin{verbatim}
# create 'cx_table_stu01', 'cf1'
    \end{verbatim}
    \item list:显示所有的表
    \begin{verbatim}
# create 'cx_table_stu01', 'cf1'
    \end{verbatim}
    如下图,创建了一个名为cx\_table\_stu01的空表,list指令显示了当前有一个表cx\_table\_stu01
    \begin{figure}[H]
        \centering
        \includegraphics[width=0.85\textwidth]{figures/fig/Snipaste_2024-06-06_22-00-04.png}
        \caption{创建和显示表}
        \label{fig:8}
    \end{figure}
    \item 增加数据
    \begin{verbatim}
# put 'cx_table_stu01','20200001','cf1:name','tom'
# put 'cx_table_stu01','20200001','cf1:gender','male'
# put 'cx_table_stu01','20200001', 'cf1:age','20'
# put 'cx_table_stu01','20200002', 'cf1:name','hanmeimei'
# put 'cx_table_stu01','20200002', 'cf1:gender','female'
# put 'cx_table_stu01','20200002', 'cf1:age','19'
# scan 'cx_table_stu01'
    \end{verbatim}
    如下图,向表中添加了两条数据。第一条数据的rowkey为20200001,包含name、gender、age三个列族,第二条数据的rowkey为20200002,包含name、gender、age三个列族。scan指令显示了表中的所有数据
    \begin{figure}[H]
        \centering
        \includegraphics[width=0.85\textwidth]{figures/fig/Snipaste_2024-06-06_22-01-09.png}
        \caption{向表中添加数据}
        \label{fig:8}
    \end{figure} 
    \item Scan方式查询数据
    \begin{verbatim}
#  scan 'cx_table_stu01',{COLUMNS=>'cf1'}
# scan 'cx_table_stu01',{COLUMNS=>'cf1:name'}
    \end{verbatim}
    如下图,scan指令可以查询表中的所有数据,也可以指定列族查询数据
    \begin{figure}[H]
        \centering
        \includegraphics[width=0.85\textwidth]{figures/fig/Snipaste_2024-06-06_22-04-46.png}
        \caption{scan查询数据结果}
        \label{fig:8}
    \end{figure} 
    \item Get方式查询数据
    \begin{verbatim}
# get 'cx_table_stu01','20200001'
# get 'cx_table_stu01','20200001','cf1:name'
    \end{verbatim}
    \item 指定条件查询数据
    \begin{verbatim}
#  scan 'cx_table_stu01',{STARTROW=>'20200001','LIMIT'=>2,STOPROW=>'20200002'}
    \end{verbatim}
    如下图,get指令可以查询指定rowkey的数据,也可以指定列族查询数据。scan指令可以指定条件查询数据,可以指定起始行、终止行、查询条数等条件
    \begin{figure}[H]
        \centering
        \includegraphics[width=0.85\textwidth]{figures/fig/Snipaste_2024-06-06_22-07-53.png}
        \caption{get与指定条件查询数据结果}
        \label{fig:8}
    \end{figure} 
    \item 过滤器查询
    \begin{verbatim}
# scan 'cx_table_stu01',{FILTER=>"ValueFilter(=,'binary:20')"}
# scan 'cx_table_stu01',{FILTER=>"ValueFilter(=,'binary:tom')"}
# scan 'cx_table_stu01', {FILTER => "ColumnPrefixFilter('gender')"}
# scan 'cx_table_stu01',{FILTER=>"ColumnPrefixFilter('name') 
    AND ValueFilter(=,'binary:hanmeimei')"}
    \end{verbatim}    
    如下图,过滤器查询可以通过指定条件查询数据,可以通过值、列族、列名等条件查询数据
    \begin{figure}[H]
        \centering
        \includegraphics[width=0.85\textwidth]{figures/fig/Snipaste_2024-06-06_22-18-28.png}
        \caption{过滤器查询结果}
        \label{fig:8}
    \end{figure} 
    \item 查询多版本数据,Hbase可以存储历史版本的数据,通过设定版本号VERSIONS值,版本号为几就是存储几个版本。
    \begin{verbatim}
    增加数据执行命令:
# put 'cx_table_stu01','20200001','cf1:name','ZhangSan'
# put 'cx_table_stu01','20200001','cf1:name','LiSi'
# put 'cx_table_stu01','20200001','cf1:name','WangWu'
    然后get时指定多版本查询:
# get 'cx_table_stu01','20200001',{COLUMNS=>'cf1',VERSIONS=>5}
    \end{verbatim}      
    如下图,添加不同版本的数据,然后查询多版本数据,可以看到get指令只查询了20200001版本的数据 
    \begin{figure}[H]
        \centering
        \includegraphics[width=0.85\textwidth]{figures/fig/Snipaste_2024-06-06_22-22-45.png}
        \caption{多版本查询}
        \label{fig:8}
    \end{figure}    
    \item 查看表的属性,执行语句:
    \begin{verbatim}
#desc 'cx_table_stu01'
    \end{verbatim}       
    如下图,可以看到表的属性,包括表名、列族、版本数等信息
    \begin{figure}[H]
        \centering
        \includegraphics[width=0.85\textwidth]{figures/fig/Snipaste_2024-06-06_22-30-56.png}
        \caption{表属性查看}
        \label{fig:8}
    \end{figure} 

    \item 查看多个版本的数据:需要修改表的VERSIONS值(或者在建表的时候就指定VERSIONS值)
    \begin{verbatim}
# alter 'cx_table_stu01',{NAME=>'cf1','VERSIONS'=>5}
# put 'cx_table_stu01','20200001','cf1:name','ZhangSan' 
# put 'cx_table_stu01','20200001','cf1:name','LiSi'
# put 'cx_table_stu01','20200001','cf1:name','WangWu'
# get 'cx_table_stu01','20200001',{COLUMN=>'cf1',VERSIONS=>5}
    \end{verbatim}  
    alter指令可以修改表的属性,包括列族、版本数等信息。如下图,修改了表的版本数为5,然后添加了不同版本的数据,查询多版本数据,可以看到get指令查询了20200001版本的数据
    \begin{figure}[H]
        \centering
        \includegraphics[width=0.85\textwidth]{figures/fig/Snipaste_2024-06-06_22-32-57.png}
        \caption{多版本数据查询}
        \label{fig:8}
    \end{figure} 
    
    
    \item 删除数据
    \begin{verbatim}
删除某列族下数据,执行命令:
# delete 'cx_table_stu01','20200002','cf1:age'
# get 'cx_table_stu01','20200002'
 
删除整行数据,执行命令:
# deleteall 'cx_table_stu01','20200002'
# get 'cx_table_stu01','20200002'
    \end{verbatim} 
    \begin{figure}[H]
        \centering
        \includegraphics[width=0.85\textwidth]{figures/fig/Snipaste_2024-06-06_22-36-13.png}
        \caption{删除表中数据}
        \label{fig:8}
    \end{figure} 


    \item 删除表
    \begin{verbatim}
用drop命令可以删除表。但是在删除一个表之前必须先将其禁用。
第一步:disable“表名称”
# disable ‘cx_table_stu01’
第二步:drop“表名称”
# drop ‘cx_table_stu01’
    \end{verbatim} 
    \begin{figure}[H]
        \centering
        \includegraphics[width=0.85\textwidth]{figures/fig/Snipaste_2024-06-06_22-41-23.png}
        \caption{删除表}
        \label{fig:8}
    \end{figure} 

\end{enumerate}





\section{实验三:创建预分region表}
HBase默认建表时只有一个region,这个region的rowkey是没有边界的,即没有startkey,也没有endkey。在数据写入时,所有数据都会写入这个默认的region,随着数据量的不断增加,此region已经不能承受不断增长的数据量,会进行split,分成2个region。在此过程中,会产生两个问题:
\begin{enumerate}
    \item 数据往一个region上写,会有写热点问题。
    \item region split会消耗宝贵的集群I/O资源。
\end{enumerate}


基于此我们可以在建表的时候,创建多个空region,并确定每个region的起始和终止rowky,这样只要我们的rowkey设计能均匀的命中各个region,就不会存在写热点问题,自然split的几率也会大大降低。hbase提供了两种pre-split算法:HexStringSplit和UniformSplit,前者适用于十六进制字符的rowkey,后者适用于随机字节数组的rowkey。以rowkey切分,随机分为4个region。
以rowkey切分,随机分为4个region。

\begin{enumerate}
    \item 创建表执行命令:
    \begin{verbatim}
# create 'cx_table_stu02','cf2',{NUMREGIONS => 4,SPLITALGO =>'UniformSplit'}
Region命名格式:[table],[region start key],[region id]
    \end{verbatim} 
    \begin{figure}[H]
        \centering
        \includegraphics[width=0.85\textwidth]{figures/fig/Snipaste_2024-06-06_22-45-45.png}
        \caption{创建表}
        \label{fig:8}
    \end{figure} 


    \item 登录HBase WebUI查看表的分区情况\\
    MRS Manager界面,点击“HBase”服务,点击HMaster(主)进入HBase UI,“User Tables”下点击创建好的表名“cxtablestu02”
    \begin{figure}[H]
        \centering
        \includegraphics[width=0.8\textwidth]{figures/fig/Snipaste_2024-06-06_22-50-22.png}
        \caption{查看该表分区情况}
        \label{fig:8}
    \end{figure} 


    \item 指定region的startKey和endKey
    \begin{verbatim}
创建表执行命令:
# create 'cx_table_stu03', 'cf3', SPLITS => ['10000', '20000', '30000']
    \end{verbatim} 
    \begin{figure}[H]
        \centering
        \includegraphics[width=0.85\textwidth]{figures/fig/Snipaste_2024-06-06_22-54-47.png}
        \caption{创建表}
        \label{fig:8}
    \end{figure} 
    同理,查看该表分区情况:
    \begin{figure}[H]
        \centering
        \includegraphics[width=0.8\textwidth]{figures/fig/Snipaste_2024-06-06_22-55-08.png}
        \caption{查看该表分区情况}
        \label{fig:8}
    \end{figure} 
\end{enumerate}


\section{实验结束:云服务资源释放}

\subsection*{释放MapReduce服务}
\begin{enumerate}
    \item 进入MRS控制台,点击集群后的删除按钮
\end{enumerate}

\subsection*{释放网络资源VPC}
\begin{enumerate}
    \item 进入VPC控制台界面
    \item 删除IP:点击“弹性公网IP和带宽”,选择弹性公网IP,选中所有IP,点击释放按钮.
    \item 删除安全组规则:点击“访问控制”,选择安全组,点击安全组名称,选择入方向规则,点击列表后的删除按钮。
\end{enumerate}




\section{实验总结与心得体会}

在这次KV查询实验中,我通过模拟高校学生系统,在华为MRS平台上进行了HBase操作,主要包括创建表、CURD操作、创建预分region表及使用Filter过滤器等内容。整个实验过程让我深刻理解了HBase的基础操作以及在实际应用中的重要性。

首先,实验准备阶段,我体验了从申请弹性公网IP到创建虚拟私有云的完整过程。在此过程中,我通过华为云控制台完成了相关配置,成功购买了弹性公网IP和虚拟私有云。这不仅让我熟悉了华为云平台的操作,也让我认识到在大数据处理环境中,网络资源的配置和管理是基础且关键的一步。通过这一过程,我深刻体会到了云计算资源的灵活性和便捷性,同时也学会了如何高效地管理和使用这些资源。

在购买MapReduce服务并配置集群时,我进一步加深了对大数据处理框架的理解。通过配置集群并进入MRS Manager,我了解了集群管理的基本操作。这一步骤让我认识到,合理配置和管理集群对于大数据处理的效率和性能至关重要。在实际操作中,我学会了如何根据实验要求配置集群参数,以及如何通过MRS Manager进行集群管理和监控。

实验二的重点是远程登录云服务器并进行HBase操作。这部分内容让我对HBase的基本操作有了全面的了解。从配置安全组、远程登录master节点到设置环境变量,再到创建表、增加数据、查询数据、删除数据等一系列操作,每一步都让我对HBase的工作原理有了更深入的理解。例如,通过创建一个简单的表并向其中添加数据,我体会到了HBase在处理大规模数据时的高效性和灵活性。此外,通过使用不同的查询命令(如scan和get),我学会了如何根据具体需求灵活查询数据。而在使用过滤器进行数据查询时,我认识到了HBase在数据过滤和查询方面的强大功能,这对于实际应用中快速定位和提取数据非常有帮助。

实验三中创建预分region表的操作让我认识到在大数据处理过程中如何避免写热点问题。通过预先定义多个region并设置rowkey的起始和终止值,我们可以有效分散数据写入的压力,避免单一region过载问题。这不仅提高了数据写入的效率,也减少了region split对集群资源的消耗。通过这一操作,我深刻体会到在实际应用中,合理设计和管理数据存储结构对于提高系统性能至关重要。

在实验过程中遇到的一些问题和挑战也让我受益匪浅。例如,在配置虚拟私有云时,由于对华为云平台的操作不够熟悉,我曾一度在网络配置上遇到困难。但通过查阅相关资料并不断尝试,我最终解决了这些问题。这不仅增强了我的问题解决能力,也让我对华为云平台有了更深入的了解。

此外,在操作HBase时,由于对HBase的命令不够熟悉,我曾在数据查询和过滤器使用上遇到了一些问题。但通过反复练习和实验,我逐渐掌握了这些命令的使用方法。这让我认识到,在学习和掌握新技术时,动手实践和反复操作是非常重要的。

通过这次实验,我不仅掌握了HBase的基本操作和使用技巧,也对大数据处理的整个流程有了更全面的理解。从申请和配置云资源,到管理和操作HBase,再到解决实际操作中遇到的问题,每一步都让我受益匪浅。这些知识和技能不仅对我今后的学习和工作非常有帮助,也让我对大数据处理技术充满了兴趣和信心。

总的来说,这次KV查询实验不仅让我掌握了许多实用的技能,也让我对大数据处理技术有了更深刻的理解和认识。在未来的学习和工作中,我将继续深入学习和探索大数据处理技术,不断提升自己的专业能力和水平。通过这次实验,我深刻体会到,只有不断实践和总结,才能真正掌握和运用所学的知识,实现自己的职业目标和价值。






%%----------- 参考文献 -------------------%%
%在reference.bib文件中填写参考文献,此处自动生成




\end{document}