%!TeX program = xelatex
\documentclass[12pt,hyperref,a4paper,UTF8]{ctexart}
\usepackage{zjureport}
\usepackage{listings}
\usepackage{enumitem}
\usepackage{float}
\usepackage{xcolor}
\usepackage{graphicx}
\usepackage{hyperref}
\usepackage{subcaption}
\lstset{
    %backgroundcolor=\color{red!50!green!50!blue!50},%代码块背景色为浅灰色
    rulesepcolor= \color{gray}, %代码块边框颜色
    breaklines=true,  %代码过长则换行
    numbers=left, %行号在左侧显示
    numberstyle= \small,%行号字体
    keywordstyle= \color{blue},%关键字颜色
    commentstyle=\color{gray}, %注释颜色
    frame=shadowbox%用方框框住代码块
    }


%%-------------------------------正文开始---------------------------%%
\begin{document}

%%-----------------------封面--------------------%%
\cover

%%------------------摘要-------------%%
%\begin{abstract}
%
%在此填写摘要内容
%
%\end{abstract}

\thispagestyle{empty} % 首页不显示页码

%%--------------------------目录页------------------------%%
\newpage
\tableofcontents

%%------------------------正文页从这里开始-------------------%
\newpage

%%可选择这里也放一个标题
%\begin{center}
%    \title{ \Huge \textbf{{标题}}}
%\end{center}

\section{项目介绍}



\subsection*{项目背景}
在当今社会,随着城市化进程和生活节奏的加快,外卖服务已经成为了现代都市人生活中不可或缺的一部分。

然而,随着外卖服务的普及,也出现了一系列新的问题,其中之一便是外卖送达后的管理和保温难题。当外卖送达到用户手中后,如果用户无法及时领取,食物很可能会在短时间内迅速冷却,从而影响口感。此外,经常存在用户取餐后不关门,忘记关门登现象,影响后续外卖柜的使用。


\subsection*{项目简介}
为了解决这一问题,智能外卖柜应运而生。智能外卖柜是一种利用物联网技术的创新项目,旨在提供一种智能化的解决方案,解决外卖送达后的保温和提醒问题。该项目通过将外卖柜与用户手机连接,实现了实时监控和远程控制功能。当外卖送达时,外卖柜能够通过手机通知用户,提醒他们尽快领取外卖,以避免食物冷却影响口感。同时,用户可以通过手机小程序随时监测外卖柜的温度和湿度情况,并根据需要进行调整,以确保外卖的质量和安全。此外,我们设置了OLED屏幕用于在设备端实时显示外卖柜状态,也为外卖柜增加了语音助手提醒用户取餐后及时关门。

这种智能化的解决方案旨在提升用户体验,解决外卖送达后的管理难题,为用户提供更加便利和安心的外卖体验,同时也为外卖平台和餐饮商户提供了一种全新的管理和服务模式。

相较于传统外卖柜,智能外卖柜具有多重优势:用户可以通过手机小程序实时监控外卖柜的温度和湿度情况,并通过远程控制功能调整目标温度和湿度,以确保外卖处于适宜的保存状态;外卖送达后,外卖柜会自动发送通知给用户,提醒其及时领取外卖,避免食物冷却影响口感和食品安全。
\begin{figure}[H]
    \centering
    \includegraphics[width =.8\textwidth]{figures/fig/d7020f9157bda80ac815774e5ac4acf.png}
    \caption{项目流程图}
    \label{fig:enter-label}
\end{figure}



\section{硬件架构与原理图}
\subsection*{设计思路}
本智能外卖柜目前处于早期演示阶段,采用快递盒进行改装。

外卖柜主体采用一个去口的纸板箱,为了考虑承重和受力情况,设计了双层柜门,并在中间采用了三角结构以增强柜门的强度。为了安装风扇和布线,柜体进行了部分开孔处理。在布线过程中,我们尽量避免折线,优先选择贴边进行布线,并使用锡纸进行固定。在一些脆弱的接口处,我们采用了透明胶进行固定,以增强整体的稳固性。
\begin{figure}[H]
    \centering
    \includegraphics[width =.8\textwidth]{figures/fig/main.jpeg}
    \caption{整体视图}
    \label{fig:enter-label}
\end{figure}
\subsection*{顶部视图}
顶部放置了Arduino板作为主控,使得走线最为密集。

在右上方的I2C接口分别连接了红外温度传感器和OLED显示器,右侧连接了风扇、灯和12V电源(同时将12V引线连接至加热板)。左侧接口分别连接了DHT11传感器、控制语音模块的输出、控制锁和加热器继电器的输出以及锁的返回输入。此外,顶部还放置了连接语音模块的扬声器。

\begin{figure}[H]
    \centering
    \includegraphics[width =.8\textwidth]{figures/fig/up view.jpeg}
    \caption{顶部视图}
    \label{fig:enter-label}
\end{figure}


\subsection*{侧视与后视}
左侧面板使用双面胶固定了语音模块和用于控制加热的继电器。左上角有一孔用于与内部走线,中间有一孔用于手动开锁便于调试。

\begin{figure}[H]
    \centering
    \includegraphics[width =.8\textwidth]{figures/fig/left view.jpeg}
    \caption{左侧视图}
    \label{fig:enter-label}
\end{figure}

右侧面板安装了OLED显示,开孔嵌入了风扇,同时走12V电源线。
\begin{figure}[H]
    \centering
    \includegraphics[width =.8\textwidth]{figures/fig/right view.jpeg}
    \caption{右侧视图}
    \label{fig:enter-label}
\end{figure}
后面板走用于给锁提供电源的12V电源线和用于走电源线的开孔。

\begin{figure}[H]
    \centering
    \includegraphics[width =.8\textwidth]{figures/fig/back view.jpeg}
    \caption{后侧视图}
    \label{fig:enter-label}
\end{figure}

\subsection*{内部视图}
内部走线主要集中在底侧和后侧,以避免空间干扰并减少与加热器靠近造成的安全隐患。左侧安装了锁头。

底板靠左侧设有集中走线,并与前述左侧面板的开孔相通,同时使用双面胶固定了控制锁通断的继电器。底板靠右侧固定了DHT11温湿度传感器,底部安置了加热器,顶部安装了红外温度传感器,其角度经过精心设计以直接测量加热板的温度。此外,顶部还安置了灯。风扇和传感器全部围绕加热板放置,以提高控制的精确性。

\begin{figure}[H]
    \centering
    \includegraphics[width =.8\textwidth]{figures/fig/inside view.jpeg}
    \caption{内部视图}
    \label{fig:enter-label}
\end{figure}

\newpage



\section{嵌入式代码}
\textbf{\large 详细源码参加附录}

\subsection*{代码结构}

\begin{itemize}
  \item 代码结构如下图所示
  \item function源文件中存放相关函数和接口
  \item combine源文件为主文件,包括变量定义,setup初始化和主函数代码等
  \item vscode为项目配置文件,build中为项目构建生成文件。
\end{itemize}

    \begin{figure}[H]
        \centering
        \includegraphics[width =.8\textwidth]{figures/fig/Snipaste_2024-04-28_17-27-32.png}
        \caption{代码结构}
        \label{fig:enter-label}
    \end{figure}



\subsection*{宏和变量介绍}

\begin{itemize}
  \item 代码中的宏定义和变量主要分为以下几个部分:
  \item 头文件部分
  \item wifi和mqtt参数所需宏定义
  \item 引脚宏定义 (所有用到的外设引脚的定义)
  \item AT指令相关宏定义
  \item 变量定义
\end{itemize}

宏定义相关内容参见源码,现介绍程序的主要变量:
\begin{lstlisting}[language=C++]
// Arduino json解析函数需要的对象
StaticJsonDocument<300> doc;
String data; // 存放json的string

float temperature;//temperature data
float humidity;//humidity data
int lockStatus;//lock status, 0 is lock, 1 is unlock
int hasTakeout;//takeout status, 0 is none, 1 is positive
int heaterStatus;//0 is off, 1 is on
int fanStatus;//0 is off, 1 is on
float targetTemperature;
float targetHumidity;
int voiceOutput;//0 is no output,,
int lightswitch;//0 is off, 1 is on
unsigned long timeStart;
//dht11传感器对象
dht11 DHT;
//oled对象
U8G2_SSD1306_128X64_NONAME_1_HW_I2C u8g2(U8G2_R0, /* reset=*/ U8X8_PIN_NONE);
//infrared temp红外传感器对象
Adafruit_MLX90614 mlx = Adafruit_MLX90614(); 
  
\end{lstlisting}








\subsection*{函数介绍}
\subsection*{初始化函数声明如下:}
\begin{itemize}
  \item 该部分函数主要用于setup中对所有硬件和外设进行初始化操作
  \item 具体函数实现可查看附录中的源码
\end{itemize}

\begin{lstlisting}[language=C++]

/**
 * @brief 初始化数据变量。
 */
void dataInit();

/**
 * @brief 初始化语音控制。
 */
void voiceInit();

/**
 * @brief 初始化灯光控制。
 */
void lightInit();

/**
 * @brief 初始化风扇控制。
 */
void fanInit();

/**
 * @brief 初始化 OLED 显示。
 */
void oledInit();

/**
 * @brief 初始化锁控制。
 */
void lockInit();

/**
 * @brief 初始化加热器控制。
 */
void heaterInit();

/**
 * @brief 初始化红外温度传感器。
 */
void itemperatureInit();
  
\end{lstlisting}


\subsection*{硬件控制函数声明如下:}
\begin{itemize}
  \item 该部分函数主要用于操控外设
  \item 该部分函数会在主循环中由updateDate和getData函数调用来根据参数控制相关的外设硬件。
  \item 具体函数实现可查看附录中的源码
\end{itemize}

\begin{lstlisting}[language=C++]

/**
 * @brief 控制语音播报。
 * 
 * @param voice 控制语音的参数,1 表示播报。
 */
void voiceCtrl(int voice);

/**
 * @brief 控制灯光状态。
 * 
 * @param light 控制灯光的参数,1 表示打开,0 表示关闭。
 */
void lightCtrl(int light);

/**
 * @brief 控制风扇状态。
 * 
 * @param status 控制风扇的状态,1 表示打开,0 表示关闭。
 */
void fanCtrl(int status);

/**
 * @brief 获取 DHT11 温湿度传感器的数据。
 */
void getDHT11Data();

/**
 * @brief 在 OLED 显示屏上显示内容。
 */
void oledDisplay();

/**
 * @brief 在 OLED 显示屏上显示温湿度信息。
 */
void oledDisplay_th();

/**
 * @brief 控制锁的状态。
 * 
 * @param status 控制锁的状态,1 表示解锁,0 表示上锁。
 */
void lockCtrl(int status);

/**
 * @brief 检查锁的状态。
 */
void lockCheck();

/**
 * @brief 控制加热器状态。
 * 
 * @param status 控制加热器的状态,1 表示打开,0 表示关闭。
 */
void heaterCtrl(int status);

/**
 * @brief 读取红外温度传感器的数据。
 */
void itemperatureRead();

  
\end{lstlisting}

\subsection*{功能函数声明和实现如下}
\begin{itemize}
  \item Upload,Ali$\underline{}$connect,WiFi$\underline{}$init用于连接wifi和阿里云以及上传参数,具体实现与之前的报告相似,不多赘述,完整函数实现参见附录源码。该部分函数主要用于操控外设
\end{itemize}



\begin{lstlisting}[language=C++]

  /**
 * @brief 获取所需参数值并调用控制函数。
 */
void getData()
{
    getDHT11Data();
    itemperatureRead();
    lockCheck();
}

/**
 * @brief 根据设置的参数控制外设并调用控制函数。
 */
void updateData()
{
    //fan
    if (humidity < targetHumidity - 3) {
        fanStatus = 0;
    } else if (humidity > targetHumidity + 3) {
        fanStatus = 1;
    }
    fanCtrl(fanStatus);

    //heater
    if (temperature < targetTemperature - 3) {
        heaterStatus = 1;
    } else if (temperature > targetTemperature + 3) {
        heaterStatus = 0;
    }
    heaterCtrl(heaterStatus);

    //lock
    if(lockStatus == 1){
        lockCtrl(1);
    }else{
        lockCtrl(0);
    }
    //oled
    oledDisplay_th();
}
/**
 * @brief 上传参数至阿里云。
 * 
 * @return 返回上传是否成功的布尔值。
 */
bool Upload();

/**
 * @brief 连接阿里云。
 * 
 * @return 返回连接是否成功的布尔值。
 */
bool Ali_connect();

/**
 * @brief 初始化 WiFi。
 * 
 * @return 返回初始化是否成功的布尔值。
 */
bool WiFi_init();

  
\end{lstlisting}

\subsection*{主函数介绍}
\begin{itemize}
  \item setup函数主要进行各种初始化操作。
  \item loop函数为程序主体代码
  \item loop持续检测阿里云下发的消息并且持续更新硬件设备的状态,定时每隔五秒向阿里云上报硬件参数。
\end{itemize}


\begin{lstlisting}[language=C++]
  /**
 * @brief 初始化串口,连接wifi和阿里云,调用函数进行参数和外设的初始化。
 */
void setup()
{
  timeStart = millis();
  // Serial Initial
  Serial3.begin(115200);
  Serial.begin(115200);
  String inString = "";

  //  Pin_init();
  Serial.println("Setup Start");

  // Cloud Initial
  while (1)
  {
    if (!WiFi_init())
      continue;
    Serial.println("WiFi Connected");
    if (!Ali_connect())
      continue;
    break;
  }
  Serial.println("Ali Connected");

  dataInit();
  fanInit();
  oledInit();
  lockInit();
  heaterInit();
  lightInit();
  itemperatureInit();
  voiceInit();
  Serial.println("dataInit Done");
}

/**
 * @brief 主函数
 */
void loop()
{

  delay(10);
  String inString = "";

  //接受阿里云下发的参数
  if (Serial3.available() > 0)
  {
    delay(10);
    inString = Serial3.readString();
    if (inString.indexOf("MQTTRECV") != -1)
    {
      Serial.println("==============print aliyun receive data================");
      Serial.print("json:");
      Serial.println(inString);
      data = inString;
      parse(data);
      Serial.println("==================End====================\n\n\n");
    }
  }

  //持续更新数据
  if (Serial3.available() == 0)
  {
    updateData();
  }

//五秒上传一次数据
  if ((millis() - timeStart) > 5000)
  { 
    getData();
    Upload();
    timeStart = millis();
  }
}

\end{lstlisting}




\newpage

\section{前端小程序}
前端小程序页面排布如下图所示,主要分为主页面,存件页面和取件页面。
取件页面可以查看当前外卖柜的温度和湿度,以及设置温度和湿度,操控风扇和加热器,打开和关闭灯光,解锁和上锁外卖柜等外设操作。

移动应用基于阿里云Iot Studio开发,通过阿里云Iot Studio提供的API接口实现与硬件设备的通信,实现了远程监控和控制功能。用户可以通过手机小程序随时随地监测外卖柜的温度和湿度情况,并根据需要进行调整,以确保外卖的质量和安全。同时,用户还可以通过手机小程序远程控制外卖柜的风扇、加热器、灯光等外设。
\begin{figure}[H]
  \centering
  \begin{minipage}{.3\textwidth}
    \centering
    \includegraphics[width=.85\linewidth]{figures/fig/Snipaste_2024-04-28_20-26-15.png}
    \caption{\small 主页面}
    \label{fig:image1}
  \end{minipage}%
  \begin{minipage}{.3\textwidth}
    \centering
    \includegraphics[width=0.8\linewidth]{figures/fig/Snipaste_2024-04-28_20-26-50.png}
    \caption{\small 存件页面}
    \label{fig:image2}
  \end{minipage}
  \begin{minipage}{.3\textwidth}
    \centering
    \includegraphics[width=0.8\linewidth]{figures/fig/Snipaste_2024-04-28_20-31-18.png}
    \caption{\small 取件页面}
    \label{fig:image3}
  \end{minipage}
\end{figure}

\newpage

\section{阿里云后端}

本项目的后端基于阿里云平台,
\begin{itemize}
  \item 阿里云设备和产品
  \begin{figure}[H]
        \centering
        \includegraphics[width =.8\textwidth]{figures/fig/Snipaste_2024-04-28_20-51-16.png}
        \caption{阿里云设备和产品}
        \label{fig:enter-label}
  \end{figure}
  \item 产品物模型参数
    \begin{figure}[H]
        \centering
        \includegraphics[width =.8\textwidth]{figures/fig/Snipaste_2024-04-28_20-51-16.png}
        \caption{产品物模型参数}
        \label{fig:enter-label}
  \end{figure}
  \item 移动应用开发
    \begin{figure}[H]
        \centering
        \includegraphics[width =.8\textwidth]{figures/fig/Snipaste_2024-04-28_20-55-26.png}
        \caption{基于阿里云Iot Studio开发的移动应用}
        \label{}
  \end{figure}
\end{itemize}
\newpage

\section{项目制作过程和结果展示}
  \subsection*{制作过程}
\begin{figure}[htbp]
  \centering
  \begin{minipage}{0.46\textwidth}
    \includegraphics[width=\textwidth]{./figures/fig/微信图片_20240428230617.jpg}
    \caption{硬件装配1}
  \end{minipage}
  \begin{minipage}{0.46\textwidth}
    \includegraphics[width=0.73\textwidth,angle=90]{./figures/fig/微信图片_20240428230707.jpg}
    \caption{硬件装配2}
  \end{minipage}
  \begin{minipage}{0.46\textwidth}
    \includegraphics[width=\textwidth]{./figures/fig/微信图片_20240428230711.jpg}
    \caption{硬件装配3}
  \end{minipage}
  \begin{minipage}{0.46\textwidth}
    \includegraphics[width=\textwidth]{./figures/fig/微信图片_20240428230715.jpg}
    \caption{硬件装配4}
  \end{minipage}
  \centering
  \begin{minipage}{0.46\textwidth}
    \includegraphics[width=\textwidth]{./figures/fig/微信图片_20240428230603.png}
    \caption{软件开发}
  \end{minipage}
  \begin{minipage}{0.46\textwidth}
    \includegraphics[width=\textwidth]{./figures/fig/微信图片_20240428230549.png}
    \caption{git记录}
  \end{minipage}
\end{figure}
\newpage

  \subsection*{成果展示}
  部分成果展示,详细过程参见视频

\begin{figure}[htbp]
  \centering
  \begin{minipage}{0.48\textwidth}
    \includegraphics[width=\textwidth]{./figures/fig/微信图片_20240428234925.png}
    \caption{成果展示}
  \end{minipage}
  \begin{minipage}{0.48\textwidth}
    \includegraphics[width=\textwidth]{./figures/fig/微信图片_20240428234937.png}
    \caption{成果展示}
  \end{minipage}
  \begin{minipage}{0.48\textwidth}
    \includegraphics[width=\textwidth]{./figures/fig/微信图片_20240428234940.png}
    \caption{成果展示}
  \end{minipage}
  \begin{minipage}{0.46\textwidth}
    \includegraphics[width=\textwidth]{./figures/fig/微信图片_20240428235042.png}
    \caption{成果展示}
  \end{minipage}

\end{figure}




\newpage


\appendix
\section{附录}
项目相关资料已开源至GitHub仓库:
\url{https://github.com/cyc-987/lotFinal_arduinoCode}


演示视频见提交附件,也已上传到B站:
\url{https://www.bilibili.com/video/BV1nb421a7qJ/?vd_source=dbfe1e12112250bef05c381df34391a0}

%%----------- 参考文献 -------------------%%
%在reference.bib文件中填写参考文献,此处自动生成




\end{document}