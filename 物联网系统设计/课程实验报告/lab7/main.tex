%!TeX program = xelatex
\documentclass[12pt,hyperref,a4paper,UTF8]{ctexart}
\usepackage{zjureport}
\usepackage{listings}
\usepackage{enumitem}
\usepackage{float}
\usepackage{xcolor}
\lstset{
    %backgroundcolor=\color{red!50!green!50!blue!50},%代码块背景色为浅灰色
    rulesepcolor= \color{gray}, %代码块边框颜色
    breaklines=true,  %代码过长则换行
    numbers=left, %行号在左侧显示
    numberstyle= \small,%行号字体
    keywordstyle= \color{blue},%关键字颜色
    commentstyle=\color{gray}, %注释颜色
    frame=shadowbox%用方框框住代码块
    }


%%-------------------------------正文开始---------------------------%%
\begin{document}

%%-----------------------封面--------------------%%
\cover

%%------------------摘要-------------%%
%\begin{abstract}
%
%在此填写摘要内容
%
%\end{abstract}

\thispagestyle{empty} % 首页不显示页码

%%--------------------------目录页------------------------%%
\newpage
\tableofcontents

%%------------------------正文页从这里开始-------------------%
\newpage

%%可选择这里也放一个标题
%\begin{center}
%    \title{ \Huge \textbf{{标题}}}
%\end{center}

\section{实验介绍}
\subsection{实验要求}

\begin{enumerate}
  \item 在IoT Studio中 建设一个网页,要求能控制你的灯的三色变化。
  \item 上交的报告中应包括界面截图,回传的数据、发送的控制命令时戳和相应的调光结果。
  \item 使用钉钉机器人实时接收灯光数据。
\end{enumerate}
\subsection{实验器材}
\begin{itemize}[itemsep=-5pt, topsep=0pt, partopsep=0pt]
    \item Arduino mega 开发板及载板
    \item 阿里云平台
    \item 阿里云Iot Studio
\end{itemize}


\section{实验过程}

\noindent 首先在阿里云Iot Studio中创建一个新的项目,关联产品和设备,详细操作同lab7。

\subsection{web控制三色灯业务逻辑开发}
\begin{itemize}[itemsep=-5pt, topsep=0pt, partopsep=0pt]
    \item 从web控制智能灯模板创建一个业务逻辑,命名为“web控制三色灯”
    \begin{figure}[H]
      \centering
      \includegraphics[width =1.0\textwidth]{figures/fig/Snipaste_2024-04-05_23-22-14.png}
      \caption{创建web控制智能灯业务}
      \label{fig:enter-label}
    \end{figure}
    \item 如图设置HTTP请求中的入参配置
    \begin{figure}[H]
      \centering
      \includegraphics[width =1.0\textwidth]{figures/fig/Snipaste_2024-04-05_23-23-41.png}
      \caption{设置HTTP请求节点}
      \label{fig:enter-label}
    \end{figure}
    
    \item 设置设备下发节点,下发属性来源为HTTP请求节点
    \begin{figure}[H]
      \centering
      \includegraphics[width =1.0\textwidth]{figures/fig/Snipaste_2024-04-05_23-24-12.png}
      \caption{设置设备下发节点}
      \label{fig:enter-label}
    \end{figure}
    \item 设置HTTP返回节点
    \begin{figure}[H]
      \centering
      \includegraphics[width =1.0\textwidth]{figures/fig/Snipaste_2024-04-05_23-24-36.png}
      \caption{置HTTP返回节点}
      \label{fig:enter-label}
    \end{figure}
    \item 设置钉钉机器人节点
    \begin{figure}[H]
      \centering
      \includegraphics[width =1.0\textwidth]{figures/fig/Snipaste_2024-04-05_23-24-48.png}
      \caption{设置钉钉机器人节点}
      \label{fig:enter-label}
    \end{figure}
    \item 钉钉机器人节点参数如下
    \begin{lstlisting}[language=C]
//query代表HTTP请求中的入参
{
  "msgtype": "text", 
  "text": {
    "content": "web控制三色灯,三色灯设备属性:ColorRed:{{query.ColorRed}}   ColorBlue:{{query.ColorBlue}}   ColorGreen:{{query.ColorGreen}}   LightSwitch:{{query.Switch}}   Frequency:{{query.Frequency}}"
 }, 
      "isAtAll": false
}
    \end{lstlisting}
    \item 调试无误后发布业务逻辑
\end{itemize}
\subsection{硬件设备代码}
本次实验硬件代码基本同lab6中代码。详情见之前报告代码,不再赘述。

\subsection{web控制三色灯前端网页开发}

\begin{itemize}
  \item 如图新建一个web应用。
  \begin{figure}[H]
    \centering
    \includegraphics[width =1.0\textwidth]{figures/fig/Snipaste_2024-04-05_23-30-54.png}
    \caption{新建web应用}
    \label{fig:enter-label}
  \end{figure}
  \item 设置页面导航布局和基础样式等配置。
  \begin{figure}[H]
    \centering
    \includegraphics[width =1.0\textwidth]{figures/fig/Snipaste_2024-04-05_23-31-59.png}
    \caption{设置页面布局和基础样式}
    \label{fig:enter-label}
  \end{figure}
  \item 新建一个页面,设置页面的布局和样式。
  \begin{figure}[H]
    \centering
    \includegraphics[width =1.0\textwidth]{figures/fig/Snipaste_2024-04-05_23-35-16.png}
    \caption{设置页面的布局和样式}
    \label{fig:enter-label}
  \end{figure}
  \item 在组件库中选择卡片,托条,按钮等组件,进行页面布局。
  \begin{figure}[H]
    \centering
    \includegraphics[width =1.0\textwidth]{figures/fig/Snipaste_2024-04-05_23-36-06.png}
    \caption{设置钉钉机器人节点}
    \label{fig:enter-label}
  \end{figure}
  \item 给卡片组件设置好样式参数以及展示数据来源,数据来源来自于设备属性。
  \item 同理给实数曲线图等其它组件设置好样式和展示数据来源,数据来源来自于设备属性。
  \item 给滑条和按钮等组件设置好样式等参数
  \begin{figure}[H]
    \centering
    \begin{minipage}{.22\textwidth}
      \centering
      \includegraphics[width=.55\linewidth]{figures/fig/Snipaste_2024-04-05_23-37-23.png}
      \caption{\footnotesize 卡片组件设置}
      \label{fig:image1}
    \end{minipage}%
    \begin{minipage}{.22\textwidth}
      \centering
      \includegraphics[width=.7\linewidth]{figures/fig/Snipaste_2024-04-05_23-37-47.png}
      \caption{\footnotesize 数据来源}
      \label{fig:image2}
    \end{minipage}
    \begin{minipage}{.22\textwidth}
      \centering
      \includegraphics[width=.55\linewidth]{figures/fig/Snipaste_2024-04-05_23-39-28.png}
      \caption{\footnotesize 滑条组件设置}
      \label{fig:image2}
    \end{minipage}
    \begin{minipage}{.22\textwidth}
      \centering
      \includegraphics[width=.55\linewidth]{figures/fig/Snipaste_2024-04-05_23-39-56.png}
      \caption{\footnotesize 按钮组件设置}
      \label{fig:image2}
    \end{minipage}
  \end{figure}


  \item 给滑条组件设置交互,将滑条的值传递给页面变量。页面变量的设置如下图所示。
  \begin{figure}[H]
    \centering
    \begin{minipage}{.3\textwidth}
      \centering
      \includegraphics[width=.85\linewidth]{figures/fig/Snipaste_2024-04-05_23-55-22.png}
      \caption{\small 交互动作设置}
      \label{fig:image1}
    \end{minipage}%
    \begin{minipage}{.6\textwidth}
      \centering
      \includegraphics[width=1\linewidth]{figures/fig/Snipaste_2024-04-05_23-55-30.png}
      \caption{\small 页面变量}
      \label{fig:image2}
    \end{minipage}
  \end{figure}

  \item 同理给按键组件设置交互,调用上文所创建的业务逻辑,将页面变量的值传递给业务逻辑。
  \begin{figure}[H]
    \centering
    \begin{minipage}{.4\textwidth}
      \centering
      \includegraphics[width=.85\linewidth]{figures/fig/Snipaste_2024-04-05_23-56-00.png}
      \caption{\small 交互动作设置}
      \label{fig:image1}
    \end{minipage}%
    \begin{minipage}{.4\textwidth}
      \centering
      \includegraphics[width=0.7\linewidth]{figures/fig/Snipaste_2024-04-05_23-56-24.png}
      \caption{\small 调用服务}
      \label{fig:image2}
    \end{minipage}
  \end{figure}
  \item 前端搭建完成后,点击预览按钮,查看效果。
  \begin{figure}[H]
    \centering
    \includegraphics[width =0.85\textwidth]{figures/fig/Snipaste_2024-04-06_00-06-37.png}
    \caption{预览效果}
    \label{fig:enter-label}
  \end{figure}

\end{itemize}


\section{实验结果}
\subsection*{三色灯控制}
\begin{itemize}
  \item 连接上硬件设备,此时灯初始不亮。进入web页面拖到滑条设置颜色,点击按钮后,灯亮起。同时页面上的卡片和实时曲线图也会有相应的变化。
  \begin{figure}[H]
    \centering
    \includegraphics[width =0.85\textwidth]{figures/fig/Snipaste_2024-04-06_00-12-03.png}
    \caption{设置开关和颜色,显示设备状态}
    \label{fig:enter-label}
  \end{figure}
  \begin{figure}[H]
    \centering
    \includegraphics[width =0.85\textwidth]{figures/fig/7d0254665632c7af28b1e6dbd9722a3.jpg}
    \caption{硬件设备灯亮起}
    \label{fig:enter-label}
  \end{figure}
  \item 在设备日志中可以看到,调用API,设置设备属性等控制命令时戳信息。同时钉钉群也会收到消息。
  \begin{figure}[H]
    \centering
    \includegraphics[width =0.85\textwidth]{figures/fig/Snipaste_2024-04-06_00-21-44.png}
    \caption{设备日志}
    \label{fig:enter-label}
  \end{figure}
  \begin{figure}[H]
    \centering
    \includegraphics[width =0.85\textwidth]{figures/fig/Snipaste_2024-04-06_00-13-25.png}
    \caption{钉钉群通知}
    \label{fig:enter-label}
  \end{figure}
\end{itemize}





%%----------- 参考文献 -------------------%%
%在reference.bib文件中填写参考文献,此处自动生成




\end{document}