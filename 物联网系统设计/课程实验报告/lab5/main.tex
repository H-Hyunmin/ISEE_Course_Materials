%!TeX program = xelatex
\documentclass[12pt,hyperref,a4paper,UTF8]{ctexart}
\usepackage{zjureport}
\usepackage{listings}
\usepackage{enumitem}
\usepackage{float}
\usepackage{xcolor}
\lstset{
    %backgroundcolor=\color{red!50!green!50!blue!50},%代码块背景色为浅灰色
    rulesepcolor= \color{gray}, %代码块边框颜色
    breaklines=true,  %代码过长则换行
    numbers=left, %行号在左侧显示
    lineskip={-1.5pt},
    numberstyle= \small,%行号字体
    keywordstyle= \color{blue},%关键字颜色
    commentstyle=\color{gray}, %注释颜色
    frame=shadowbox%用方框框住代码块
    }


%%-------------------------------正文开始---------------------------%%
\begin{document}

%%-----------------------封面--------------------%%
\cover

%%------------------摘要-------------%%
%\begin{abstract}
%
%在此填写摘要内容
%
%\end{abstract}

\thispagestyle{empty} % 首页不显示页码

%%--------------------------目录页------------------------%%
\newpage
\tableofcontents

%%------------------------正文页从这里开始-------------------%
\newpage

%%可选择这里也放一个标题
%\begin{center}
%    \title{ \Huge \textbf{{标题}}}
%\end{center}

\section{实验介绍}
\subsection{实验要求}

\begin{enumerate}[itemsep=-5pt, topsep=0pt, partopsep=0pt]
    \item 在设备端直接采样温度的数据,将数据发送云端,十秒更新一次。(周期性上报温度数据)
    \item 设置温度报警并推送钉钉群。当温度>25℃时,自动推送报警信息到钉钉群。
    \item 在事件响应中创建定时开关灯的场景,实现在指定时间开启LED,在特定时间关闭LED。
\end{enumerate}
\subsection{实验平台}
\begin{itemize}[itemsep=-5pt, topsep=0pt, partopsep=0pt]
    \item 阿里云物联网平台
    \item Arduino硬件平台
\end{itemize}


\section{任务一:采样温度数据,发送云端}
\subsection{实验过程}
\subsection*{实验代码}
    
\begin{lstlisting}[language=C++]
    //自定义的宏
    #define My_JSON_PACK_1        "{\"id\":\"66666\",\"version\":\"1.0\",\"method\":\"thing.event.property.post\",\"params\":{\"Temperature\":%d.%02d}}\r"

    //loop中的上传温度代码
    if (bmp.takeForcedMeasurement()) {
        // can now print out the new measurements
        temperature=bmp.readTemperature();
        Serial.print(F("Temperature = "));
        Serial.print(bmp.readTemperature());
        Serial.println(" *C");
      } 
      else {
        Serial.println("Forced measurement failed!");
      }
      if(Update()) Serial.println("Update success");
\end{lstlisting}

温度上传函数如下:
    \begin{lstlisting}[language=C++]
bool Update()
    {
    bool flag;
    int  frac;
    int zs;
    int len;
    cleanBuffer(ATcmd,BUF_LEN);
    snprintf(ATcmd,BUF_LEN,AT_MQTT_PUB_SET,ProductKey,DeviceName);
    flag = check_send_cmd(ATcmd,AT_OK,DEFAULT_TIMEOUT);
    if(!flag) {
        Serial.println("send AT_MQTT_PUB_SET false");
        return false;
    }

    cleanBuffer(ATdata,BUF_LEN_DATA);
    zs=(int) temperature;
    frac=(temperature-zs)*100;
    len = snprintf(ATdata,BUF_LEN_DATA,My_JSON_PACK_1,zs,frac);

    cleanBuffer(ATcmd,BUF_LEN);
    snprintf(ATcmd,BUF_LEN,AT_MQTT_PUB_DATA,len-1);
    flag=0;
    flag = check_send_cmd(ATcmd,">",DEFAULT_TIMEOUT);
    if(flag) {
        flag = check_send_cmd(ATdata,AT_MQTT_PUB_DATA_SUCC,20);
        
    } 
    else {
        Serial.println("> accept wrong");
        return false;
    }
        return true;
}
    \end{lstlisting}



\begin{itemize}[itemsep=-5pt, topsep=0pt, partopsep=0pt]
    \item 连接硬件,烧写程序,打开串口监视器
    \item 在串口监视器中可以看到设备上传的温度数据
    \item 打开阿里云平台,可以看到设备已经成功连接
    \item 在监控运维中可以看到设备上传的温度数据
\end{itemize}




\subsection{实验结果}
\begin{figure}[H]
    \centering
    \includegraphics[width =.9\textwidth]{figures/fig/Snipaste_2024-03-22_15-46-55.png}
    \caption{串口显示}
    \label{fig:enter-label}
\end{figure}
\begin{figure}[H]
    \centering
    \includegraphics[width =.9\textwidth]{figures/fig/Snipaste_2024-03-22_15-46-17.png}
    \caption{阿里云接受温度数据}
    \label{fig:enter-label}
\end{figure}




\section{任务二:设置温度报警并推送钉钉群}

\subsection{实验过程}
\begin{itemize}[]
    \item 在钉钉群中设置机器人
    \begin{figure}[H]
        \centering
        \begin{minipage}{0.45\textwidth}
            \centering
            \includegraphics[width=\textwidth]{figures/fig/Snipaste_2024-03-21_21-30-21.png}
            \caption{设置钉钉机器人}
            \label{fig:enter-label1}
        \end{minipage}\hfill
        \begin{minipage}{0.45\textwidth}
            \centering
            \includegraphics[width=\textwidth]{figures/fig/Snipaste_2024-03-21_21-30-37.png}
            \caption{设置钉钉机器人}
            \label{fig:enter-label2}
        \end{minipage}
    \end{figure}

    \item 在阿里云上-事件响应-联系人管理中-创建事件联系人
    \item 把钉钉机器人的webhook地址和密钥填入
    \item 在事件响应,场景管理中创建温度报警场景
    \item 设置好场景定义和场景动作,立即启用场景
    \begin{figure}[H]
        \centering
        \includegraphics[width =.5\textwidth]{figures/fig/Snipaste_2024-03-21_21-32-41.png}
        \caption{阿里云创建联系人}
        \label{fig:enter-label}
    \end{figure} 
    \begin{figure}[H]
        \centering
        \includegraphics[width =1.0\textwidth]{figures/fig/Snipaste_2024-03-21_21-35-04.png}
        \caption{温度报警场景创建}
        \label{fig:enter-label}
    \end{figure} 
    \item 加热传感器到25摄氏度以上,可以在钉钉中看到消息推送
\end{itemize}

\subsection{实验结果}
\begin{figure}[H]
    \centering
    \includegraphics[width =1.0\textwidth]{figures/fig/Snipaste_2024-03-22_15-58-37.png}
    \caption{温度报警日志}
    \label{fig:enter-label}
\end{figure} 
\begin{figure}[H]
    \centering
    \includegraphics[width =1.0\textwidth]{figures/fig/Snipaste_2024-03-22_15-58-58.png}
    \caption{钉钉收到温度报警}
    \label{fig:enter-label}
\end{figure} 





\section{任务三:创建定时开关灯的场景,实现在指定时间开启 LED}
\subsection{实验过程}

\subsection*{实验代码}
    
\begin{lstlisting}[language=C++]
    //定时开启闪烁灯或关闭
    unsigned long starmil=millis();
    unsigned long currmil;
    do{
        if (Serial3.available() > 0) {
            handleSerialData();//串口数据处理函数
          }
        if (IsFlickering){//闪烁灯实现
          delay(delay_seconds/2);
          analogWrite(9, ColorBlue);
          analogWrite(7, ColorGreen);
          analogWrite(8, ColorRed);

          delay(delay_seconds/2);
          analogWrite(9, 0);
          analogWrite(7, 0);
          analogWrite(8, 0);   
      }
      currmil=millis();
    }while (currmil-starmil<10000);//10s计时结束后要发送一次温度数据
  
\end{lstlisting}


\begin{itemize}[itemsep=-5pt, topsep=0pt, partopsep=0pt]
    
    \item 在事件响应中创建定时开关灯的场景,设置好参数,开启事件。
    \begin{figure}[H]
        \centering
        \includegraphics[width =.95\textwidth]{figures/fig/Snipaste_2024-03-21_21-37-41.png}
        \caption{设置开关灯场景}
        \label{fig:enter-label}
    \end{figure}
    \item 同理设置好定时关灯场景,启用场景
\end{itemize}

\subsection{实验结果}
可以看到定时开灯和定时关灯场景都成功执行,设备上的灯定时开启闪烁,定时关闭闪烁。
\begin{figure}[H]
    \centering
    \includegraphics[width =.95\textwidth]{figures/fig/Snipaste_2024-03-22_16-11-24.png}
    \caption{定时开灯}
    \label{fig:enter-label}
\end{figure}
\begin{figure}[H]
    \centering
    \includegraphics[width =.95\textwidth]{figures/fig/Snipaste_2024-03-22_16-13-09.png}
    \caption{定时关灯}
    \label{fig:enter-label}
\end{figure}
\newpage
\subsection*{附录}
实验关键代码如下:
\begin{lstlisting}[language=C++]
//setup部分代码
    void setup() {
        //Serial Initial
        Serial3.begin(115200);
        Serial.begin(115200);
        String inString="";
        pinMode(7,OUTPUT);  //RGB引脚
        pinMode(8,OUTPUT); 
        pinMode(9,OUTPUT); 
        pinMode(13,OUTPUT); //板载led
        
        //  BMP module test
        Serial.println(F("BMP280 Forced Mode Test."));
        if (!bmp.begin(0x76,0x58)) {
          Serial.println(F("Could not find a valid BMP280 sensor, check wiring or "
                            "try a different address!"));
          while (1) delay(10);
        }
        bmp.setSampling(Adafruit_BMP280::MODE_FORCED,     /* Operating Mode. */
                        Adafruit_BMP280::SAMPLING_X2,     /* Temp. oversampling */
                        Adafruit_BMP280::SAMPLING_X16,    /* Pressure oversampling */
                        Adafruit_BMP280::FILTER_X16,      /* Filtering. */
                        Adafruit_BMP280::STANDBY_MS_500); /* Standby time. */
      
        
        //  Pin_init();
        BEEP(1);//蜂鸣器响一声
        
        //wifi连接和阿里云连接
        while(1)
        {
          if(!WiFi_init())continue;
          BEEP(2);
          Serial.println("WiFi init success");
          if(!Ali_connect())continue;
          break;
        }
        BEEP(3);
        Serial.println("Ali connect success");
      }
      
//————————————————————————————————————————————————
void loop() {
    delay(10);
    while (1)
    {
    //温度传感器上传
      if (bmp.takeForcedMeasurement()) {
        // can now print out the new measurements
        temperature=bmp.readTemperature();
        Serial.print(F("Temperature = "));
        Serial.print(bmp.readTemperature());
        Serial.println(" *C");
      }
      else {
        Serial.println("Forced measurement failed!");
      }
      if(Update()) Serial.println("Update success");
      Serial.println("============================");
      while(Serial3.available()){ //清楚串口缓冲区
        Serial3.read();
      }
//闪烁灯
      unsigned long starmil=millis();
      unsigned long currmil;
      do{
          if (Serial3.available() > 0) {
              // 读取并处理串口数据
              handleSerialData();
            }
          if (IsFlickering){//闪烁灯实现
            delay(delay_seconds/2);
            analogWrite(9, ColorBlue);
            analogWrite(7, ColorGreen);
            analogWrite(8, ColorRed);

            delay(delay_seconds/2);
            analogWrite(9, 0);
            analogWrite(7, 0);
            analogWrite(8, 0);   
        }
        currmil=millis();
      }while (currmil-starmil<10000);
    }
  }
//————————————————————————————————————————————————
//upload函数
bool Update()
{
  bool flag;
  int  frac;
  int zs;
  int len;
  cleanBuffer(ATcmd,BUF_LEN);
  snprintf(ATcmd,BUF_LEN,AT_MQTT_PUB_SET,ProductKey,DeviceName);
  flag = check_send_cmd(ATcmd,AT_OK,DEFAULT_TIMEOUT);
  if(!flag) {
    Serial.println("send AT_MQTT_PUB_SET false");
    return false;
  }

  cleanBuffer(ATdata,BUF_LEN_DATA);
  zs=(int) temperature;
  frac=(temperature-zs)*100;
  len = snprintf(ATdata,BUF_LEN_DATA,My_JSON_PACK_1,zs,frac);

  cleanBuffer(ATcmd,BUF_LEN);
  snprintf(ATcmd,BUF_LEN,AT_MQTT_PUB_DATA,len-1);
  flag=0;
  flag = check_send_cmd(ATcmd,">",DEFAULT_TIMEOUT);
  if(flag) {
    flag = check_send_cmd(ATdata,AT_MQTT_PUB_DATA_SUCC,20);
    
  } 
  else {
    Serial.println("> accept wrong");
    return false;
  }
    return true;
}
//————————————————————————————————————————————————
//串口数据处理函数
void handleSerialData() {
  delay(10);
  inString = Serial3.readString();  // 读取串口数据
  if (inString != "") {
    // 在这里处理接收到的串口数据
    data=inString;
    Serial.println(inString);
  }
    frequency=parse(data);   //解析数据并打印
  if (frequency==0) { frequency=1;}
  Serial.print("Frequency:");
  Serial.println(frequency);  
  delay_seconds=1.0/frequency*1000; //单位为毫秒
  Serial.print("delay_seconds(ms):");
  Serial.println(delay_seconds);
  if(IsBreathing && IsFlickering){//防止两个同时为真
      IsBreathing==0;
      Serial.println("IsBreathing and IsFlickering can't be true at the same time,set IsBreathing to false");
  }
}

\end{lstlisting}

%%----------- 参考文献 -------------------%%
%在reference.bib文件中填写参考文献,此处自动生成




\end{document}