%!TeX program = xelatex
\documentclass[12pt,hyperref,a4paper,UTF8]{ctexart}
\usepackage{zjureport}
\usepackage{listings}
\usepackage{enumitem}
\usepackage{float}
\usepackage{xcolor}
\lstset{
    %backgroundcolor=\color{red!50!green!50!blue!50},%代码块背景色为浅灰色
    rulesepcolor= \color{gray}, %代码块边框颜色
    breaklines=true,  %代码过长则换行
    numbers=left, %行号在左侧显示
    numberstyle= \small,%行号字体
    keywordstyle= \color{blue},%关键字颜色
    commentstyle=\color{gray}, %注释颜色
    frame=shadowbox%用方框框住代码块
    }


%%-------------------------------正文开始---------------------------%%
\begin{document}

%%-----------------------封面--------------------%%
\cover

%%------------------摘要-------------%%
%\begin{abstract}
%
%在此填写摘要内容
%
%\end{abstract}

\thispagestyle{empty} % 首页不显示页码

%%--------------------------目录页------------------------%%
\newpage
\tableofcontents

%%------------------------正文页从这里开始-------------------%
\newpage

%%可选择这里也放一个标题
%\begin{center}
%    \title{ \Huge \textbf{{标题}}}
%\end{center}

\section{实验介绍}
\subsection{实验要求}

\begin{enumerate}[itemsep=-5pt, topsep=0pt, partopsep=0pt]
    \item 在阿里云端注册,创建产品和设备,直接导入物模型model.json 
    \item 计算MQTT和阿里云链接的参数,连接Aliyun。
    \item 通过sub订阅消息,通过pub发布消息。 
    \item 利用Alink Json 格式改变温度、闪烁数据。(注意:pub的主题不同,状态的改变)
    \item 利用阿里云的自定义topic,发送自定义文本数据,MQTT.fx接收。
\end{enumerate}
\subsection{实验平台}
\begin{itemize}[itemsep=-5pt, topsep=0pt, partopsep=0pt]
    \item 阿里云物联网平台
    \item MQTTFX软件
    \item 阿里云平台配置软件
\end{itemize}


\section{任务一:阿里云端注册,创建产品和设备,导入物模型}
\subsection{实验过程}

\begin{itemize}[itemsep=-5pt, topsep=0pt, partopsep=0pt]
    \item 打开阿里云平台,注册账号并登入,开通物联网平台。(注册等具体操作过程略)
    \begin{figure}[H]
        \centering
        \includegraphics[width =.85\textwidth]{figures/fig/Snipaste_2024-03-07_16-28-24.png}
        \caption{开通账号,进入物联网平台}
        \label{fig:enter-label}
    \end{figure}    
    
    \item 进入公共实例->设备管理->产品页面
    \begin{figure}[H]
        \centering
        \includegraphics[width =.85\textwidth]{figures/fig/Snipaste_2024-03-07_16-37-54.png}
        \caption{进入产品页面}
        \label{fig:enter-label}
    \end{figure}  

    \item 选择创建产品,输入产品名称,自定义品类,WIFI连网方式,JSON数据格式
    \begin{figure}[H]
        \centering
        \includegraphics[width =.85\textwidth]{figures/fig/Snipaste_2024-03-07_16-38-54.png}
        \caption{创建产品}
        \label{fig:enter-label}
    \end{figure} 

    \item 进入设备管理->设备页面,选择添加设备,产品选择刚刚创建的产品的MQTTlab,输入设备名称
    \begin{figure}[H]
        \centering
        \includegraphics[width =.85\textwidth]{figures/fig/Snipaste_2024-03-07_16-43-12.png}
        \caption{创建设备}
        \label{fig:enter-label}
    \end{figure} 

    \item 回到原来的产品页面,找到刚刚创建的产品MQTTlab,点击查看,在功能定义中选择前往编辑草稿
    \begin{figure}[H]
        \centering
        \includegraphics[width =.85\textwidth]{figures/fig/Snipaste_2024-03-07_16-56-27.png}
        \caption{在产品功能定义中编辑物模型}
        \label{fig:enter-label}
    \end{figure} 

    \item 在编辑草稿页面,选择快速导入->导入物模型->点击上传,在打开的文件选择框中选择model.json文件,之后点击创建物模型
    \item \bf{注意:创建后记得选择发布上线,发布物模型,否则无法在设备列表中看到设备的物模型}
    \begin{figure}[H]
        \centering
        \includegraphics[width =.85\textwidth]{figures/fig/Snipaste_2024-03-07_17-01-20.png}
        \caption{导入物模型}
        \label{fig:enter-label}
    \end{figure} 
\end{itemize}




\subsection{实验结果}
\begin{itemize}
    \item 在产品列表中可以看到刚刚创建的产品MQTTlab
    \item 在设备列表中可以看到刚刚创建的设备
    \item 在产品功能定义中可以看到刚刚导入的物模型
    \begin{figure}[H]
        \centering
        \includegraphics[width =.85\textwidth]{figures/fig/Snipaste_2024-03-07_17-05-58.png}
        \caption{创建的物模型}
        \label{fig:enter-label}
    \end{figure} 
\end{itemize}




\section{任务二:计算 MQTT 和阿里云链接的参数,连接 Aliyun}

\subsection{实验过程}
\begin{itemize}[]
    \item 在产品页面中找到刚刚创建的产品MQTTlab,选择发布,发布产品
    \begin{figure}[H]
        \centering
        \includegraphics[width =.8\textwidth]{figures/fig/Snipaste_2024-03-07_17-25-44.png}
        \caption{发布产品}
        \label{fig:enter-label}
    \end{figure}


    \item 在设备页面中找到创建的MQTTlabdevice1设备,点击查看,在设备信息中选择查看Deviceserver,一键复制设备的三元组信息
    \begin{figure}[H]
        \centering
        \includegraphics[width =.8\textwidth]{figures/fig/Snipaste_2024-03-07_17-12-21.png}
        \caption{获取三元组}
        \label{fig:enter-label}
    \end{figure} 

    \item 打开阿里云平台配置应用程序,粘贴三元组信息,点击生成
    \begin{figure}[H]
        \centering
        \includegraphics[width =.6\textwidth]{figures/fig/Snipaste_2024-03-07_17-15-40.png}
        \caption{生成信息}
        \label{fig:enter-label}
    \end{figure}

    \item 随意输入时间戳,建议clientid输入当前设备名称,再点击生成,得到所需的参数
    \begin{figure}[H]
        \centering
        \includegraphics[width =.6\textwidth]{figures/fig/Snipaste_2024-03-07_17-18-30.png}
        \caption{生成参数}
        \label{fig:enter-label}
    \end{figure}
    
    \
    \item 打开MQTT.fx软件,选择设置,在对应位置输入阿里云的参数,点击应用
    \begin{figure}[H]
        \centering
        \includegraphics[width =1\textwidth]{figures/fig/Snipaste_2024-03-07_17-30-49.png}
        \caption{生成参数}
        \label{fig:enter-label}
    \end{figure}

    \item 选择connect,可以看到绿灯代表成功连接上
    \begin{figure}[H]
        \centering
        \includegraphics[width =0.8\textwidth]{figures/fig/Snipaste_2024-03-07_17-34-06.png}
        \caption{连接云平台}
        \label{fig:enter-label}
    \end{figure}
\end{itemize}


\subsection{实验结果}
    可以在阿里云平台的设备监控中看到设备的连接状态,成功连接上阿里云
    \begin{figure}[H]
        \centering
        \includegraphics[width =.9\textwidth]{figures/fig/Snipaste_2024-03-07_21-31-47.png}
        \caption{设备状态显示在线,连接成功}
        \label{fig:enter-label}
    \end{figure}



\section{任务三:通过sub订阅消息,通过pub发布消息}
\subsection{sub订阅消息实验过程}


\begin{itemize}[itemsep=-5pt, topsep=0pt, partopsep=0pt]
    \item 查看MQTTlab产品,在Topic类列表,物模型通信Topic中可以看到属性设置功能
    \begin{figure}[H]
        \centering
        \includegraphics[width =.75\textwidth]{figures/fig/Snipaste_2024-03-07_21-22-58.png}
        \caption{查看物模型通信topic}
        \label{fig:enter-label}
    \end{figure}
    \item 将Topic类复制到MQTT软件的Subscribe框中,将\$\{deviceName\}换为设备名称MQTT\_lab\_device1
    \item 点击subscribe可以看到已经成功订阅
    \begin{figure}[H]
        \centering
        \includegraphics[width =.9\textwidth]{figures/fig/Snipaste_2024-03-07_21-33-49.png}
        \caption{订阅云平台}
        \label{fig:enter-label}
    \end{figure}

    \item 选择物联网云平台中的在线调试,选择产品和设备,设置一下参数,然后点击设置发送指令
    \subsection{实验结果}
    \begin{figure}[H]
        \centering
        \includegraphics[width =.9\textwidth]{figures/fig/Snipaste_2024-03-07_21-36-30.png}
        \caption{云平台发送消息}
        \label{fig:enter-label}
    \end{figure}

    \item 在MQTT软件中查看接受到的消息


\end{itemize}

\subsection{sub订阅消息实验结果}
可以在MQTT软件上看到订阅后收到的消息

\begin{figure}[H]
    \centering
    \includegraphics[width =.85\textwidth]{figures/fig/Snipaste_2024-03-07_21-39-04.png}
    \caption{接受到的订阅消息}
    \label{fig:enter-label}
\end{figure}


\subsection{pub发布消息实验过程}

\begin{itemize}
    \item 查看MQTTlab产品,在Topic类列表,物模型通信Topic中可以看到属性上报功能。
    \begin{figure}[H]
        \centering
        \includegraphics[width =.95\textwidth]{figures/fig/Snipaste_2024-03-07_21-43-58.png}
        \caption{查看要发布的topic类}
        \label{fig:enter-label}
    \end{figure}

    \item 选择操作类型为发布的topic类,将Topic类复制到MQTT软件的publication框中,将\$\{deviceName\}换为设备名称MQTT\_lab\_device1
    \item 根据JSON格式写上要发布的消息,点击publish发布
    \begin{figure}[H]
        \centering
        \includegraphics[width =.9\textwidth]{figures/fig/Snipaste_2024-03-07_22-14-29.png}
        \caption{MQTT软件端设置发布消息}
        \label{fig:enter-label}
    \end{figure}

\end{itemize}

\subsection{pub发布消息实验结果}
可以在云端的日志服务中看到运动接受本地发布的消息日志,在设备上也可以直接看到接受到的本地发布的数据。
\begin{figure}[H]
    \centering
    \includegraphics[width =1\textwidth]{figures/fig/Snipaste_2024-03-07_22-02-07.png}
    \caption{云端接受到发布的消息}
    \label{fig:enter-label}
\end{figure}


\section{任务四:利用Alink Json 格式改变物模型的参数}
\begin{itemize}
    \item 选择操作类型为发布的topic类,将Topic类复制到MQTT软件的publication框中,将\$\{deviceName\}换为设备名称MQTT\_lab\_device1
    \item 根据JSON格式写上要发布的消息,点击publish发布
\end{itemize}
\begin{figure}[H]
    \centering
    \includegraphics[width =1.0\textwidth]{figures/fig/Snipaste_2024-03-07_21-58-50.png}
    \caption{自定义topic}
    \label{fig:enter-label}
\end{figure}

\section{任务五:利用阿里云的自定义topic,发送自定义文本数据,MQTT.fx接收}
\begin{itemize}
    \item 在产品中设置自定义topic
    \item \bf{在MQTT的订阅}
    \begin{figure}[H]
        \centering
        \includegraphics[width =1.1\textwidth]{figures/fig/Snipaste_2024-03-07_22-19-02.png}
        \caption{自定义topic}
        \label{fig:enter-label}
    \end{figure}

    \item 在MQTT的订阅刚刚自定义的消息
    \begin{figure}[H]
        \centering
        \includegraphics[width =1.0\textwidth]{figures/fig/Snipaste_2024-03-07_22-25-53.png}
        \caption{在MQTT订阅}
        \label{fig:enter-label}
    \end{figure}
    
    \item 在云平台发布消息
    \begin{figure}[H]
        \centering
        \includegraphics[width =.9\textwidth]{figures/fig/Snipaste_2024-03-07_22-21-48.png}
        \caption{云平台发布自定义消息}
        \label{fig:enter-label}
    \end{figure}
    \item 在MQTT软件中查看接受到的消息
    \begin{figure}[H]
        \centering
        \includegraphics[width =.9\textwidth]{figures/fig/Snipaste_2024-03-07_22-26-46.png}
        \caption{在MQTT软件中查看接受到的自定义消息}
        \label{fig:enter-label}
    \end{figure}

\end{itemize}

%%----------- 参考文献 -------------------%%
%在reference.bib文件中填写参考文献,此处自动生成




\end{document}