%!TeX program = xelatex
\documentclass[12pt,hyperref,a4paper,UTF8]{ctexart}
\usepackage{zjureport}
\usepackage{listings}
\usepackage{enumitem}
\usepackage{float}
\usepackage{xcolor}
\lstset{
    %backgroundcolor=\color{red!50!green!50!blue!50},%代码块背景色为浅灰色
    rulesepcolor= \color{gray}, %代码块边框颜色
    breaklines=true,  %代码过长则换行
    numbers=left, %行号在左侧显示
    numberstyle= \small,%行号字体
    keywordstyle= \color{blue},%关键字颜色
    commentstyle=\color{gray}, %注释颜色
    frame=shadowbox%用方框框住代码块
    }


%%-------------------------------正文开始---------------------------%%
\begin{document}

%%-----------------------封面--------------------%%
\cover

%%------------------摘要-------------%%
%\begin{abstract}
%
%在此填写摘要内容
%
%\end{abstract}

\thispagestyle{empty} % 首页不显示页码

%%--------------------------目录页------------------------%%
\newpage
\tableofcontents

%%------------------------正文页从这里开始-------------------%
\newpage

%%可选择这里也放一个标题
%\begin{center}
%    \title{ \Huge \textbf{{标题}}}
%\end{center}

\section{实验介绍}
\subsection{实验要求}

\begin{enumerate}[itemsep=-5pt, topsep=0pt, partopsep=0pt]
    \item 在Arduino IDE的串口监视器中读取光敏电阻的读数
    \item 在Arduino IDE的串口监视器中读取BMP280传感器的温度信息
    \item 利用串口调试助手,向Arduino发送三色小灯的RGB参数(0~255)并解析,将三色小灯混合出不同的颜色
\end{enumerate}
\subsection{实验器材}
\begin{itemize}[itemsep=-5pt, topsep=0pt, partopsep=0pt]
    \item Arduino mega 开发板及载板
    \item BMP280传感器模块,三色灯模块,光敏电阻模块
    \item vscode搭配Arduino平台
\end{itemize}


\section{任务:光敏电阻的读数}
\subsection{实验过程}
\begin{itemize}[itemsep=-5pt, topsep=0pt, partopsep=0pt]
    \item 将模块正确连接到载板上,将开发板通过数据线连接到电脑
    \item 打开工程文件,选择串口和开发板
    \item 编译验证,上传代码
    \item 打开串口监视器,设置正确的波特率,查看结果
\end{itemize}

    \begin{figure}[H]
        \centering
        \includegraphics[width =.7\textwidth]{figures/fig/微信图片_20240304155932.jpg}
        \caption{硬件连接}
        \label{fig:enter-label}
    \end{figure}

    \begin{figure}
        \centering
        \includegraphics[width =.9\textwidth]{figures/fig/2.png}
        \caption{打开串口}
        \label{fig:enter-label}
    \end{figure}

    \begin{figure}
        \centering
        \includegraphics[width =.8\textwidth]{figures/fig/3.png}
        \caption{设置开发板}
        \label{fig:enter-label}
    \end{figure}

    \begin{figure}
        \centering
        \includegraphics[width =.8\textwidth]{figures/fig/4.png}
        \caption{编译上传代码}
        \label{fig:enter-label}
    \end{figure}
\subsection{实验结果}

    \begin{figure}[H]
        \centering
        \includegraphics[width =.8\textwidth]{figures/fig/5.png}
        \caption{通过串口监视器查看返回数值}
        \label{fig:enter-label}
    \end{figure}

    \begin{figure}[H]
        \centering
        \includegraphics[width =.8\textwidth]{figures/fig/6.png}
        \caption{用手电筒照射光敏电阻模块,可见返回值发生变化}
        \label{fig:enter-label}
    \end{figure}


\subsection{实验代码}

\begin{lstlisting}[language=C]
// the setup routine runs once when you press reset:
void setup() {
  // initialize serial communication at 9600 bits per second:
  Serial.begin(9600);
}

// the loop routine runs over and over again forever:
void loop() {
  // read the input on analog pin 0:
  int sensorValue = analogRead(A2);
  // print out the value you read:
  Serial.println(sensorValue);
  delay(1);  // delay in between reads for stability
}

\end{lstlisting}





\section{任务二:BMP280传感器信息读取}

\subsection{实验过程}
\begin{itemize}[]
    \item 打开工程文件,选择串口和开发板
    \item 下载源码所需的库文件
    \item 编译验证,上传代码
    \item 打开串口监视器,设置正确的波特率,查看结果
\end{itemize}

    \begin{figure}[H]
        \centering
        \includegraphics[width =1.1\textwidth]{figures/fig/7.png}
        \caption{用Arduino-cli下载所需的库文件}
        \label{fig:enter-label}
    \end{figure}


\newpage
\subsection{实验结果}
    \begin{figure}[H]
        \centering
        \includegraphics[width =.9\textwidth]{figures/fig/8.png}
        \caption{从串口监视器中可以看到BMP280传感器发送的信息}
        \label{fig:enter-label}
    \end{figure}
\subsection{实验代码}

\begin{lstlisting}[language=C]
#include <Adafruit_BMP280.h>

#define BMP_SCK  (13)
#define BMP_MISO (12)
#define BMP_MOSI (11)
#define BMP_CS   (10)

Adafruit_BMP280 bmp; // I2C
//Adafruit_BMP280 bmp(BMP_CS); // hardware SPI
//Adafruit_BMP280 bmp(BMP_CS, BMP_MOSI, BMP_MISO,  BMP_SCK);

void setup() {
  Serial.begin(9600);
  Serial.println(F("BMP280 Forced Mode Test."));
  //if (!bmp.begin(BMP280_ADDRESS_ALT, BMP280_CHIPID)) {
  if (!bmp.begin(0x76,0x58)) {
    Serial.println(F("Could not find a valid BMP280 sensor, check wiring or "
                      "try a different address!"));
    while (1) delay(10);
  }


  /* Default settings from datasheet. */
  bmp.setSampling(Adafruit_BMP280::MODE_FORCED,     /* Operating Mode. */
                  Adafruit_BMP280::SAMPLING_X2,     /* Temp. oversampling */
                  Adafruit_BMP280::SAMPLING_X16,    /* Pressure oversampling */
                  Adafruit_BMP280::FILTER_X16,      /* Filtering. */
                  Adafruit_BMP280::STANDBY_MS_500); /* Standby time. */
}

void loop() {
  // must call this to wake sensor up and get new measurement data
  // it blocks until measurement is complete
  if (bmp.takeForcedMeasurement()) {
    // can now print out the new measurements
    Serial.print(F("Temperature = "));
    Serial.print(bmp.readTemperature());
    Serial.println(" *C");

    Serial.print(F("Pressure = "));
    Serial.print(bmp.readPressure());
    Serial.println(" Pa");

    Serial.print(F("Approx altitude = "));
    Serial.print(bmp.readAltitude(1013.25)); /* Adjusted to local forecast! */
    Serial.println(" m");

    Serial.println();
    delay(2000);
  } else {
    Serial.println("Forced measurement failed!");
  }
}
\end{lstlisting}



\section{任务三:}
\subsection{实验过程}
\begin{itemize}[itemsep=-5pt, topsep=0pt, partopsep=0pt]
    \item 打开工程文件,选择串口和开发板
    \item 编译验证,上传代码
    \item 打开串口监视器,设置正确的波特率,发送RGB参数给三色小灯
    \item 观察小灯颜色变化
\end{itemize}

\subsection{实验结果}
    \begin{figure}[H]
        \centering
        \includegraphics[width =.9\textwidth]{figures/fig/9.png}
        \caption{给小灯发送不同的RGB参数}
        \label{fig:enter-label}
    \end{figure}


    \begin{figure}[H]
        \centering
        \includegraphics[width =.9\textwidth]{figures/fig/微信图片_20240304155933.jpg}
        \caption{小灯显示与RGB相对应的颜色}
        \label{fig:enter-label}
    \end{figure}
    \begin{figure}[H]
        \centering
        \includegraphics[width =.9\textwidth]{figures/fig/微信图片_20240304155934.jpg}
        \caption{小灯显示与RGB相对应的颜色}
        \label{fig:enter-label}
    \end{figure}
    \begin{figure}[H]
        \centering
        \includegraphics[width =.9\textwidth]{figures/fig/微信图片_20240304155935.jpg}
        \caption{小灯显示与RGB相对应的颜色}
        \label{fig:enter-label}
    \end{figure}



\subsection{实验代码}
\begin{lstlisting}[language={C}]
const int redPin = 8;
const int greenPin = 7;
const int bluePin = 9;

void setup() {
  // initialize serial:
  Serial.begin(115200);
  // make the pins outputs:
  pinMode(redPin, OUTPUT);
  pinMode(greenPin, OUTPUT);
  pinMode(bluePin, OUTPUT);
}

void loop() {
  // if there's any serial available, read it:
  while (Serial.available() > 0) {

    // look for the next valid integer in the incoming serial stream:
    int red = Serial.parseInt();
    // do it again:
    int green = Serial.parseInt();
    // do it again:
    int blue = Serial.parseInt();

    Serial.println(red);
    Serial.println(green);
    Serial.println(blue);
    
    analogWrite(redPin, red);
    analogWrite(greenPin, green);
    analogWrite(bluePin, blue);
    }
  }
\end{lstlisting}


%%----------- 参考文献 -------------------%%
%在reference.bib文件中填写参考文献,此处自动生成




\end{document}