%!TeX program = xelatex
\documentclass[12pt,hyperref,a4paper,UTF8]{ctexart}
\usepackage{zjureport}
\usepackage{listings}
\usepackage{enumitem}
\usepackage{float}
\usepackage{xcolor}
\usepackage{animate}
\lstset{
    %backgroundcolor=\color{red!50!green!50!blue!50},%代码块背景色为浅灰色
    rulesepcolor= \color{gray}, %代码块边框颜色
    breaklines=true,  %代码过长则换行
    basicstyle=\small\ttfamily,
    lineskip={-1.5pt},
    numbers=left, %行号在左侧显示
    numberstyle= \small,%行号字体
    keywordstyle= \color{blue},%关键字颜色
    commentstyle=\color{gray}, %注释颜色
    frame=shadowbox%用方框框住代码块
    }


    
%%-------------------------------正文开始---------------------------%%
\begin{document}

%%-----------------------封面--------------------%%
\cover

%%------------------摘要-------------%%
%\begin{abstract}
%
%在此填写摘要内容
%
%\end{abstract}

\thispagestyle{empty} % 首页不显示页码

%%--------------------------目录页------------------------%%
\newpage
\tableofcontents

%%------------------------正文页从这里开始-------------------%
\newpage

%%可选择这里也放一个标题
%\begin{center}
%    \title{ \Huge \textbf{{标题}}}
%\end{center}

\section{实验介绍}
\subsection{实验要求}

\begin{enumerate}[itemsep=-5pt, topsep=0pt, partopsep=0pt]
    \item 在阿里云平台端,给自己的产品加上三基色属性。
    \item 使用阿里云的在线调试功能,发送数据到设备端,能以PWM的形式实现调光。
    \item 控制三色灯呈现出不同的颜色。
    \item 控制三色灯实现闪烁、呼吸等其他效果?
\end{enumerate}
\subsection{实验平台}
\begin{itemize}[itemsep=-5pt, topsep=0pt, partopsep=0pt]
    \item 阿里云物联网平台
    \item Arduino硬件平台
\end{itemize}


\section{实验过程}
\subsection*{阿里云平台产品和设备设置}
    在阿里云平台新建一个空产品,设置自定义的属性参数。
    在阿里云平台新建一个设备,对应刚刚创建的产品。
    \begin{figure}[H]
        \centering
        \includegraphics[width =.8\textwidth]{figures/fig/Snipaste_2024-03-16_16-23-51.png}
        \caption{创建产品设置参数}
        \label{fig:enter-label}
    \end{figure}

    \begin{figure}[H]
        \centering
        \includegraphics[width =.75\textwidth]{figures/fig/Snipaste_2024-03-16_16-33-24.png}
        \caption{新建设备}
        \label{fig:enter-label}
    \end{figure}

\subsection*{连接参数生成}

    复制三元组,使用阿里云平台配置软件生成对应参数
    \begin{figure}[H]
        \centering
        \includegraphics[width =.6\textwidth]{figures/fig/Snipaste_2024-03-16_16-35-20.png}
        \caption{生成参数}
        \label{fig:enter-label}
    \end{figure}

\subsection*{程序参数设置}
    打开arduino程序,将生成的连接参数,手机热点信息等填写到对应的宏名处
\begin{lstlisting}[language=C++]
    //User Modified Part
    #define wifi_ssid     "xm"    
    #define wifi_psw      "12090217"     
    #define clientIDstr   "3color_light"
    #define timestamp     "6666"
    #define ProductKey    "k0v6olZyIU8"
    #define DeviceName    "3color_light"
    #define DeviceSecret  "ffcb212d922f0285f5cf08cc1495ed2b"
    #define password      "55203F6BA2AAEEAD5421602929B1C3F127A70323"
\end{lstlisting}

\subsection*{程序上传}
Arduino载板拨码开关设置成串口烧写模式,链接硬件,上传Arduino代码。手机打开热点


一次蜂鸣器表示初始化引脚,两次表示成功连上WiFi,三次表示连接上阿里云平台

\subsection*{阿里云平台参数下发}
进入云平台参数调试可以看到设备已经在线,设置好参数下发即可。

    \begin{figure}[H]
        \centering
        \includegraphics[width =.85\textwidth]{figures/fig/Snipaste_2024-03-16_16-47-39.png}
        \caption{下发参数}
        \label{fig:6}
    \end{figure}  

\section{实验结果}
\subsection{三色灯颜色调控}
    阿里云平台下发参数,打开串口监视器,可以看到串口输出了相关信息,表示接受到阿里云下发的数据。\\
    可以看到硬件上的三色灯成功被点亮。
    \begin{figure}[H]
        \centering
        \includegraphics[width =.85\textwidth]{figures/fig/Snipaste_2024-03-16_16-55-35.png}
        \caption{串口显示接受数据}
        \label{fig:6}
    \end{figure}
    \begin{figure}[H]
        \centering
        \includegraphics[width =.85\textwidth]{figures/fig/b0d6d7303cdcf7e6474b1f717be56ff.jpg}
        \caption{串口显示接受数据}
        \label{fig:6}
    \end{figure}


\subsection{三色灯闪烁调控}
    将IsFlickering参数设置为1表示开启闪烁,下发参数。可以看到串口接受了数据,硬件上的三色灯开始闪烁。
    \begin{figure}[H]
        \centering
        \includegraphics[width =.85\textwidth]{figures/fig/Snipaste_2024-03-16_16-59-17.png}
        \caption{串口显示接受数据}
        \label{fig:6}
    \end{figure}


    \begin{figure}[ht]
            \centering
            \begin{animateinline}[autoplay,loop]{6}
                    \multiframe{20}{i=1+1}{
                        \includegraphics[width=0.8\linewidth]{figures/fig/GIF1/\i.png}
                    }
            \end{animateinline}
            \caption{闪烁效果(请在JavaScript的PDF阅读器中查看)}
            \label{fig:animation}
    \end{figure}



\subsection{三色灯呼吸调控}
    将IsBreathing参数设置为1表示开启闪烁,下发参数。可以看到串口接受了数据,硬件上的三色灯开始闪烁。
    \begin{figure}[H]
        \centering
        \includegraphics[width =.85\textwidth]{figures/fig/Snipaste_2024-03-16_17-04-07.png}
        \caption{串口显示接受数据}
        \label{fig:6}
    \end{figure}

    \begin{figure}[ht]
        \centering
        \begin{animateinline}[autoplay,loop]{8}
                \multiframe{63}{i=1+1}{
                    \includegraphics[width=0.8\linewidth]{figures/fig/GIF2/\i.png}
                }
        \end{animateinline}
        \caption{呼吸效果(请在JavaScript的PDF阅读器中查看)}
        \label{fig:animation}
\end{figure}



\section{实验代码介绍}
有关宏名和参数定义如下:
\begin{lstlisting}[language=C++]
    //头文件
#include <string.h>
#include <ArduinoJson.h>

//User Modified Part
//WIFI,MQTT等参数设置
#define wifi_ssid     "xm"    
#define wifi_psw      "12090217"     
#define clientIDstr   "3color_light"
#define timestamp     "6666"
#define ProductKey    "k0v6olZyIU8"
#define DeviceName    "3color_light"
#define DeviceSecret  "ffcb212d922f0285f5cf08cc1495ed2b"
#define password      "55203F6BA2AAEEAD5421602929B1C3F127A70323"


//开关等宏
#define OFF           0
#define ON            1
#define MUTE          2
#define KEEP_OFF      2
#define KEEP_ON       3

//蜂鸣器宏
#define Buzzer_ON   digitalWrite(BuzzerPin,HIGH)
#define Buzzer_OFF  digitalWrite(BuzzerPin,LOW)


//AT指令宏设置
#define AT                    "AT\r"
#define AT_OK                 "OK"
#define AT_REBOOT             "AT+REBOOT\r"
#define AT_ECHO_OFF           "AT+UARTE=OFF\r"
#define AT_MSG_ON             "AT+WEVENT=ON\r"

#define AT_WIFI_START         "AT+WJAP=%s,%s\r"
#define AT_WIFI_START_SUCC    "+WEVENT:STATION_UP"

#define AT_MQTT_AUTH          "AT+MQTTAUTH=%s&%s,%s\r"
#define AT_MQTT_CID           "AT+MQTTCID=%s|securemode=3\\,signmethod=hmacsha1\\,timestamp=%s|\r"
#define AT_MQTT_SOCK          "AT+MQTTSOCK=%s.iot-as-mqtt.cn-shanghai.aliyuncs.com,1883\r"

#define AT_MQTT_AUTOSTART_OFF "AT+MQTTAUTOSTART=OFF\r"
#define AT_MQTT_ALIVE         "AT+MQTTKEEPALIVE=500\r"
#define AT_MQTT_START         "AT+MQTTSTART\r"
#define AT_MQTT_START_SUCC    "+MQTTEVENT:CONNECT,SUCCESS"
#define AT_MQTT_PUB_SET       "AT+MQTTPUB=/sys/%s/%s/thing/event/property/post,1\r"
#define AT_MQTT_PUB_ALARM_SET "AT+MQTTPUB=/sys/%s/%s/thing/event/GasAlarm/post,1\r"
#define AT_MQTT_PUB_DATA      "AT+MQTTSEND=%d\r"
#define JSON_DATA_PACK        "{\"id\":\"100\",\"version\":\"1.0\",\"method\":\"thing.event.property.post\",\"params\":{\"RoomTemp\":%d.%02d,\"AC\":%d,\"Fan\":%d,\"Buzzer\":%d,\"GasDetector\":%d}}\r"
#define JSON_DATA_PACK_2      "{\"id\":\"110\",\"version\":\"1.0\",\"method\":\"thing.event.property.post\",\"params\":{\"LightDetector\":%d,\"Curtain\":%d,\"Light\":%d,\"SoilHumi\":%d,\"Pump\":%d,\"eCO2\":%d,\"TVOC\":%d}}\r"
#define JSON_DATA_PACK_ALARM  "{\"id\":\"110\",\"version\":\"1.0\",\"method\":\"thing.event.GasAlarm.post\",\"params\":{\"GasDetector\":%d}}\r"
#define AT_MQTT_PUB_DATA_SUCC "+MQTTEVENT:PUBLISH,SUCCESS"
#define AT_MQTT_UNSUB         "AT+MQTTUNSUB=2\r"
#define AT_MQTT_SUB           "AT+MQTTSUB=2,/sys/%s/%s/thing/service/property/set,1\r"
#define AT_MQTT_SUB_SUCC      "+MQTTEVENT:2,SUBSCRIBE,SUCCESS"
#define AT_MQTT_CLOSE          "AT+MQTTCLOSE\r"

#define AT_BUZZER_MUTE           "\"Buzzer\":2"


#define DEFAULT_TIMEOUT       10   //默认超时时间
#define BUF_LEN               100  //缓冲区长度
#define BUF_LEN_DATA          190  //缓冲区数据长度

char      ATcmd[BUF_LEN];       //AT指令缓冲区
char      ATbuffer[BUF_LEN];    //AT返回数据缓冲区
char      ATdata[BUF_LEN_DATA]; //AT数据缓冲区
#define BuzzerPin             3 //蜂鸣器引脚
int   Buzzer = OFF;             //蜂鸣器状态

String data;  //串口数据
double frequency; 
int ColorGreen;
int ColorRed; 
int ColorBlue; 
int IsFlickering;
int IsBreathing;
StaticJsonDocument<300> doc; //创建一个json文档
\end{lstlisting}

串口初始化
\begin{lstlisting}[language=C++]
    void setup() {
  //Serial Initial
  Serial3.begin(115200);
  Serial.begin(115200);
  String inString="";
  pinMode(7,OUTPUT);  //RGB引脚
  pinMode(8,OUTPUT); 
  pinMode(9,OUTPUT); 
  pinMode(13,OUTPUT); //板载led
  
  
  //  Pin_init();
  BEEP(1);//蜂鸣器响一声
  
  //wifi连接和阿里云连接
  while(1)
  {
    if(!WiFi_init())continue;
    BEEP(2);
    if(!Ali_connect())continue;
    break;
  }
  BEEP(3);
}
\end{lstlisting}

主循环代码
\begin{lstlisting}[language=C++]
    void loop() {
    String inString;
    delay(10);

    //有串口的数据进来就暂存在inString里
    if (Serial3.available()>0){
      delay(10);
      inString=Serial3.readString();//读取串口数据
      if (inString!=""){
        data=inString;
      }
      Serial.println(data);
   }

    frequency=parse(data);   //解析数据并打印
    if (frequency==0) { frequency=1;}
    Serial.print("Frequency:");
    Serial.println(frequency);  
    int delay_seconds=1.0/frequency*1000; //单位为毫秒
    Serial.print("delay_seconds(ms):");
    Serial.println(delay_seconds);
    if(IsBreathing && IsFlickering){//防止两个同时为真
        IsBreathing==0;
        Serial.println("IsBreathing and IsFlickering can't be true at the same time,set IsBreathing to false");
    }

    //如果串口没有收到新的数据,就一直循环
    while(Serial3.available()==0) {

      if (IsBreathing){//呼吸灯实现
        for (int i = 0; i <= 20; i++)
        {
          analogWrite(9, 0+ColorBlue*i/20);
          analogWrite(7, 0+ColorGreen*i/20);
          analogWrite(8, 0+ColorRed*i/20);
          delay(delay_seconds/20);
        }
        for (int i = 20; i >= 0; i--)
        {
          analogWrite(9, 0+ColorBlue*i/20);
          analogWrite(7, 0+ColorGreen*i/20);
          analogWrite(8, 0+ColorRed*i/20);
          delay(delay_seconds/20);
        }
      }
      else if (IsFlickering){//闪烁灯实现
          delay(delay_seconds/2);
          analogWrite(9, ColorBlue);
          analogWrite(7, ColorGreen);
          analogWrite(8, ColorRed);

          delay(delay_seconds/2);
          analogWrite(9, 0);
          analogWrite(7, 0);
          analogWrite(8, 0);   
      }
      else{//常亮灯实现
        analogWrite(9, ColorBlue);
        analogWrite(7, ColorGreen);
        analogWrite(8, ColorRed);
      }
    }
  }
\end{lstlisting}

AT指令发送函数
\begin{lstlisting}[language=C++]
    //发送AT指令并检查返回值
bool check_send_cmd(const char* cmd,const char* resp,unsigned int timeout)
{
  int i = 0;
  unsigned long timeStart;
  timeStart = millis();
  cleanBuffer(ATbuffer,BUF_LEN);
  Serial3.print(cmd);
  Serial3.flush();//阻塞程序,等到所有发送完后程序再继续
  while(1)
  {
    while(Serial3.available())//如果串行端口有可用数据
    {
      ATbuffer[i++] = Serial3.read();//就读取数据并存入缓冲区
      if(i >= BUF_LEN)i = 0;//如果缓冲区满了,就从头开始覆盖。
    }
    if(NULL != strstr(ATbuffer,resp))break;
    if((unsigned long)(millis() - timeStart > timeout * 1000)) break;//超时
  }
  
  if(NULL != strstr(ATbuffer,resp))return true;
  return false;
}

\end{lstlisting}

Json解析函数
\begin{lstlisting}[language=C++]
    //设置了一个解析的函数,利用库ArdunioJson版本是6.2.13
double parse(String data){

  //找到“{”,并截断到倒数第二个字符
  int commaPosition;  
  commaPosition = data.indexOf('{');
  data= data.substring(commaPosition, data.length());
  Serial.println(data);
  
  // Deserialize the JSON document
  DeserializationError error = deserializeJson(doc, data);

  // 错误处理
  if (error) {
    Serial.print(F("deserializeJson() failed: "));
    Serial.println(error.f_str());
    return;
  }

  // 获取数据
  const char* method  = doc["method"];
  const char* id      = doc["id"];
  frequency  = doc["params"]["Frequency"];
  IsFlickering=doc["params"]["IsFlickering"];
  IsBreathing=doc["params"]["IsBreathing"];
  ColorRed=doc["params"]["ColorRed"];
  ColorGreen=doc["params"]["ColorGreen"];
  ColorBlue=doc["params"]["ColorBlue"];
  

  // 打印数据
  Serial.println("==============Start================");
  Serial.print("Frequency:");
  Serial.println(frequency);
  Serial.print("IsFlickering:");
  Serial.println(IsFlickering);
  Serial.print("IsBreathing:");
  Serial.println(IsBreathing);
  Serial.print("ColorRed:");
  Serial.println(ColorRed);
  Serial.print("ColorGreen:");
  Serial.println(ColorGreen);
  Serial.print("ColorBlue:");
  Serial.println(ColorBlue);
  return frequency;
}
\end{lstlisting}

WIFI连接函数
\begin{lstlisting}[language=C++]
    //该函数通过调用check_send_cmd函数发送AT指令并检查返回值,初始化WiFi模块
bool WiFi_init()
{
  bool flag;

  flag = check_send_cmd(AT,AT_OK,DEFAULT_TIMEOUT);
  if(!flag)return false;
  
  flag = check_send_cmd(AT_REBOOT,AT_OK,20);
  if(!flag)return false;
  delay(5000);

  flag = check_send_cmd(AT_ECHO_OFF,AT_OK,DEFAULT_TIMEOUT);
  if(!flag)return false;

  flag = check_send_cmd(AT_MSG_ON,AT_OK,DEFAULT_TIMEOUT);
  if(!flag)return false;
  
  cleanBuffer(ATcmd,BUF_LEN);//清理缓冲区,然后使用snprintf函数将WiFi启动的AT命令、WiFi的SSID和密码格式化到ATcmd缓冲区。
  snprintf(ATcmd,BUF_LEN,AT_WIFI_START,wifi_ssid,wifi_psw);//snprintf是一个C语言中的函数,用于将格式化的数据写入字符串
  flag = check_send_cmd(ATcmd,AT_WIFI_START_SUCC,20);
  return flag;
}
\end{lstlisting}

阿里云连接函数
\begin{lstlisting}[language=C++]
    //该函数通过调用check_send_cmd函数发送AT指令并检查返回值,进行阿里云连接
bool Ali_connect()
{
  bool flag;
  bool flag1;

  cleanBuffer(ATcmd,BUF_LEN);
  snprintf(ATcmd,BUF_LEN,AT_MQTT_AUTH,DeviceName,ProductKey,password);
  flag = check_send_cmd(ATcmd,AT_OK,DEFAULT_TIMEOUT);
  if(!flag)return false;

  cleanBuffer(ATcmd,BUF_LEN);
  snprintf(ATcmd,BUF_LEN,AT_MQTT_CID,clientIDstr,timestamp);
  flag = check_send_cmd(ATcmd,AT_OK,DEFAULT_TIMEOUT);
  if(!flag)return false;

  cleanBuffer(ATcmd,BUF_LEN);
  snprintf(ATcmd,BUF_LEN,AT_MQTT_SOCK,ProductKey);
  flag = check_send_cmd(ATcmd,AT_OK,DEFAULT_TIMEOUT);
  if(!flag)return false;

  flag = check_send_cmd(AT_MQTT_AUTOSTART_OFF,AT_OK,DEFAULT_TIMEOUT);
  if(!flag)return false;

  flag = check_send_cmd(AT_MQTT_ALIVE,AT_OK,DEFAULT_TIMEOUT);
  if(!flag)return false;

  flag = check_send_cmd(AT_MQTT_START,AT_MQTT_START_SUCC,20);
  if(!flag)return false;

  cleanBuffer(ATcmd,BUF_LEN);
  snprintf(ATcmd,BUF_LEN,AT_MQTT_PUB_SET,ProductKey,DeviceName);
  flag = check_send_cmd(ATcmd,AT_OK,DEFAULT_TIMEOUT);
  if(!flag)return false;

  //flag = check_send_cmd(AT_MQTT_UNSUB,AT_OK,DEFAULT_TIMEOUT);
  //if(!flag)return false;
  
  cleanBuffer(ATcmd,BUF_LEN);
  snprintf(ATcmd,BUF_LEN,AT_MQTT_SUB,ProductKey,DeviceName);
  flag = check_send_cmd(ATcmd,AT_MQTT_SUB_SUCC,DEFAULT_TIMEOUT);
  if(!flag){ BEEP(4);flag1 = check_send_cmd(AT_MQTT_CLOSE,AT_OK,DEFAULT_TIMEOUT);}
  return flag;
}
\end{lstlisting}


% \begin{lstlisting}[language=C++]
    
% \end{lstlisting}

%%----------- 参考文献 -------------------%%
%在reference.bib文件中填写参考文献,此处自动生成




\end{document}