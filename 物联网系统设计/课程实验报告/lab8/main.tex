%!TeX program = xelatex
\documentclass[12pt,hyperref,a4paper,UTF8]{ctexart}
\usepackage{zjureport}
\usepackage{listings}
\usepackage{enumitem}
\usepackage{float}
\usepackage{xcolor}
\lstset{
    %backgroundcolor=\color{red!50!green!50!blue!50},%代码块背景色为浅灰色
    basicstyle=\small\linespread{0.85}\selectfont, % 设置代码字体大小和行间距
    rulesepcolor= \color{gray}, %代码块边框颜色
    breaklines=true,  %代码过长则换行
    numbers=left, %行号在左侧显示
    numberstyle= \small,%行号字体
    keywordstyle= \color{blue},%关键字颜色
    commentstyle=\color{gray}, %注释颜色
    frame=shadowbox%用方框框住代码块
    }


%%-------------------------------正文开始---------------------------%%
\begin{document}

%%-----------------------封面--------------------%%
\cover

%%------------------摘要-------------%%
%\begin{abstract}
%
%在此填写摘要内容
%
%\end{abstract}

\thispagestyle{empty} % 首页不显示页码

%%--------------------------目录页------------------------%%
\newpage
\tableofcontents

%%------------------------正文页从这里开始-------------------%
\newpage

%%可选择这里也放一个标题
%\begin{center}
%    \title{ \Huge \textbf{{标题}}}
%\end{center}

\section{实验介绍}
\subsection{实验要求}

\begin{enumerate}
  \item 开发一个微信小程序,能够控制你的灯的开关,频率和三色变化。
  \item 小程序能够接受设备的温度和光照数据,并且能够实时显示。
\end{enumerate}
\subsection{实验器材}
\begin{itemize}[itemsep=-5pt, topsep=0pt, partopsep=0pt]
    \item Arduino mega 开发板及载板
    \item 阿里云平台
    \item 微信小程序开发工具
\end{itemize}


\section{实验过程}

\subsection{阿里云平台设置}
\begin{itemize}[itemsep=-5pt, topsep=0pt, partopsep=0pt]
    \item 新建产品和两个设备,一个用于微信小程序端,一个用于设备端。
    \begin{figure}[H]
      \centering
      \includegraphics[width =1.0\textwidth]{figures/fig/Snipaste_2024-04-06_15-35-57.png}
      \caption{新建产品和设备}
      \label{fig:enter-label}
    \end{figure}
    \item 设置产品的物模型,添加属性。
    \begin{figure}[H]
      \centering
      \includegraphics[width =1.0\textwidth]{figures/fig/Snipaste_2024-04-06_15-36-38.png}
      \caption{设置物模型}
      \label{fig:enter-label}
    \end{figure}
    
    \item 设置自定义topic,用于之后设备端和微信小程序端的通信。
    \begin{figure}[H]
      \centering
      \includegraphics[width =1.0\textwidth]{figures/fig/Snipaste_2024-04-06_15-36-51.png}
      \caption{设置自定义topic}
      \label{fig:enter-label}
    \end{figure}
    \item 参照下图进行云产品流程的设置
    \begin{figure}[H]
      \centering
      \includegraphics[width =0.85\textwidth]{figures/fig/Snipaste_2024-04-06_15-37-13.png}
      \caption{云产品流程图}
      \label{fig:enter-label}
    \end{figure}
    \item 创建云产品流转规则
    \begin{figure}[H]
      \centering
      \includegraphics[width =0.85\textwidth]{figures/fig/Snipaste_2024-04-06_15-38-16.png}
      \caption{创建云产品流转规则}
      \label{fig:enter-label}
    \end{figure}
    \item 编写SQL处理数据和添加转发数据操作
    \begin{figure}[H]
      \centering
      \includegraphics[width =0.85\textwidth]{figures/fig/Snipaste_2024-04-06_15-39-41.png}
      \caption{编写数据处理和转发数据}
      \label{fig:enter-label}
    \end{figure}
    \item 具体操作如下图。SQL语句选择Arduino上报的所有数据,然后转发到自定义的微信端topic。
    \begin{figure}[H]
      \centering
      \begin{minipage}{.5\textwidth}
        \centering
        \includegraphics[width=.8\linewidth]{figures/fig/Snipaste_2024-04-06_15-40-02.png}
        \caption{\small SQL创建}
        \label{fig:image1}
      \end{minipage}%
      \begin{minipage}{.5\textwidth}
        \centering
        \includegraphics[width=1.1\linewidth]{figures/fig/Snipaste_2024-04-06_15-40-12.png}
        \caption{\small 数据转发设置}
        \label{fig:image2}
      \end{minipage}
    \end{figure}
    \item 同理微信端到设备端的云产品流转规则如下图所示。
    \begin{figure}[H]
      \centering
      \includegraphics[width =0.85\textwidth]{figures/fig/Snipaste_2024-04-06_15-39-19.png}
      \caption{编写数据处理和转发数据}
      \label{fig:enter-label}
    \end{figure}

\end{itemize}
\subsection{硬件设备代码}
本次实验硬件代码基本同lab5中代码。新增了光敏电阻的读取,并修改了上传的数据。
增添的代码如下:
\begin{lstlisting}[language=C]
  //增加两个全局变量
  int LightSwitch=0;
  int photon=0; //光照强度
  //在parse函数中增加对LightSwitch的解析
  //增加对光敏电阻的读取
  photon = analogRead(A2);
  //update函数中修改为上传光照强度和温度
  //将阿里云链接的所需的productKey,deviceName,deviceSecret,clientID等信息改为ArduinoDev设备的信息
\end{lstlisting}
其余主体代码与lab5中相同,不多赘述。详细代码见附录。

\subsection{微信小程序端开发}

\begin{itemize}
  \item 注册微信小程序开发者账号,下载微信小程序开发工具,打开并导入学在浙大上给出的模板项目。
  \item 打开index.js文件,修改其中的设备信息,将设备的productKey,deviceName,deviceSecret等信息修改为之前创建的WechatControl设备的信息。注意将publish和subscribe的topic也修改为之前设置的自定义topic。
  \begin{figure}[H]
    \centering
    \includegraphics[width =1.0\textwidth]{figures/fig/Snipaste_2024-04-06_16-16-01.png}
    \caption{修改参数}
    \label{fig:enter-label}
  \end{figure}
  \item 在消息处理函数中,添加对Json数据的解析,将设备上报的数据显示在页面上。数据对应物模型中的属性名称。
  \begin{figure}[H]
    \centering
    \includegraphics[width =1.0\textwidth]{figures/fig/Snipaste_2024-04-06_16-24-59.png}
    \caption{编写Json数据解析}
    \label{fig:enter-label}
  \end{figure}

  \item 修改app.json文件和index.wxml文件,可以设置小程序的页面布局和样式。增添组件等。这里我修改了小程序的名称,改变了按钮的位置,添加了设置灯的颜色的按钮等。
  \item 修改对应的按钮的事件处理函数和发送的json数据,从而将需要的数据发送到设备端。
  \item \textbf{本人的微信小程序源码和详细注释详见报告附录}
  \begin{figure}[H]
    \centering
    \includegraphics[width =1.0\textwidth]{figures/fig/Snipaste_2024-04-06_16-37-10.png}
    \caption{最终效果图}
    \label{fig:enter-label}
  \end{figure}





  % \item 给滑条组件设置交互,将滑条的值传递给页面变量。页面变量的设置如下图所示。
  % \begin{figure}[H]
  %   \centering
  %   \begin{minipage}{.3\textwidth}
  %     \centering
  %     \includegraphics[width=.85\linewidth]{figures/fig/Snipaste_2024-04-05_23-55-22.png}
  %     \caption{\small 交互动作设置}
  %     \label{fig:image1}
  %   \end{minipage}%
  %   \begin{minipage}{.6\textwidth}
  %     \centering
  %     \includegraphics[width=1\linewidth]{figures/fig/Snipaste_2024-04-05_23-55-30.png}
  %     \caption{\small 页面变量}
  %     \label{fig:image2}
  %   \end{minipage}
  % \end{figure}

\end{itemize}


\section{实验结果展示}
\subsection{设备链接阿里云}
\begin{itemize}
  \item 在硬件设备上电,连接到阿里云平台。
  \item 在微信小程序开发工具中,编译项目,只要MQTT链接参数设置正确,设备就会连接到阿里云平台。
  \begin{figure}[H]
    \centering
    \includegraphics[width =0.85\textwidth]{figures/fig/Snipaste_2024-04-06_16-44-55.png}
    \caption{设备均已连接到阿里云平台}
    \label{fig:enter-label}
  \end{figure}
\end{itemize}

\subsection{微信小程序控制三色灯}
\begin{itemize}
  \item 首先打开微信小程序,拖动滑块,设置参数,点击开灯按钮,在console中可以看到如下图红框中的数据,包括拖动滑条设置参数的提示消息和微信小程序调用了publish方法,将数据发送到阿里云平台的消息。
  \begin{figure}[H]
    \centering
    \includegraphics[width =0.9\textwidth]{figures/fig/Snipaste_2024-04-06_16-46-43.png}
    \caption{微信小程序设置参数点击开灯}
    \label{fig:enter-label}
  \end{figure}
  \item 此时微信小程序调用了publish方法,将数据发送到阿里云平台,阿里云平台通过云产品流转规则转发到设备端,设备端解析数据,控制灯的开关和颜色。
  \item 下图红框圈出的三个日志分别是微信小程序发布数据到阿里云平台,阿里云平台接收到数据并云产品流转,阿里云将数据转发到设备端的过程。
  \begin{figure}[H]
    \centering
    \includegraphics[width =0.9\textwidth]{figures/fig/Snipaste_2024-04-06_16-53-04.png}
    \caption{云平台日志}
    \label{fig:enter-label}
  \end{figure}
 
  \begin{figure}[H]
    \centering
    \includegraphics[width =0.7\textwidth]{figures/fig/f5d98f34e3aac0ba1a308abf8e610cb.jpg}
    \caption{设备端效果图}
    \label{fig:enter-label}
  \end{figure}
  \item 设备端接收到数据,解析数据,控制灯的开关和颜色,上图为设备端的效果图。
\end{itemize}

\subsection{设备端上传,小程序显示}
\begin{itemize}
  \item 设备端已经写好代码,定时上传温度和光照强度数据,数据上传到阿里云平台,阿里云平台通过云产品流转规则转发到微信小程序端。
  \item 微信小程序端接收到数据,解析数据,显示在页面上。
  \item 在微信小程序中点击订阅按钮,此时会提示订阅成功,并调用subscribe方法,订阅设备端的数据。在console中可以看到如下图红框中的数据,包括订阅成功的提示消息和微信小程序接受到的数据。
  \begin{figure}[H]
    \centering
    \includegraphics[width =1\textwidth]{figures/fig/Snipaste_2024-04-06_17-01-40.png}
    \caption{微信小程序订阅数据}
    \label{fig:enter-label}
  \end{figure}
  \item 同样在阿里云平台的日志中可以看到数据的流转过程。如下图所示,框出的五个日志分别是设备端上传数据到阿里云平台,阿里云平台接收到数据并设置物模型属性,云产品流转,阿里云将数据转发到微信小程序端的过程,微信小程序端接收到数据返回的日志。
  \begin{figure}[H]
    \centering
    \includegraphics[width =1\textwidth]{figures/fig/Snipaste_2024-04-06_17-04-55.png}
    \caption{云平台日志}
    \label{fig:enter-label}
  \end{figure}
  \item 最终效果为,微信小程序端接收到数据,解析数据,显示在页面的按钮上,如前文图片所示。
\end{itemize}

\newpage


\appendix
\section{微信小程序源码}
index.js文件是微信小程序的主要逻辑代码,主要实现了与阿里云平台的连接,数据的发送和接收,页面的数据绑定等功能。
index.js源码和详细注释如下:
\begin{lstlisting}[language=python]
import mqtt from '../../utils/mqtt.js';  //引入mqtt.js库文件
const aliyunOpt = require('../../utils/aliyun/aliyun_connect.js');
var RGBtemp = [0, 0, 0, 0, 0];
let that = null; //记录this的指向
var ison = 0; //记录灯的开关状态的变量

//page是一个页面的实例对象,page对象的属性和方法可以通过this访问
Page({
  data: {//用于页面的数据绑定

    //设置温度值和湿度值 
    temperature: "",
    humidity: "",

    client: null,//记录重连的次数
    reconnectCounts: 0,//MQTT连接的配置
    options: {
      protocolVersion: 4, //MQTT连接协议版本
      clean: false,
      reconnectPeriod: 1000, //1000毫秒,两次重新连接之间的间隔
      connectTimeout: 30 * 1000, //1000毫秒,两次重新连接之间的间隔
      resubscribe: true, //如果连接断开并重新连接,则会再次自动订阅已订阅的主题(默认true)
      clientId: 'WechatControl',
      password: 'A35FEE5433054A8001EE247B5F315F2CF4EB41FC',
      username: 'WechatControl&k0v6onEdM6S',
    },

    aliyunInfo: {
      productKey: 'k0v6onEdM6S', //阿里云连接的三元组 ,请自己替代为自己的产品信息!!
      deviceName: 'WechatControl', //阿里云连接的三元组 ,请自己替代为自己的产品信息!!
      deviceSecret: 'b8ec96a34b1b17a13ff90b8b0f617bfe', //阿里云连接的三元组 ,请自己替代为自己的产品信息!!
      regionId: 'cn-shanghai', //阿里云连接的三元组 ,请自己替代为自己的产品信息!!
      pubTopic: '/k0v6onEdM6S/WechatControl/user/WechatControl', //发布消息的主题
      subTopic: '/k0v6onEdM6S/WechatControl/user/get', //订阅消息的主题

    },
  },

  //onLoad函数会在页面加载后执行
  onLoad: function () {
    //配置MQTT连接
    that = this;
    let clientOpt = aliyunOpt.getAliyunIotMqttClient({
      productKey: that.data.aliyunInfo.productKey,
      deviceName: that.data.aliyunInfo.deviceName,
      deviceSecret: that.data.aliyunInfo.deviceSecret,
      regionId: that.data.aliyunInfo.regionId,
      port: that.data.aliyunInfo.port,
    });
  
    console.log("get data:" + JSON.stringify(clientOpt));
    let host = 'wxs://' + clientOpt.host;
   //设置连接的参数
    this.setData({
      'options.clientId': clientOpt.clientId,
      'options.password': clientOpt.password,
      'options.username': clientOpt.username,
    })
    console.log("this.data.options host:" + host);
    console.log("this.data.options data:" + JSON.stringify(this.data.options));

    //调用mqtt.connect方法连接服务器
    this.data.client = mqtt.connect(host, this.data.options);
    //当连接服务器时的回调函数
    this.data.client.on('connect', function (connack) {
      wx.showToast({
        title: '连接成功'
      })
      console.log("连接成功");
    })

    //接收消息监听,当接受到消息时执行的回调函数
    this.data.client.on("message", function (topic, payload) {
      console.log(" 收到 topic:" + topic + " , payload :" + payload);
      if(topic=='/k0v6onEdM6S/WechatControl/user/get'){
      that.setData({
        //转换成JSON格式的数据进行读取
        temperature: JSON.parse(payload).items.Temperature.value,
        photores:JSON.parse(payload).items.Photon.value,
      })
    }
    })

    //服务器连接异常的回调
    that.data.client.on("error", function (error) {
      console.log(" 服务器 error 的回调" + error)

    })
    //服务器重连连接异常的回调
    that.data.client.on("reconnect", function () {
      console.log(" 服务器 reconnect的回调")

    })
    //服务器连接异常的回调
    that.data.client.on("offline", function (errr) {
      console.log(" 服务器offline的回调")
    })
  },
  //开灯按钮的事件处理函数
  onClickOpen() {
    ison=1;
    that.sendCommond('set',1);
  },
  //关灯按钮的事件处理函数
  onClickOff() {
    ison=0;
    that.sendCommond('set',0);
  },
  //订阅按钮的事件处理函数
  setcolor() {
    that.sendCommond('set',ison);
  },

  //下面是各个滑条的事件处理函数
  slider1change: function (e) {  //RED
    RGBtemp[1] = e.detail.value;
    console.log(`需要发送的RGB数据分别为:`, 'R:', RGBtemp[1], 'G:', RGBtemp[3], 'B:', RGBtemp[2], 'fre:', RGBtemp[4]);
    console.log(`slider1发生change事件,携带值为`, e.detail.value);
  },

  slider2change: function (e) {  //BLUE
    RGBtemp[2] = e.detail.value;
    console.log(`需要发送的RGB数据分别为:`, 'R:', RGBtemp[1], 'G:', RGBtemp[3], 'B:', RGBtemp[2], 'fre:', RGBtemp[4]);
    console.log(`slider2发生change事件,携带值为`, e.detail.value);
  },
  slider3change: function (e) { //Green
    RGBtemp[3] = e.detail.value;
    console.log(`需要发送的RGB数据分别为:`, 'R:', RGBtemp[1], 'G:', RGBtemp[3], 'B:', RGBtemp[2], 'fre:', RGBtemp[4]);
    console.log(`slider3发生change事件,携带值为`, e.detail.value);
  },

  slider4change: function (e) {  //Fre
    RGBtemp[4] = e.detail.value;
    console.log(`需要发送的RGB数据分别为:`, 'R:', RGBtemp[1], 'G:', RGBtemp[3], 'B:', RGBtemp[2], 'fre:', RGBtemp[4]);
    console.log(`slider3发生change事件,携带值为`, e.detail.value);
  },

  //发送数据的函数,被上面的按钮事件处理函数调用
  sendCommond(cmd,data) {

  let sendData = {
    id: '12233443',
    version: '1.0',
    params:{//设置要发送的数据
      ColorRed: RGBtemp[1],
      ColorGreen: RGBtemp[3],
      ColorBlue:RGBtemp[2],
      Frequency:RGBtemp[4],
      LightSwitch:data
    }
};
  
    //发布消息
    if (this.data.client && this.data.client.connected) {//判断是否连接服务器
      //发送数据并打印
      this.data.client.publish(this.data.aliyunInfo.pubTopic, JSON.stringify(sendData));
      console.log("************************")
      console.log(this.data.aliyunInfo.pubTopic)
      console.log(JSON.stringify(sendData))
    } else {
      wx.showToast({
        title: '请先连接服务器',
        icon: 'none',
        duration: 2000
      })
    }
  },

  //订阅按钮的事件处理函数
  subTopic() {
    //此函数是订阅的函数,因为放在访问服务器的函数后面没法成功订阅topic,因此把他放在这个确保订阅topic的时候已成功连接服务器
    //订阅消息函数,订阅一次即可 如果云端没有订阅的话,需要取消注释,等待成功连接服务器之后,在随便点击(开灯)或(关灯)就可以订阅函数
    this.data.client.subscribe(this.data.aliyunInfo.subTopic, function (err) {
      if (!err) {
        console.log("订阅成功");
        //that.page.subTopic.disabled = "true";
      };
      wx.showModal({
        content: "订阅成功",
        showCancel: false,
      })
    })

  }

})
  
\end{lstlisting}

index.wxml文件是微信小程序的页面布局代码,主要实现了页面的布局和样式。源码如下:
\begin{lstlisting}[language=html]
  <view class="page">
    <view class="button-sp-are">
      //设置了按钮的位置和样式
      <button class="weui-btn" type="primary" size="mini" bindtap="onClickOpen" style="position: relative; left: 78rpx; top: 529rpx; width: 445rpx; height: 51rpx; display: inline-block; box-sizing: border-box">开灯</button>
      <button type="warn" size="mini" bindtap="onClickOff" style="position: relative; left: 77rpx; top: 400rpx; width: 445rpx; height: 51rpx; display: inline-block; box-sizing: border-box">关灯</button>
      <button type="warn" size="mini" bindtap="setcolor" style="position: relative; left: 77rpx; top: 270rpx; width: 445rpx; height: 51rpx; display: inline-block; box-sizing: border-box">设置</button>

      <button type="warn" size="mini" bindtap="subTopic" style="position: relative; left: 78rpx; top: 130rpx; width: 445rpx; height: 51rpx; display: inline-block; box-sizing: border-box">订阅</button>
      <button size="mini" style="position: relative; left: 77rpx; top: 144rpx; width: 445rpx; height: 51rpx; display: inline-block; box-sizing: border-box">温度:{{temperature}}℃</button>
      <button size="mini" style="position: relative; left: 77rpx; top: 20rpx; width: 445rpx; height: 51rpx; display: inline-block; box-sizing: border-box">光照:{{photores}}</button>
    </view>
  
    <view class="section section_gap">
        //设置滑条的位置和样式
        <text class="section__title">设置频率</text>
        <view class="body-view">
            <slider bindchange="slider4change" min="0" max="100" show-value />
        </view>
    </view>

    <view class="page__bd">
        //设置三个滑条的位置和样式
        <view class="section section_gap">
            <text class="section__title">设置红色</text>
            <view class="body-view">
                <slider bindchange="slider1change" activeColor="#ff4a4a"  min="0" max="255" show-value/>
            </view>
        </view>

        <view class="section section_gap">
            <text class="section__title">设置蓝色</text>
            <view class="body-view">
                <slider bindchange="slider2change" activeColor="#4a4aff" min="0" max="255" show-value/>
            </view>
        </view>

        <view class="section section_gap">
            <text class="section__title">设置绿色</text>
            <view class="body-view">
                <slider bindchange="slider3change" activeColor="#4aff4a" min="0" max="255" show-value/>
            </view>
        </view>
    


    </view>
</view>
\end{lstlisting}

index.wxss文件是微信小程序的页面样式代码,主要实现了页面的样式。源码如下:
\begin{lstlisting}[language=html]
  button{
    margin:40px;
    font-size:10px;
    width:80%;
  }
  .userinfo-nickname{
    color :#aaa;
  }
  .button-sp-are{margin-top:60px;
  height:50rpx}
  .weui-btn{margin-top:50px}
\end{lstlisting}
其余文件为小程序的配置文件,不再赘述。

\section{硬件设备代码}
Arduino代码如下:
\begin{lstlisting}[language=C]

  //#include <Wire.h>
  //头文件
  #include <string.h>
  #include <ArduinoJson.h>
  #include <Adafruit_BMP280.h>
  
  // #define CLOSE_LIGHT //
  
  //————————————————————————————————————————————————————————————————————————
  #define My_JSON_PACK_1        "{\"id\":\"66666\",\"version\":\"1.0\",\"method\":\"thing.event.property.post\",\"params\":{\"Temperature\":%d.%02d,\"Photon\":%d}}\r"
  
  //————————————————————————————————————————————————————————————————————————
  //BMP280
  #define BMP_SCK  (13)
  #define BMP_MISO (12)
  #define BMP_MOSI (11)
  #define BMP_CS   (10)
  
  Adafruit_BMP280 bmp;
  //————————————————————————————————————————————————————————————————————————
  //User Modified Part
  //WIFI,MQTT等参数设置
  #define wifi_ssid     "**"    
  #define wifi_psw      "*******"     
  
  #define clientIDstr   "*******"
  #define timestamp     "*******"
  #define ProductKey    "*******"
  #define DeviceName    "*******"
  #define DeviceSecret  "*******"
  #define password      "*******"
  //————————————————————————————————————————————————————————————————————————
  
  
  
  
  
  //开关等宏
  #define OFF           0
  #define ON            1
  #define MUTE          2
  #define KEEP_OFF      2
  #define KEEP_ON       3
  
  
  //蜂鸣器宏
  #define Buzzer_ON   digitalWrite(BuzzerPin,HIGH)
  #define Buzzer_OFF  digitalWrite(BuzzerPin,LOW)
  
  
  //AT指令宏设置
  #define AT                    "AT\r"
  #define AT_OK                 "OK"
  #define AT_REBOOT             "AT+REBOOT\r"
  #define AT_ECHO_OFF           "AT+UARTE=OFF\r"
  #define AT_MSG_ON             "AT+WEVENT=ON\r"
  
  #define AT_WIFI_START         "AT+WJAP=%s,%s\r"
  #define AT_WIFI_START_SUCC    "+WEVENT:STATION_UP"
  
  #define AT_MQTT_AUTH          "AT+MQTTAUTH=%s&%s,%s\r"
  #define AT_MQTT_CID           "AT+MQTTCID=%s|securemode=3\\,signmethod=hmacsha1\\,timestamp=%s|\r"
  #define AT_MQTT_SOCK          "AT+MQTTSOCK=%s.iot-as-mqtt.cn-shanghai.aliyuncs.com,1883\r"
  
  #define AT_MQTT_AUTOSTART_OFF "AT+MQTTAUTOSTART=OFF\r"
  #define AT_MQTT_ALIVE         "AT+MQTTKEEPALIVE=500\r"
  #define AT_MQTT_START         "AT+MQTTSTART\r"
  #define AT_MQTT_START_SUCC    "+MQTTEVENT:CONNECT,SUCCESS"
  #define AT_MQTT_PUB_SET       "AT+MQTTPUB=/sys/%s/%s/thing/event/property/post,1\r"
  #define AT_MQTT_PUB_ALARM_SET "AT+MQTTPUB=/sys/%s/%s/thing/event/GasAlarm/post,1\r"
  #define AT_MQTT_PUB_DATA      "AT+MQTTSEND=%d\r"
  
  #define JSON_DATA_PACK        "{\"id\":\"100\",\"version\":\"1.0\",\"method\":\"thing.event.property.post\",\"params\":{\"RoomTemp\":%d.%02d,\"AC\":%d,\"Fan\":%d,\"Buzzer\":%d,\"GasDetector\":%d}}\r"
  #define JSON_DATA_PACK_2      "{\"id\":\"110\",\"version\":\"1.0\",\"method\":\"thing.event.property.post\",\"params\":{\"LightDetector\":%d,\"Curtain\":%d,\"Light\":%d,\"SoilHumi\":%d,\"Pump\":%d,\"eCO2\":%d,\"TVOC\":%d}}\r"
  #define JSON_DATA_PACK_ALARM  "{\"id\":\"110\",\"version\":\"1.0\",\"method\":\"thing.event.GasAlarm.post\",\"params\":{\"GasDetector\":%d}}\r"
  #define AT_MQTT_PUB_DATA_SUCC "+MQTTEVENT:PUBLISH,SUCCESS"
  #define AT_MQTT_UNSUB         "AT+MQTTUNSUB=2\r"
  #define AT_MQTT_SUB           "AT+MQTTSUB=2,/sys/%s/%s/thing/service/property/set,1\r"
  #define AT_MQTT_SUB_SUCC      "+MQTTEVENT:2,SUBSCRIBE,SUCCESS"
  #define AT_MQTT_CLOSE          "AT+MQTTCLOSE\r"
  
  #define AT_BUZZER_MUTE           "\"Buzzer\":2"
  
  
  #define DEFAULT_TIMEOUT       10   //默认超时时间
  #define BUF_LEN               100  //缓冲区长度
  #define BUF_LEN_DATA          190  //缓冲区数据长度
  
  char      ATcmd[BUF_LEN];       //AT指令缓冲区
  char      ATbuffer[BUF_LEN];    //AT返回数据缓冲区
  char      ATdata[BUF_LEN_DATA]; //AT数据缓冲区
  #define BuzzerPin             3 //蜂鸣器引脚
  int   Buzzer = OFF;             //蜂鸣器状态
  
  String data;  //串口数据
  float temperature=0; //温度
  int photon=0; //光照强度
  double frequency=1; 
  int delay_seconds;
  int ColorGreen=0;
  int ColorRed=0; 
  int ColorBlue=0; 
  int IsFlickering=0;
  int IsBreathing=0;
  int LightSwitch=0;
  StaticJsonDocument<300> doc; //创建一个json文档
  String inString;//串口数据
  
  void setup() {
    //Serial Initial
    Serial3.begin(115200);
    Serial.begin(115200);
    String inString="";
    pinMode(7,OUTPUT);  //RGB引脚
    pinMode(8,OUTPUT); 
    pinMode(9,OUTPUT); 
    pinMode(13,OUTPUT); //板载led
    
    //  BMP module test
    Serial.println(F("BMP280 Forced Mode Test."));
    if (!bmp.begin(0x76,0x58)) {
      Serial.println(F("Could not find a valid BMP280 sensor, check wiring or "
                        "try a different address!"));
      while (1) delay(10);
    }
    bmp.setSampling(Adafruit_BMP280::MODE_FORCED,     /* Operating Mode. */
                    Adafruit_BMP280::SAMPLING_X2,     /* Temp. oversampling */
                    Adafruit_BMP280::SAMPLING_X16,    /* Pressure oversampling */
                    Adafruit_BMP280::FILTER_X16,      /* Filtering. */
                    Adafruit_BMP280::STANDBY_MS_500); /* Standby time. */
  
    
    //  Pin_init();
    //BEEP(1);//蜂鸣器响一声
    
    //wifi连接和阿里云连接
    while(1)
    {
      if(!WiFi_init())continue;
     // BEEP(2);
      Serial.println("WiFi init success");
      if(!Ali_connect())continue;
      break;
    }
    //BEEP(3);
    Serial.println("Ali connect success");
  }
  
  
  //设置了一个解析的函数,利用库ArdunioJson版本是6.2.13
  double parse(String data){
  
    //找到“{”,并截断到倒数第二个字符
    int commaPosition;  
    commaPosition = data.indexOf('{');
    data= data.substring(commaPosition, data.length());
    Serial.println(data);
    
    // Deserialize the JSON document
    DeserializationError error = deserializeJson(doc, data);
  
    // 错误处理
    if (error) {
      Serial.print(F("deserializeJson() failed: "));
      Serial.println(error.f_str());
      return;
    }
  
    // 获取数据
    const char* method  = doc["method"];
    const char* id      = doc["id"];
    frequency  = doc["params"]["Frequency"];
    IsFlickering=doc["params"]["IsFlickering"];
    IsBreathing=doc["params"]["IsBreathing"];
    ColorRed=doc["params"]["ColorRed"];
    ColorGreen=doc["params"]["ColorGreen"];
    ColorBlue=doc["params"]["ColorBlue"];
    LightSwitch=doc["params"]["LightSwitch"];
  
    // 打印数据
    Serial.println("==============Start================");
    Serial.print("Frequency:");
    Serial.println(frequency);
    Serial.print("IsFlickering:");
    Serial.println(IsFlickering);
    Serial.print("IsBreathing:");
    Serial.println(IsBreathing);
    Serial.print("ColorRed:");
    Serial.println(ColorRed);
    Serial.print("ColorGreen:");
    Serial.println(ColorGreen);
    Serial.print("ColorBlue:");
    Serial.println(ColorBlue);
    Serial.print("LightSwitch:");
    Serial.println(LightSwitch);
    return frequency;
  }
    
  void loop() {
      delay(10);
      while (1)
      {
        if (bmp.takeForcedMeasurement()) {
          // can now print out the new measurements
          temperature=bmp.readTemperature();
          Serial.print(F("Temperature = "));
          Serial.print(bmp.readTemperature());
          Serial.println(" *C");
        } 
        else {
          Serial.println("Forced measurement failed!");
        }
        photon = analogRead(A2);
        Serial.print("Photon:");
        Serial.println(photon);
        if(Update()) Serial.println("Update success");
        Serial.println("============================");
        while(Serial3.available()){ //清楚串口缓冲区
          Serial3.read();
        }
  //闪烁灯
        unsigned long starmil=millis();
        unsigned long currmil;
        do{
            if (Serial3.available() > 0) {
                // 读取并处理串口数据
                handleSerialData();
              }
            if (LightSwitch){//闪烁灯实现
              delay(delay_seconds/2);
              analogWrite(9, ColorBlue);
              analogWrite(7, ColorGreen);
              analogWrite(8, ColorRed);
  
              delay(delay_seconds/2);
              analogWrite(9, 0);
              analogWrite(7, 0);
              analogWrite(8, 0);   
          }
          currmil=millis();
        }while (currmil-starmil<10000);
      }
    }
  
  
  void handleSerialData() {
    delay(10);
    inString = Serial3.readString();  // 读取串口数据
    if (inString != "") {
      // 在这里处理接收到的串口数据
      data=inString;
      Serial.println(inString);
    }
      frequency=parse(data);   //解析数据并打印
    if (frequency==0) { frequency=1;}
    Serial.print("Frequency:");
    Serial.println(frequency);  
    delay_seconds=1.0/frequency*1000; //单位为毫秒
    Serial.print("delay_seconds(ms):");
    Serial.println(delay_seconds);
    if(IsBreathing && IsFlickering){//防止两个同时为真
        IsBreathing==0;
        Serial.println("IsBreathing and IsFlickering can't be true at the same time,set IsBreathing to false");
    }
  }
  
  bool Update()
  {
    bool flag;
    int  frac;
    int zs;
    int len;
    cleanBuffer(ATcmd,BUF_LEN);
    snprintf(ATcmd,BUF_LEN,AT_MQTT_PUB_SET,ProductKey,DeviceName);
    flag = check_send_cmd(ATcmd,AT_OK,DEFAULT_TIMEOUT);
    if(!flag) {
      Serial.println("send AT_MQTT_PUB_SET false");
      return false;
    }
  
    cleanBuffer(ATdata,BUF_LEN_DATA);
    zs=(int) temperature;
    frac=(temperature-zs)*100;
    len = snprintf(ATdata,BUF_LEN_DATA,My_JSON_PACK_1,zs,frac,photon);
  
    cleanBuffer(ATcmd,BUF_LEN);
    snprintf(ATcmd,BUF_LEN,AT_MQTT_PUB_DATA,len-1);
    flag=0;
    flag = check_send_cmd(ATcmd,">",DEFAULT_TIMEOUT);
    if(flag) {
      flag = check_send_cmd(ATdata,AT_MQTT_PUB_DATA_SUCC,20);
      
    } 
    else {
      Serial.println("> accept wrong");
      return false;
    }
      return true;
  
  }
  
  
  
  //该函数通过调用check_send_cmd函数发送AT指令并检查返回值,进行阿里云连接
  bool Ali_connect()
  {
    bool flag;
    bool flag1;
  
    cleanBuffer(ATcmd,BUF_LEN);
    snprintf(ATcmd,BUF_LEN,AT_MQTT_AUTH,DeviceName,ProductKey,password);
    flag = check_send_cmd(ATcmd,AT_OK,DEFAULT_TIMEOUT);
    if(!flag) {
      Serial.println("send AT_MQTT_AUTH false");
      return false;
    }
  
  
    cleanBuffer(ATcmd,BUF_LEN);
    snprintf(ATcmd,BUF_LEN,AT_MQTT_CID,clientIDstr,timestamp);
    flag = check_send_cmd(ATcmd,AT_OK,DEFAULT_TIMEOUT);
    if(!flag){
      Serial.println("send AT_MQTT_CID false");
      return false;
    }
    cleanBuffer(ATcmd,BUF_LEN);
    snprintf(ATcmd,BUF_LEN,AT_MQTT_SOCK,ProductKey);
    flag = check_send_cmd(ATcmd,AT_OK,DEFAULT_TIMEOUT);
    if(!flag){
      Serial.println("send AT_MQTT_SOCK false");
      return false;
    }
    flag = check_send_cmd(AT_MQTT_AUTOSTART_OFF,AT_OK,DEFAULT_TIMEOUT);
    if(!flag){
      Serial.println("send AT_MQTT_AUTOSTART_OFF false");
      return false;
    }
    flag = check_send_cmd(AT_MQTT_ALIVE,AT_OK,DEFAULT_TIMEOUT);
    if(!flag){
      Serial.println("send AT_MQTT_ALIVE false");
      return false;
    }
    flag = check_send_cmd(AT_MQTT_START,AT_MQTT_START_SUCC,50);
    if(!flag){
      Serial.println("send AT_MQTT_START false");
      return false;
    }
    cleanBuffer(ATcmd,BUF_LEN);
    snprintf(ATcmd,BUF_LEN,AT_MQTT_PUB_SET,ProductKey,DeviceName);
    flag = check_send_cmd(ATcmd,AT_OK,DEFAULT_TIMEOUT);
    if(!flag){
      Serial.println("send AT_MQTT_PUB_SET false");
      return false;
    }
    cleanBuffer(ATcmd,BUF_LEN);
    snprintf(ATcmd,BUF_LEN,AT_MQTT_SUB,ProductKey,DeviceName);
    flag = check_send_cmd(ATcmd,AT_MQTT_SUB_SUCC,DEFAULT_TIMEOUT);
    // if(!flag){ BEEP(4);flag1 = check_send_cmd(AT_MQTT_CLOSE,AT_OK,DEFAULT_TIMEOUT);}
    return flag;
  }
  
  
  //该函数通过调用check_send_cmd函数发送AT指令并检查返回值,初始化WiFi模块
  bool WiFi_init()
  {
    bool flag;
  
    flag = check_send_cmd(AT,AT_OK,DEFAULT_TIMEOUT);
    if(!flag)return false;
    
    flag = check_send_cmd(AT_REBOOT,AT_OK,20);
    if(!flag)return false;
    delay(5000);
  
    flag = check_send_cmd(AT_ECHO_OFF,AT_OK,DEFAULT_TIMEOUT);
    if(!flag)return false;
  
    flag = check_send_cmd(AT_MSG_ON,AT_OK,DEFAULT_TIMEOUT);
    if(!flag)return false;
    
    cleanBuffer(ATcmd,BUF_LEN);//清理缓冲区,然后使用snprintf函数将WiFi启动的AT命令、WiFi的SSID和密码格式化到ATcmd缓冲区。
    snprintf(ATcmd,BUF_LEN,AT_WIFI_START,wifi_ssid,wifi_psw);//snprintf是一个C语言中的函数,用于将格式化的数据写入字符串
    flag = check_send_cmd(ATcmd,AT_WIFI_START_SUCC,20);
    return flag;
  }
  
  
  //发送AT指令并检查返回值
  bool check_send_cmd(const char* cmd,const char* resp,unsigned int timeout)
  {
    int i = 0;
    unsigned long timeStart;
    timeStart = millis();
    cleanBuffer(ATbuffer,BUF_LEN);
    Serial3.print(cmd);
    Serial3.flush();//阻塞程序,等到所有发送完后程序再继续
    while(1)
    {
      while(Serial3.available())//如果串行端口有可用数据
      {
        ATbuffer[i++] = Serial3.read();//就读取数据并存入缓冲区
        if(i >= BUF_LEN)i = 0;//如果缓冲区满了,就从头开始覆盖。
      }
      if(NULL != strstr(ATbuffer,resp))break;
      if((unsigned long)(millis() - timeStart > timeout * 1000)) break;//超时
    }
    
    if(NULL != strstr(ATbuffer,resp))return true;
    return false;
  }
  
  void cleanBuffer(char *buf,int len)
  {
    for(int i = 0;i < len;i++)
    {
      buf[i] = '\0';
    } 
  }
  
  void BEEP(int b_time)
  {
    for(int i = 1;i <= b_time;i++)
    { 
      digitalWrite(BuzzerPin,HIGH);
      delay(100);
      digitalWrite(BuzzerPin,LOW);
      delay(100);
    }
  }
  void Buzzer_mute()
  {
    Buzzer_OFF;
    Buzzer = MUTE;
  }
    
\end{lstlisting}  

%%----------- 参考文献 -------------------%%
%在reference.bib文件中填写参考文献,此处自动生成




\end{document}