%!TeX program = xelatex
\documentclass[12pt,hyperref,a4paper,UTF8]{ctexart}
\usepackage{zjureport}
\usepackage{listings}
\usepackage{enumitem}
\usepackage{float}
\usepackage{xcolor}
\lstset{
    %backgroundcolor=\color{red!50!green!50!blue!50},%代码块背景色为浅灰色
    rulesepcolor= \color{gray}, %代码块边框颜色
    breaklines=true,  %代码过长则换行
    numbers=left, %行号在左侧显示
    numberstyle= \small,%行号字体
    keywordstyle= \color{blue},%关键字颜色
    commentstyle=\color{gray}, %注释颜色
    frame=shadowbox%用方框框住代码块
    }


%%-------------------------------正文开始---------------------------%%
\begin{document}

%%-----------------------封面--------------------%%
\cover

%%------------------摘要-------------%%
%\begin{abstract}
%
%在此填写摘要内容
%
%\end{abstract}

\thispagestyle{empty} % 首页不显示页码

%%--------------------------目录页------------------------%%
\newpage
\tableofcontents

%%------------------------正文页从这里开始-------------------%
\newpage

%%可选择这里也放一个标题
%\begin{center}
%    \title{ \Huge \textbf{{标题}}}
%\end{center}

\section{实验介绍}
\subsection{实验要求}

\begin{enumerate}[itemsep=-5pt, topsep=0pt, partopsep=0pt]
    \item 利用XCOM串口调试助手,借助AT指令获取当前时间。
    \item 利用XCOM模拟设备端,借助AT指令连接阿里云,描述设备端如何和阿里云进行连接的。
    \item 执行AT指令,完成属性订阅和属性发布。监控并截图阿里云数据的格式。
\end{enumerate}
\subsection{实验平台}
\begin{itemize}[itemsep=-5pt, topsep=0pt, partopsep=0pt]
    \item 阿里云物联网平台
    \item Arduino硬件平台
    \item XCOM串口调试助手
    \item \href{https://help.aliyun.com/document_detail/73742.html?spm=a2c4g.11186623.6.540.3f7e3e3b3z3z3z}{AT指令参考文档}
\end{itemize}


\section{任务一:借助AT指令获取当前时间}
\subsection{实验过程}
\subsection*{硬件连接}
    将载板上的拨码开关拨成上左下左的形式,此时WIFI模块通过串口与Arduino通信。
    将一个空的Arduino程序烧写入Arduino板,以防止之后电脑与WIFI通信串口被占用。\\
    \begin{figure}[H]
        \centering
        \includegraphics[width =.45\textwidth,angle=90]{figures/fig/460627bbf49bf7a7182da6b8d972fd7.jpg}
        \caption{连接硬件}
        \label{fig:enter-label}
    \end{figure}
    \begin{figure}[H]
        \centering
        \includegraphics[width =.85\textwidth]{figures/fig/Snipaste_2024-03-15_18-51-13.png}
        \caption{烧写空程序}
        \label{fig:enter-label}
    \end{figure}
    先断开连接,将载板上的拨码开关拨成上左下左的形式,此时WIFI模块通过串口与电脑直接通信。
    \begin{figure}[H]
        \centering
        \includegraphics[width =.45\textwidth,angle=90]{figures/fig/18adb66e2eac966a3a1331601309e81.jpg}
        \caption{连接硬件}
        \label{fig:enter-label}
    \end{figure}

    \subsection*{串口通信}
\begin{itemize}[itemsep=-5pt, topsep=0pt, partopsep=0pt]
    \item 打开XCOM串口调试助手,选择正确的串口和波特率
    \item 发送AT指令,收到OK表示和WIFI模块通信正常
    \item 发送AT+UARTE=ON,打开指令回显
    \item 手机打开热点,发送AT+CWJAP="热点名称","密码",WIFI模块连接上热点,
    \item 发送AT+WEVENT=ON开启状态通知,发送AT+REBOOT重启模块
    \begin{figure}[H]
        \centering
        \includegraphics[width =.85\textwidth]{figures/fig/Snipaste_2024-03-15_19-17-48.png}
        \caption{发送指令连接热点}
        \label{fig:enter-label}
    \end{figure}    
    \begin{figure}[H]
        \centering
        \includegraphics[width =.85\textwidth]{figures/fig/Snipaste_2024-03-15_19-18-54.png}
        \caption{发送指令连接热点}
        \label{fig:enter-label}
    \end{figure} 
    
    \item 发送AT+SNTPCFG=+8,ntp.aliyun.com,设置时区为+8,NTP服务器为阿里云
    \item 发送AT+SNTPTIME,通过NTP服务器获取当前时间(该操作需要连接网络,因此需要先连接热点才能成功接受)
    \item 发送AT+RTCGET,可以查看当前设备的RTC时间
    \begin{figure}[H]
        \centering
        \includegraphics[width =.85\textwidth]{figures/fig/Snipaste_2024-03-15_19-20-23.png}
        \caption{借助AT指令获取当前时间}
        \label{fig:6}
    \end{figure}  

  
\end{itemize}




\subsection{实验结果}
\begin{itemize}
    \item 通过图\ref{fig:6},成功通过AT指令获取到了当前的时间
\end{itemize}




\section{任务二:利用XCOM模拟设备端,借助AT指令连接阿里云}

\subsection{实验过程}
\begin{itemize}[]
    \item 在阿里云平台准备好对应的设备,通过三元组信息生成好相关的连接参数
    \begin{figure}[H]
        \centering
        \includegraphics[width =.75\textwidth]{figures/fig/Snipaste_2024-03-15_19-34-44.png}
        \caption{准备好设备,生成连接参数}
        \label{fig:enter-label}
    \end{figure}

    \item 开始MQTT连接:下文的参数均由生成器生成 
    \item 输入AT+MQTTAUTH=username,password
    \item 输入AT+MQTTCID=ClientID (主要分割符的,需要加上$\backslash$来转义 )
    \item 输入AT+MQTTSOCK=连接域参数
    \begin{figure}[H]
        \centering
        \includegraphics[width =.75\textwidth]{figures/fig/Snipaste_2024-03-15_19-40-53.png}
        \caption{使用对应的AT指令输入MQTT相关参数}
        \label{fig:enter-label}
    \end{figure} 

    \item 输入AT+MQTTAUTOSTART=OFF 关闭上电自动连接
    \item 输入AT+MQTTKEEPALIVE=500 设置心跳包时间
    \item 输入AT+MQTTSTART 开始连接,可以看到返回了connet succes下信息
    \begin{figure}[H]
        \centering
        \includegraphics[width =.8\textwidth]{figures/fig/Snipaste_2024-03-15_19-41-09.png}
        \caption{MQTT连接阿里云}
        \label{fig:enter-label}
    \end{figure}

\end{itemize}
\subsection*{连接过程描述}
    首先通过串口给WiFi芯片发送AT指令,控制WIFI芯片连接上手机热点,这样WIFI芯片便可以通过手机热点访问互联网。再通过串口发送AT指令控制WiFi芯片通过MQTT协议和阿里云建立连接即可。

\subsection{实验结果}
    可以在XCOM串口助手中看到返回的消息,表示成功连接上阿里云





\section{任务三:执行AT指令,完成属性订阅和属性发布}
\subsection{sub订阅实验过程}

\begin{itemize}[itemsep=-5pt, topsep=0pt, partopsep=0pt]
    \item 输入AT+MQTTSUB=0,topic类,1 订阅物模型。其中第一个数字是订阅的消息编号,按顺序自行设置,第二个数字是选择使用QoS0 $\sim$ 协议
    \item 在阿里云在线调试中设置参数值下发
    \begin{figure}[H]
        \centering
        \includegraphics[width =.95\textwidth]{figures/fig/Snipaste_2024-03-15_20-10-16.png}
        \caption{通过AT指令进行属性订阅}
        \label{fig:enter-label}
    \end{figure}



\end{itemize}

\subsection{sub订阅消息实验结果}
如上图所示,当设置阿里云参数下发后,WIFI模块成功接受到了阿里云下发的数据,并在串口助手中显示了出来,串口显示的数据如图所示。


\subsection{pub发布消息实验过程}

\begin{itemize}
    \item 首先输入AT指令设置发布信息:AT+MQTTPUB=topic类,1
    \item 输入AT+MQTTSEND=170表示开始发布,后面的数字是$\{ \}$和内部字符个数
    \item 输入AT+MQTTSEND马上发送要发布的json数据
    \begin{figure}[H]
        \centering
        \includegraphics[width =.9\textwidth]{figures/fig/Snipaste_2024-03-15_20-33-17.png}
        \caption{通过AT指令发布消息}
        \label{fig:enter-label}
    \end{figure}

\end{itemize}

\subsection{pub发布消息实验结果}
如上图所示,可以在设备上直接看到接受到的本地发布的数据。


%%----------- 参考文献 -------------------%%
%在reference.bib文件中填写参考文献,此处自动生成




\end{document}